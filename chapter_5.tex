\chapter{Meet Your New Boss}
\fancyhead[CO]{\emph{Meet Your New Boss}}
\fancyhead[CE]{Freedom Beckons}

\section{Retirement}
So far, I've focused on financial independence and rarely mentioned retirement. Retirement is a choice that is only available to those who are financially independent. Many who are trying to become financially independent also want to retire early. Retirement is an inevitability for most people, and financial independence is required no matter what age you choose to retire. There are also many issues specific to retirement that are relevant for those who only want financial independence without early retirement. The two subjects are very closely related, so retirement deserves its own chapter.

When people think of retirement, many imagine a life of lethargy. There's a clich\'e image of a golf course in Florida, or a hammock on the beach. Some people picture a retirement home. Why would anyone want to retire, if that's all they have to look forward to? I don't play golf, and I don't like the beach very much. I prefer my retirement to reflect my own passions, not some stereotype.

As I explained in the Introduction, I define retirement as \emph{permanently severing the connection between labor and income.} This is my own definition, for the purpose of this book. It's the only definition that draws a distinction between unemployment, sabbatical, and changing careers, and it has no age restrictions.

Financial independence offers choices. It's not an all-or-nothing proposition. You could work only part time. You could take a lower-paying, but more enjoyable job. You could volunteer. You could take a sabbatical. You could go back to school and take your time exploring a new career. You could consult occasionally, or take a seasonal job for fun and some extra cash. You could quit any time you want, and go to the beach or play golf or ride a motorcycle or travel or go for a hike on a beautiful day or read books or spend time with friends and family. You're only limited by your own imagination.

I've seen the word \emph{semi-retirement} used to describe this partial form of retirement. Your retirement is yours to do with as you please, so call it whatever you want. Retirement is a state of mind. It helps some people to think of themselves as retired, even if they still accept money for their labor. 

For me personally, full retirement is the logical conclusion of financial independence. Selling my time feels too much like selling myself, setting aside my own values and passions and temporarily taking on someone else's. During that time, employers all too often feel entitled to treat me however they like. The knowledge that no one else has a lease on my life, not even a single minute of it, affords me the kind of freedom I've craved ever since childhood. When I retired, I felt like I could fully embrace life, because I knew it was all mine.

But that was my choice. You get to decide for yourself why you want financial independence, and what you would like to do with it once you have it. Nonetheless, here are some of my perspectives on retirement.

\section{Whim and Will}
Many people use their jobs as a self-discipline crutch. They are given deadlines, and they know people are counting on them. The stakes are high because there will be dire financial consequences if they slack off. Even when they really don't want to work, they force themselves to do it anyway. They become so accustomed to this pressure that they fear what would happen to them if they retired, and suddenly found themselves without the structure and discipline of working full-time. After decades of this kind of life, many retirees feel lost, and go back to work out of sheer boredom. Structuring their own time and looking within themselves for motivation are important skills retirees must have.

There are many analogues between employment and retirement. I still feel like I have a boss. I call him Whim. Whim expects me to work days, nights, and weekends. I'm always on-call. The pay is mediocre, but the benefits are great. I work from home every day, my hours are totally flexible, and I set my own deadlines. Best of all, I love my job. My job is to follow my passions and curiosities, wherever they lead. It's an important job, and I take it very seriously.

Whim is the most demanding boss I've ever had. He reminds me every day that the stakes are life or death. Do I want to live my life, or am I just waiting to die? When I worked full-time, it was easy to neglect the things that mattered most to me. Some things I had to put off until Saturday or Sunday, but most things I put off for an eighth day that never came: Someday.

Now that I'm retired, every day is Someday. I can't get away with putting things off because, if not now, when? This forced me to be much more serious about the trade-offs I make with my time. I no longer live under the delusion that Someday will come and give me plenty of time to live my life. This now-or-never attitude led me to learn more about self-discipline.

I was always lousy at self-discipline, which I define as \emph{the ability to force yourself to do something you don't want to do.} I've always believed self-discipline makes people mechanical. I figured, if I really don't want to do something, then why would I do it? It's my life. And if I really \emph{want} something, then why would I need to force myself? Wouldn't I just do it? Self-discipline is entirely unnecessary, and could actually mean I'm spending time doing things I don't \emph{really} want to do. So I vowed to only do things I want to do, and devoted my life to Whim.

Many motivational speakers teach that self-discipline is like a muscle. The more you use it, the more you have. They say things like, ``resolve today to\ldots'' or ``make a commitment to\ldots'' As if it's just a matter of deciding to do something and then doing it. If it was really that easy, Alcoholics Anonymous wouldn't exist.

Forcing yourself to do something you don't want to do requires a tremendous amount of willpower. Willpower is not a muscle. Each person has a limited supply, which becomes even more limited when they're stressed or tired. This is why, for example, judges give harsher sentences for trials they rule on later in the day. They lack the willpower they had in the morning to be as careful and meticulous about weighing the evidence. It's also why people tend to be exhausted after working all day, and just want to do something easy. Working full-time takes a lot of willpower. They've used up all their willpower working for someone else, and then they have very little left over for themselves.

Achieving financial independence also requires willpower. This can be a catch-22: you have to work, depleting your willpower, which you need to achieve financial independence, so you don't have to work. I had to learn to conserve my willpower as much as possible. Here's how.

I've found that it takes a lot more willpower to do something I don't want to do than something I do. As much as possible, I avoided self-discipline. If I used the word ``should'' to describe my attitude about something, I was probably using self-discipline. When I really don't want to do something, I'd procrastinate more. This requires a lot of willpower, just to not think about the thing I ``should'' be doing. By the time I get around to it, often close to the deadline, my willpower would be drained, I'd be miserable, and I usually did a lousy job on it.

The highly successful people that motivational speakers love to praise for being ``disciplined'' don't actually use a lot of self-discipline. Instead, they're highly motivated, which is self-help jargon that obscures its true meaning: \emph{they really want what they want.} They aren't forcing themselves to do things they don't want to do, because they \emph{want} to do what they do, very badly.

I focused on what I \emph{do} want, not what I don't. The mind tends to create what it thinks about. This is called the \emph{self-fulfilling prophecy.} New Age spiritualists call it the \emph{law of attraction,} but there's nothing esoteric about it. It's just a natural capacity of the mind. If you think about what you don't want, you may unconsciously work to bring it about. It's a lot more effective and motivating to focus on what you do want.

I've found it takes more willpower to do something after I've just finished doing something that takes less willpower. We naturally have a kind of inertia. When we're at rest, we like to stay at rest, and when we're busy, it's easier to stay busy. I carefully arrange my day so that I get the hard stuff out of the way before I relax and have fun. When I do have fun, I just have fun. I don't worry about everything I need to do, because worrying takes willpower.

I've discovered it takes more willpower to do something new than something old. The bigger the change, the more willpower it takes. Human beings are creatures of habit, but this varies from person to person. Some people thrive on new things, and can change habits quickly. These are the few people who can quit destructive habits ``cold turkey.'' Some habits can \emph{only} be changed suddenly, which makes them more resistant to change. Fortunately, most habits can be changed slowly. When done in this way, I find that the changes I make are fairly painless.

I've noticed there are three stages I go through whenever I start a new habit, or quit an old one. The first is the \emph{honeymoon stage.} It's brand new, and the brain loves novelty. I'm excited about the results I might see from the change. It takes less willpower in this stage, because I'm highly motivated. The second is \emph{boredom.} The novelty has worn off, and it becomes a bland routine. I start really craving my old habit. Since change takes a long time, I become discouraged because I'm not seeing results. This is the stage that requires most of the willpower, and is when I'm most likely to give up. The third is \emph{habituation.} This is when I've become so used to it that it's the new normal. It still takes some willpower, but it may actually require more willpower to quit.

One important habit I needed for financial independence is \emph{thinking long-term.} This is a bit different from a behavioral habit. It's habit of thought, which requires mindfulness. I have to watch my mind, and notice its cravings. I have to think about my conflicting desires, short-term vs. long-term, and ask myself what I \emph{really} want. My mind might say, ``chocolate donut.'' My old habit is to just head to the pantry and stuff a donut in my mouth. Instead, I have to think about what I want long-term. If what I \emph{really} want is to be healthy, then I'll skip the donut. I must be honest about what I really want. If I \emph{really} want to be healthy, then the donut won't seem so interesting by comparison. But if I really want the donut, then I usually find a way to get it, whenever I'm low on willpower.

This long-term thinking is often associated with self-discipline, when it's usually called \emph{delayed gratification.} But this isn't self-discipline as I've defined it because it's not forcing yourself to do something you don't want to do. You're only making a conscious trade-off between two conflicting wants, and you choose the greater of the two wants. Long-term goals like financial independence take time to grow and come to fruition. Anything that involves slow growth requires planting seeds and waiting. Fortunately, the delayed gratification is only temporary.

Consider this: if you plant one seed a day, then you'll have to delay gratification every day only until your first harvest. Once you harvest your first plant, then you can harvest another one every day thereafter, as long as you continue planting one seed a day. You don't really have to delay gratification indefinitely. You just need to pay that upfront cost, and then you can go right back to instant gratification. That's good news, since it takes a lot of willpower to delay gratification indefinitely.

It's not a moral failing to prefer a short-term want over a long-term want. Sometimes there's nothing better than a delicious chocolate donut. Besides, the whole point is to live a happy and fulfilling life. I can't imagine a fulfilling life without chocolate. What matters is that you make your choices with your eyes open. Understand the trade-offs, and choose wisely. Be honest with yourself. Much is at stake.

Ultimately, all wants are short-term wants. Every choice you make can only be made at the moment it arises. That's short-term. You have to decide what is most important to you right now. That's why I say Whim is my boss. He's in charge of what I really want in life. He guides how I use my willpower. Whim can work together with willpower. They make a great team.

\section{Withdrawal rate}
Whim pays me a small salary, and can sometimes be stingy about raises. The compensation package I get is called the \emph{withdrawal rate.} This is something I've alluded to a few times already, but I'll go into more detail now. The withdrawal rate is the amount of money you spend each year divided by your liquid net worth.

\begin{center}
Withdrawal rate = Annual expenses $\div$ Liquid net worth
\end{center}

In Chapter 4, you learned how to develop spreadsheets to calculate your monthly expenses and liquid net worth. Now you have everything you need to calculate your withdrawal rate, which will serve as a gauge for your progress toward financial independence.

Using the example in Chapter 4, let's say you add up all your expenses, and you find that you spend \$2,500 a month. Then you add up all of your liquid assets and subtract any money you owe, and you find that your liquid net worth is \$156,500. Your withdrawal rate would be:

\begin{center}
Withdrawal rate = \$2,500/month $\times$ 12 months $\div$ \$156,500 = 19.16\%
\end{center}

Your goal for financial independence is to decrease this withdrawal rate over time. You do this by saving more money (income, Chapter 1, and investing, Chapter 3) and spending less (frugality, Chapter 2). For example, suppose you cut your costs by \$100/month, and you've stashed away \$2,000:

\begin{center}
Withdrawal rate = \$2,400/month $\times$ 12 months $\div$ \$158,500 = 18.17\%
\end{center}

You will have achieved financial independence when you've dropped your withdrawal rate down to or below your target withdrawal rate. Your target withdrawal rate is how much of your money you plan to spend each year, and is essentially how much you expect your investments to earn on average each year, minus inflation. It depends on three factors: how long you will need to take withdrawals, what kinds of investments you have, and how much you're willing to risk that your money will run out.

Research has been conducted on which withdrawal rates are safe for various situations.\cite{trinity-study} What they do is back-test various investment strategies to give you some range of withdrawal rates and how likely they have been in the past to ensure that your money survives. In other words, in some past stock and bond market scenarios, at any given withdrawal rate, you'd have run out of money before your life expectancy. The research can't tell you what the future holds, but it can give you some range of probabilities based on history.

A good starting point is 4\%. Based on historical returns of a 50\% stock / 50\% bonds portfolio, a withdrawal rate of 4\% has lasted for 30 years in 95\% of the cases.\cite{trinity-study} Let's break that down.

Historical returns are useful for getting a sense for what the markets are capable of, but things change. Historically, we didn't understand as much about economics. We didn't have Modern Portfolio Theory. There was no such thing as mutual funds, let alone ETFs. Diversification meant owning a few stocks and a few bonds, and never rebalancing. We were tied to the gold standard, so when disaster struck, all we could do was ride it out. We didn't fully appreciate just how much the economy feeds on itself---hard times breeds mistrust and a rush to safety, which leads to more hard times. Riding it out meant a protracted recession, which could only be resolved by the government taking drastic measures.

The 50/50 portfolio used in this model is very conservative, but poorly diversified. It's now possible to squeeze much better performance from your investments with lower volatility by increasing the stock exposure but diversifying your stocks and bonds between dozens of different asset classes and regularly rebalancing, as described in Chapter 3. It is important to keep volatility down once you start taking withdrawals, but a 50/50 portfolio like this is just not necessary to control volatility if you diversify properly.

30 years is a typical time frame for those who retire at the standard age of 55--65. If you plan to retire earlier, then you need more than 30 years. The longer you will be taking withdrawals, the lower your withdrawal rate needs to be. On the other hand, part of your portfolio won't be touched for much of this time, so you can afford a less conservative portfolio, using dividends for a large portion of your withdrawals rather than selling your investments.

95\% is very good odds. The other 5\% are those rare periods in history that were just disasterous, such as the Great Depression. Economists and the Federal Reserve have learned a lot since then about how to protect against such disasters. Of course, you might beg to differ after watching what happened in 2010. This was a crash unlike any we've seen since 1929, and there was a huge recession, but maybe this is the exception that proves the rule: the recovery was swift and thorough compared to The Great Depression that followed 1929, which took decades and a major world war to recover from.

Nonetheless, there are some who believe that everything is different now, and 4\% is no longer a conservative withdrawal rate. Discussing their reasoning gives us an opportunity to explore some of the issues with withdrawal rates.

Perhaps one of the most obvious arguments is that we live longer than we did in the past, so our money has to last longer than 30 years. This would be an even greater issue for those who retire early. However, it's also been argued that 30 years accounts for all of the historical anomolies, so a portfolio that can last 30 years can last indefinitely. If you don't buy this argument, or if you just want to be conservative, you should choose a lower target withdrawal rate.

The withdrawal rate is very sensitive to inflation. Historically, inflation has been unpredictable, but on average it has been about 3\% per year.\cite{inflationdata.com} Some believe inflation may be higher than this in the future. I believe this fear is unfounded. If you look at historical inflation rates, you'll see that it was extremely volatile, and quite often uncontrollably high. It has become more steady in recent years, due to the Fed gaining a better understanding about how to control it.\cite{multpl-inflation} In Chapter 3, I warned against underestimating inflation. Here, I warn against overestimating it. I could be wrong about this, so if you believe inflation will go up, then you should choose a lower target withdrawal rate.

The withdrawal rate is also very delicate at the beginning, when you start taking withdrawals. This was explained in Chapter 3, and illustrated with a stark example. If the market crashes right after you start taking withdrawals, it will be very difficult to recover your investments. Likewise, if there's a huge surge in your investments right after you start taking withdrawals, then any crashes that happen later will be easier to weather since you've built up a lot of buffer. This sensitivity at the beginning can make a flat number like 4\% seem almost arbitrary, since it really depends on \emph{when} you start taking withdrawals.

The 4\% withdrawal rate makes an assumption of constant and consistent withdrawals, what people sometimes call a ``fixed income.'' This is fine when your investments are doing well, but after a crash, withdrawing the same amount as before can be disasterous. This too was part of the stark example in Chapter 3. In that example, the same amount was withdrawn every year, no matter how bad things got. It's possible to be more flexible in your spending, and respond to the shifting winds of the economy. This would allow you to safely have a higher target withdrawal rate.

Some also believe stocks and bonds aren't going to perform as well as they have historically. They say that stocks are uncertain and the Fed has kept interest rates extremely low.\cite{schwab-withdrawal-rate} When have stocks ever been certain? Their uncertainty is the \emph{reason} stocks have always performed well historically. However, the interest rates are indeed very low, and if this continues to be true long-term, then the performance of the bond portion of your portfolio will suffer, and you will need a lower target withdrawal rate to compensate.

In some ways, the 4\% guideline is not safe enough. We need to take withdrawals longer than we did in the past, bond rates are extremely low, stocks and bonds might not perform as well as they had in the past, and it's extremely sensitive to the performance of your investments in the first few years of taking withdrawals. In other ways, it is too safe. It survived in 95\% of the scenarios tested, making it extremely unlikely to fail. It allowed a retiree to take constant and consistent withdrawals, even when the economy was in the toilet and everyone was out of work. It was built on a 50\%/50\% portfolio since there was no better way to diversify.

So, as you can see, this whole business of choosing the perfect withdrawal rate is fraught with complications. Each person's target withdrawal rate will be different depending on when and how they retire. Without a crystal ball, it's impossible to know ahead of time what was the right withdrawal rate for each person.

You might decide they all balance each other out, and 4\% is still a good guideline. You can also use some techniques to enhance its safety, which I will explore in this chapter. If you want to play it safe, you should choose a lower target withdrawal rate. However, this will take you longer to achieve financial independence. This is the trade-off you must consider. I can't offer a magical withdrawal rate that will work for everyone. You must weigh all the issues and decide which withdrawal rate to use.

In any case, your target withdrawal rate is only a guess about what the future holds. The future always turns out different than people expect. Plan the best you can, and use reasonable guesses to facilitate that, but remember that you're only guessing. You have to be ready to adjust your plans as things change. Just don't get too paranoid. At some point, you need to trust that you've planned the best you could.

\section{Your new paycheck}
One common feature of jobs that people think goes away with retirement is the paycheck. Retired people get paychecks just like everyone else, although they may look a little different. In retirement, ``paychecks'' are called ``withdrawals.''

Once you start taking withdrawals, your ``income'' is simply your target withdrawal rate multiplied by your liquid net worth. For example, if your withdrawal rate is 4\%, and your liquid net worth is \$750,000, then your annual income is \$30,000 per year ($4\% \times \$750,000$).

You should have a cash account of some kind to take withdrawals from throughout the year. This could be your checking account, savings account, money market fund, or \emph{laddered CDs.} Laddering CDs means that you have one Certificate of Deposit for each month. As each CD matures, it gets deposited into your checking account. So, you'd have a 1-month CD, a 2-month CD, a 3-month CD, a 4-month CD, etc., up to a 12-month CD. Each year, you'd open up 12 more, or you could just open a new 12-month CD every month. The idea is that since you know you won't need that money until the maturity date, you can take advantage of the higher rates CDs theoretically offer.

However, I've found that CDs don't offer very high rates these days. I've always been able to find savings accounts that offer higher rates than even a 1-year CD. So, why bother with all the complexity of laddered CDs when I can get a higher rate in a simple savings account, with better liquidity?

If you choose to use a savings account, what you could do is deposit one year of living expenses, or maybe a little less, since you'll be earning interest on it throughout the year. Each month, you could make deposits into your checking account from your savings account, and that would be your paycheck. Each year or so, you should rebalance your portfolio. You will be selling some of your assets and buying others. That's the time to replenish your savings account or CDs.

Don't worry about running out of cash before the year is up. It's fine to sell assets that have appreciated higher than your target allocation. You should also have a good amount of short-term, investment-grade bonds in your portfolio, which are much less volatile than other investments. They aren't quite as safe and stable as a savings account, but it doesn't hurt to pull from it occasionally when you need the cash.

Another option is to sell stocks when they're higher than your target allocation. This is just a partial rebalancing. This is what I do when I need cash.

\section{Getting raises}
Since it's impossible to know ahead of time exactly what your withdrawal rate should have been, and the future \emph{will} be different than you predicted, this means that your withdrawal rate \emph{will be wrong.} If you choose a conservative enough withdrawal rate and the future turns out brighter than you predicted, then you will end up with more money than you expected. In other words, your boss gives you a raise!

The traditional withdrawal method is to simply record your liquid net worth on the day you start withdrawing from it, and withdraw only 4\% of that each year. No matter what your investments do or what your withdrawal rate becomes, you keep your spending even. This works fine except for periods of high volatility and high inflation.

After you've been taking withdrawals, you may find that your liquid net worth grows over time, and your withdrawal rate is dropping. Some of this is to be expected, since the stock market can be very volatile, so your money may shrink again, causing a spike in your withdrawal rate that restores it to its target, or goes over the target for a while. But if your liquid net worth continues to grow and your withdrawal rate keeps dropping, at some point it's good to assume that your target withdrawal rate was too low, and you're free to spend some more.

On the other hand, if you notice your liquid net worth shrinking, and your withdrawal rate climbing, after a while that could be a sign that the initial withdrawal rate you chose was too high. This is dangerous, because at some point, you'll run out of money. There are only two ways to fix this: spend less or earn more. This will be harder to do after several years of over-spending.

You could adjust your expenses every month as your withdrawal rate changes, but this is difficult because, between surprise expenses and market volatility, your withdrawal rates will jump around like a pogo stick, causing you to over-react. You'll go from feast to famine and back again. How can you know what is safe to withdrawal if your withdrawal rate won't stay still? There are a few approaches.

One method is to have a buffer. Whenever your withdrawal rate drops below your target, set aside a portion of your ``raise.'' This will make it less likely for your withdrawal rate to go over your target. For example, let's say one month, your withdrawal rate is 3.9\%. You increase your expenses to bring it up to 3.95\%. Later, it drops to 3.8\%. You increase your spending to bring it up to 3.85\%. Then a crash happens, and your withdrawal rate shoots back up to 4\%, and you're still safe.

Another method, suggested in the book \emph{Work Less, Live More,} is called the ``95\% rule.'' Whenever your withdrawal rate goes over 4\%, drop your expenses to 95\% of what you spent last year. The author found that it's fairly safe to spend that much, even during bear markets. It also means that you will slowly drop your expenses during protracted recessions.

Both of these approaches are simple and should work well to protect your money during recessions without requiring huge, sudden changes to your lifestyle. You can use either method, or both combined. The method I use is a more complex variant of the buffer method.

I don't just have a single target withdrawal rate, but a range. If it drops below the lower target, that means I'm being too conservative, and can increase my expenses. If it goes over the upper target, that means that I'm starting to get into trouble and need to get very serious about lowering my expenses. This range encapsulates what I believe to be normal volatility, and prevents me from over-reacting to it.

For my lower target, I started with 4\%. Every month, I drop this target a little bit. This moving target serves as my buffer. At the time of this writing, my lower target is at 3.72\%, and my withdrawal rate is 3.68\%. When my withdrawal rate drops below the target, I allow myself to spend more to get it back to the target. These are my ``raises.'' If my lower target ever drops to 2.5\%, then I'm going to stop dropping it and keep my spending at that level. That seems like more than enough buffer. If a major catastrophe comes in either my spending or investments, I will reset the lower target, consuming the buffer I've built up over the years.

My upper target is 5\%. My withdrawal rate is normally well below 4\%, so I don't feel concerned if it occasionally goes slightly higher. This kind of volatility is normal. But if it ever hits 5\% or higher, that raises an alarm. It's a sign that I'm in danger of running out of money too early. At this point, I either have to manage my spending and investments to get the withdrawal rate back under 5\%, or I have to go back to work. The plan is to do the former, and never the latter.

\section{Health insurance}
\emph{Health care} is deceptively named. It's actually illness care. Aside from annual exams, most people don't use it much unless they get sick.

Caring for your health is mostly up to you, by how you eat, work, and live. The best way to save on illness care costs is to decrease your need for it, which has a lot to do with how you live. Genetics and environment are major factors, but there's not much you can do about those.

\emph{Health insurance} is also deceptively named. Most health insurance is actually health maintenance contracts. Actual insurance covers you in rare cases of financial hardship. True health insurance wouldn't come into play unless you become seriously ill or injured. Such plans do exist, and they have low premiums, although their premiums are much higher ever since the Affordable Health Care Act took effect.

If you are getting health insurance from your employer, then that is a hidden part of your total employment compensation package. This is income that will go away when you leave your job, so it's something you will need to make sure you provide for yourself. You should estimate how much you'll need to pay for health care once you're financially independent, and include that cost in your expenses for your withdrawal rate calculation.

The Affordable Health Care Act brought many changes to health insurance in America. It has solved some of the problems that used to plague early retirees, but it has also made health insurance more expensive. It's impossible to predict and plan for every contingency, but health care is one of the likeliest culprits of raining on your financial independence parade, if you don't plan for it well enough.

There are several components of health insurance that you should understand.

The \emph{premium} is the amount you must pay every month to be in the plan. Most employers will pay for part or all of this, while retirees must foot the entire bill themselves.

The \emph{deductible} is the amount you must pay out-of-pocket each year before the insurance will pay for anything. Most plans have some exceptions to this, such as medications and annual physicals.

A \emph{co-pay} is an amount you must pay out-of-pocket for every doctor visit, after you've met your deductible.

\emph{Co-insurance} is a percentage you must pay out-of-pocket, after you've met your deductible and co-pay.

The \emph{out-of-pocket limit} is the maximum amount you are liable for each year. Once you've paid this, the insurance plan will pay for everything else.

The \emph{policy maximum} is the maximum amount the insurance will pay. If the plan has paid this much, then they will stop paying. This means that you essentially have no more health insurance, and you're on the hook for everything. This limit is usually extremely high---on the order of millions. If it's lower than that, do not buy the insurance.

It's extremely important when evaluating health plans to check its special policies for prescription drugs and preventive care such as periodic health exams. They may charge separately for these. They don't count toward the deductible, or else they might have special co-pays.

All of these variables make shopping for health insurance very difficult. It complicates the process of comparing health insurance plans side-by-side. Nonetheless, if you do not have employer-sponsored health care, you must decide on an individual health insurance plan that best meets your needs.

I prefer health insurance with a high deductible. This has the effect of shifting the responsibility of most my health costs to me. This doesn't mean my health care is more expensive. It's cheaper because I'm not paying for health care I don't use through high premiums. But it does mean that I feel it more when I need to spend money on health care. This is exactly what I want because my goal is to stay conscious of the trade-offs that come with spending.

Health care costs go up as we age. How do you handle that? The same way you handle everything else: by planning for it. This planning isn't much different for younger people than for older people. If you retire at 65, how can you know what your health costs will be when you're 85? You really can't. You just have to use what you do know and make predictions, with some buffer in your planning for unforeseeable circumstances.

Once you're old enough to qualify for Medicare, your health insurance costs will drop again. However, Medicare is limited, so you should still plan to have health insurance, which is called Medi-gap insurance.

Youth offers two advantages for health care planning that older retirees lack. One advantage is more time to plan, prepare, and save. It's far easier to prepare for aging when you're young than when you're already old. Another advantage---another form of preparation, really---is that you can take good care of your health while you still have it, and it's easier to make these changes if you don't have to spend so much stress and time on working.

\section{Proof of income}
I've heard all the dire warnings, so many that I devoted the entire next chapter to it. Yeah, but\ldots what if you run out of money? Yeah, but\ldots what about health insurance? Yeah, but\ldots what if the stock market crashes? None of them have been much of a problem for me. Well, yes, the stock market crashing was a problem, but mostly an emotional one. Long term, it was not a strategic problem.

There was only one thing that posed a serious problem. It was something I didn't expect, was never warned about, and have never even heard mentioned in financial independence circles: \emph{proof of income.} This especially became a problem after 2010, when cash was the only thing anyone felt they could really trust anymore. It was an over-reaction to the heady 2000's, when anyone with a pulse could get a mortgage.

We live in a world that is very accustomed to income taking the form of a check that we receive from someone, somewhere, on a regular basis, in regular amounts---paychecks and pensions, mostly. Take this away, and it's almost as if your income has vanished as far as they can see. They can't wrap their heads around how you can possibly pay your bills if not from a regular check.

The first time I experienced this was after the interest rates dropped. Banks were nagging me to refinance, and the numbers looked really good. Just by moving my mortgage to a different bank, I could drop my interest payment significantly. As part of this process, they asked for paycheck stubs. I told them that I was retired and lived on my investments---I didn't receive a paycheck.

Fine, they said, how about a tax return? Tax returns are calculating your tax bill, and not designed for the purpose of accurately reflecting your cash flow. Because of the timing of when I sold my investments, and many deductions, my taxable income was zero. That doesn't mean I have no income, only that I have none that the federal government is interested in taxing. But as far as the bank was concerned, this made my income invisible. They didn't care how many assets I had, or that I have a decade-long, documented history of paying a mortgage on time every month. None of that mattered. I've tried two or three banks, and they all said the same thing. No refinance for me.

This wasn't a big deal, but I realized it would be if I hadn't bought my home before I retired. So, that's my first warning to you. Get your mortgage \emph{before} you retire, and make sure it's a mortgage you don't mind sticking with. It's likely the last mortgage you will ever get.

But the problem went even deeper. The next issue I had was when I wanted to upgrade my house before putting it on the market. This would require a large, temporary infusion of cash. Because the stock market is volatile in the short-term, and because of transaction fees and taxes, it's best to avoid tapping such investments for large amounts of temporary cash. This is exactly why banks offer home equity loans. I figured, this is a much smaller loan than a mortgage, so it shouldn't be much of a problem. Again, the bank declined the loan, and I was forced to sell stocks to pay for the upgrades.

This problem continued to haunt me. Next I wanted a temporary apartment I could stay in while the work is being done. I only needed it for a few months. Besides, my credit score was extremely high, and I had paid my mortgage faithfully for over 10 years. I had a lot of assets, and I could even pay my entire lease upfront. Again, none of this mattered. The apartment managers only cared about one thing: a steady check coming in every month. I had to find a co-signer in order to make this work, which just seemed absurd to me.

The truth is, financial independence is so much safer than paychecks. People often get fired or laid off. Companies go out of business. Just because a check has been coming in doesn't mean it will continue to. Assets, on the other hand, don't just vanish like that. Even if the stock market crashes, the only risk for retirees is long-term solvency. There's still plenty to pay the bills for a while. If these organizations knew how to properly assess credit risk, they would look at assets first, and treat paychecks with suspicion. The only reason they don't is because paychecks are far more common, so it's what they're used to dealing with.

I suppose I could move a consistent amount from my brokerage fund to checking account, thereby creating the illusion of a steady income stream. This would require keeping large portions of my portfolio in cash, and would make it much harder to strategically rebalance my portfolio. This may or may not work, and it really just amounts to social engineering---keeping the bean counters happy---if, or when, I ever need to prove my income to someone in the future. This feels too much like ``jumping through hoops.'' I've never liked doing busy work to impress people. That's why I've always hated exams, job interviews, and annual salary reviews. It's a big reason I wanted to retire in the first place.

I'm a problem-solver. I've found solutions to most of the problems I've faced in my quest for freedom. This is one I just haven't solved yet. If someone doesn't want to trust me, there's not much I can do about it. We can't change other people, or how organizations operate. All we can do is try not to rely on them, which is very difficult.

\newpage
\section{Resources}
\begin{itemize}
\item \textbf{\url{http://www.ehealthinsurance.com/}} -- This is an online health insurance broker that is very easy to use and compare providers. 
\item \textbf{\url{http://www.madfientist.com/}} -- Provides tools to help you track your progress toward financial independence.
\end{itemize}
