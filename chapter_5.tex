\chapter{Meet Your New Boss}

\section{Retirement}
So far, I've focused primarily on financial independence, and rarely mentioned retirement. Retirement is a choice that is only available to those who are financially independent, so it's relevant, even if you choose to work till the day you die. When people think of retirement, many imagine a life of lethargy. There's a clich\'e image of a golf course in Florida, or a hammock on the beach. Some people picture a retirement home. Why would anyone want to retire, if that's what they have to look forward to? I don't play golf, and I don't like the beach very much. I prefer my retirement to reflect my own passions, not some stereotype.

As I explained in the Introduction, I define retirement as \emph{permanently severing the connection between labor and income.} This is my own definition, for the purposes of this book. It's the only definition that draws a distinction from unemployment, sabbatical, or changing careers, and has no age restrictions.

Financial independence offers choices. It's not an all-or-nothing proposition. You could work only part time. You could take a lower paying, but more enjoyable job. You could volunteer. You could take a sabbatical. You could go back to school and take your time exploring a new career. You could consult occasionally, or take a seasonal job for fun an some extra cash. You could quit any time you want, and go to the beach or play golf or ride a motorcycle or travel or go for a hike on a beautiful day or read books or spend time with friends and family. You're only limited by your own imagination.

I've seen the word \emph{semi-retirement} used to describe this partial form of retirement. Your retirement is yours to do with as you please, and you're free to call it whatever you like. Retirement is a state of mind. It helps some people to think of themselves as retired, even if they still accept money for their labor. 

For me, full retirement is the logical conclusion of financial independence. Selling my time feels too much like selling myself, setting aside my own values and passions and temporarily taking on someone else's. During that time, employers all too often feel entitled to treat me however they like. The knowledge that no one else has a lease on my life, not even one minute of it, affords me the kind of freedom I've craved since childhood. When I retired, I stopped holding back, and fully embraced life, because I knew it was all mine.

\section{Whim and Will}
Many people use their jobs as a self-discipline crutch. They know people are counting on them, and they know there will be dire financial consequences if they slack off. Even when they really don't want to work, they force themselves to do it anyway. They become so accustomed to this pressure that they fear what would happen to them if they retired, and suddenly found themselves without the structure and discipline of working full-time. After decades of this kind of life, many retirees feel lost, and go back to work out of sheer boredom. Structuring your own time and looking within yourself for motivation are important skills retirees must have.

There are many analogues between employment and retirement. For example, I still feel like I have a boss. His name is Whim. Whim expects me to work days, nights, and weekends. I'm always on-call. The pay is lousy, but the benefits are great. I work from home every day, my hours are totally flexible, and I set my own deadlines. Best of all, I love my job. My job is to follow my passions and curiosities, wherever they lead. It's an important job, and I take it very seriously.

Whim is the most demanding boss I've ever had. He reminds me every day that the stakes are very high: life or death. Do I want to live my life, or am I just waiting to die? When I worked full-time, it was easy to neglect the things that mattered most to me. Some things I had to put off until Saturday or Sunday, but most things I put off for an eighth day that never came: Someday.

Now that I'm retired, every day is Someday. It's now or never. I can't get away with putting things off, because, if not now, when? This forced me to be much more serious about the trade-offs I make in life. I no longer live under the delusion that Someday will come and give me plenty of time to live my life. This now-or-never attitude led me to learn more about self-discipline.

I was always lousy at self-discipline. I define it as \emph{the ability to force yourself to do something you don't want to do.} I've always believed self-discipline makes people mechanical. I figured, if I really don't want to do something, then why would I do it? It's my life. And if I really want something, then why would I need to force myself? Wouldn't I just do it? Self-discipline is entirely unnecessary, and could actually mean I'm spending time doing things I don't \emph{really} want to do. So I vowed to only do things I want to do, and devoted my life to Whim.

Many motivational speakers teach that self-discipline is like a muscle. The more you use it, the more you have. They say things like, ``resolve today to\ldots'' or ``make a commitment to\ldots'' As if it's just a matter of deciding to do something. If it were that easy, there would be very few smokers, and Alcoholics Anonymous wouldn't exist.

Forcing yourself to do something you don't want to do requires a tremendous amount of willpower. Willpower is not a muscle. Each person has a limited supply, which becomes even more limited when they're stressed or tired. This is why, for example, judges give harsher sentences for trials they rule on later in the day. They lack the willpower they had in the morning to be as careful and meticulous about weighing the evidence. It's also why people tend to be exhausted after working all day, and just want to do something easy. Working full-time takes tremendous willpower. They've used up all their willpower working for someone else, and then they have very little left over for themselves.

Achieving financial independence requires willpower. This can be a catch-22: you have to work, which depletes your willpower, which you need to achieve financial independence, so you don't have to work. You'll have to learn to conserve your willpower as possible. The following is what helped me do this.

It takes a lot more willpower to do something you don't want to do than something you do. As much as possible, avoid self-discipline. If you use the word ``should'' to describe your attitude about something, you're probably using self-discipline. When you really don't want to do something, you'll procrastinate more. This requires a lot of willpower, just to not think about the thing you ``should'' be doing. By the time you get around to it, often close to the deadline, your willpower will be drained, you'll be miserable, and you'll probably do a lousy job on it.

The highly successful people that motivational speakers love to praise for being ``disciplined'' don't actually use a lot of self-discipline. Instead, they're highly motivated, which is self-help jargon that obscures its true meaning: \emph{they really want what they want.} They aren't forcing themselves to do things they don't want to do because they want to do what they do, very badly.

Think about what you do want, not what you don't. The mind tends to create what it thinks about. This is called the \emph{self-fulfilling prophecy.} New Age spiritualists call it the \emph{law of attraction,} but there's nothing esoteric about it. It's just a natural capacity of the mind. If you think about what you don't want, you may unconsciously work to bring it about. It's a lot more effective and motivating to focus on what you do want.

It takes more willpower to do something after you've just finished doing something that takes less willpower. We naturally have a kind of inertia. When we're at rest, we like to stay at rest, and when we're busy, it's easier to stay busy. Imagine trying to clean your house after coming home from a party. It's a lot easier to clean the house and then go to the party. Try to arrange your day so that you get the hard stuff out of the way before you relax and have fun. When you do have fun, just have fun. Don't worry about everything you need to do, because worrying takes willpower.

It takes more willpower to do something new than something old. The bigger the change, the more willpower it takes. Human beings are creatures of habit, but this varies from person to person. Some people thrive on new things, and can change habits quickly. These are the few people who can quit destructive habits ``cold turkey.'' Some habits can only be changed suddenly, which makes them more resistant to change. Fortunately, most habits can be changed slowly. When done in this way, it should feel fairly painless.

I've found there are three stages I go through whenever I start a new habit, or quit an old one. The first is the \emph{honeymoon stage.} It's brand new, and the brain loves novelty. I'm excited about the results I might see from the change. It takes less willpower in this stage, because I'm highly motivated. The second is \emph{boredom.} The novelty has worn off, and it becomes a bland routine. Since change takes a long time, I become discouraged because I'm not seeing results. This is the stage that requires most of the willpower, and is when I'm most likely to give up. The third is \emph{habituation.} This is when I've become so used to it that it's the new normal. It still takes some willpower, but it may actually require more willpower to quit.

One important habit you'll need for financial independence is \emph{thinking long-term.} This is a bit different than a behavioral habit. It's habit of thought, which requires mindfulness. You have to watch your mind, and notice its cravings. You have to think about your conflicting desires, short-term vs. long-term, and ask yourself what you \emph{really} want. Your mind might say, ``chocolate donut.'' Your old habit is to just head to the pantry and stuff a donut in your mouth. Instead, you have to think about what you want long-term. If what you \emph{really} want is to be healthy, then you'll skip the donut. You must be honest about what you really want. If you \emph{really} want to be healthy, then the donut won't seem so interesting by comparison. But if you really want the donut, you'll find a way to get it, whenever you're low on willpower.

This long-term thinking is often associated with self-discipline, when it's usually called \emph{delayed gratification.} This isn't self-discipline as I've defined it, because it's not forcing yourself to do something you don't want to do. You're only making a conscious trade-off between two conflicting wants, and you choose the greater of the two wants. Long-term goals like financial independence take time to grow and come to fruition. Anything that involves slow growth requires planting seeds, and waiting. Fortunately, the delayed gratification is only temporary. Consider this: if you plant one seed a day, then you'll have to delay gratification every day only until your first harvest. Once you harvest your first plant, then you can harvest another one every day thereafter, as long as you continue planting one seed a day. You don't really have to delay gratification indefinitely. You just need to pay that upfront cost, and then you can go right back to instant gratification. That's good news, since it takes a lot of willpower to delay gratification indefinitely.

It's not a moral failing to prefer a short-term want over a long-term want. Sometimes there's nothing better than a delicious chocolate donut. Besides, the whole point is to live a happy and fulfilling life. How is it possible to have a fulfilling life without chocolate? What matters is that you make your choices with your eyes open. Understand the trade-offs, and choose wisely. Be honest with yourself. Much is at stake.

Ultimately, all wants are short-term wants. Every choice you make can only be made at the moment it arises. That's short-term. You have to decide what is most important to you right now. That's why I say Whim is my boss. He's in charge of what I really want in life. He guides how I use my willpower. Whim can work together with willpower. They make a great team.

\section{Withdrawal rate}
Whim pays me a very small salary, and is pretty stingy about raises. The compensation package I get is called the \emph{withdrawal rate.} This is something I've alluded to a few times already, but I'll go into more detail now. The withdrawal rate is the amount of money you spend each year, divided by your liquid net worth.

\begin{center}
Withdrawal rate = Annual expenses $\div$ Liquid net worth
\end{center}

If your liquid net worth is zero or less, then your withdrawal rate is undefined.

In Chapter 4, you learned how to develop spreadsheets to calculate your monthly expenses and liquid net worth. Now you have everything you need to calculate your withdrawal rate, which is the most important number for your quest toward financial independence. This number will serve as a gauge of your progress toward this goal.

Using the example in Chapter 4, let's say you add up all your expenses, and you find that you spend \$2,500 a month. Then you add up all of your liquid assets and subtract any money you owe, and you find that your liquid net worth is \$156,500. Your withdrawal rate would be:

\begin{center}
Withdrawal rate = \$2,500/month $\times$ 12 months $\div$ \$156,500 = 19.16\%
\end{center}

Your goal for financial independence is to decrease this withdrawal rate over time. You do this by saving more money (income, Chapter 1, and investing, Chapter 3) and spending less (frugality, Chapter 2). For example, suppose you cut your costs by \$100/month, and you've stashed away \$2,000 in your investments:

\begin{center}
Withdrawal rate = \$2,400/month $\times$ 12 months $\div$ \$158,500 = 18.17\%
\end{center}

You will have achieved financial independence when you've dropped your withdrawal rate down to or below your target withdrawal rate. Your target withdrawal rate depends on several factors, so you will need to calculate it yourself. I've included in the Resources section the best online withdrawal rate calculator I've found.

No one has just one target withdrawal rate, but a range of safety. What a withdrawal rate calculator does is back-tests a certain investment strategy to give you some range of withdrawal rates and how likely they have been in the past to ensure that your money survives. In other words, in some past stock and bond market scenarios, at any given withdrawal rate, you'd have spent too much money, and ran out before your life expectancy. Running out of money is the death blow for financial independence, especially if you've been out of the workforce for a long time. The calculators can't tell you what the future holds, but it can give you some range of probabilities, based on past behavior, what you can reasonably expect your odds to be. You have to decide for yourself what kind of odds you're comfortable with. The higher the odds of success you choose, the more money you'll have to save, so the longer it will take. It's up to you.

A very common target withdrawal rate, which leans toward the overly conservative side, is 4\%. This is the target withdrawal rate that I personally use, and will use here. 4\% is a very safe withdrawal rate. I usually keep my withdrawal rate even lower than that. My reason for this is the idealism I mentioned: I refuse to go back to work. Ever. I preferred working a little longer so I could guarantee that when I'm done, I'm really done. 4\% is also very convenient because it's a nice round number, so it's very common. Please substitute your own target withdrawal rate when you see me use 4\%. That includes the 25 rule and the 300 rule. If your target withdrawal rate is 4.5\%, for example, then you'd use a 22 rule ($1 \div 0.045$) and a 267 rule ($12 \div 0.045$).

I think it's useful to create a spreadsheet for this in your balance sheet file. Create a row for your liquid net worth, and a row for your monthly expenses. Then make a row for your current withdrawal rate, which is 12 times your monthly expenses divided by your liquid net worth. Then make a row for your target withdrawal rate. Once your withdrawal rate drops below your target, then you have achieved financial independence! Meanwhile, this number will probably be very large, maybe even in the hundreds or thousands. If you're in debt or have no savings whatsoever, this number is essentially infinite. That's fine! As you make progress, this number will drop.

You can watch the trends of your dropping withdrawal rate, and maybe even construct a graph for this. I must warn you, however, that these trends are not linear, due to how ratios work (the math term for this is \emph{asymptotic}). The progress you make early on will have much greater impacts on your withdrawal rate than the same amount of progress will have as you start approaching your target withdrawal rate. This can be exciting at first, as you watch your huge leaps of progress, but as you get close to your goal, it can be frustrating, as you watch your withdrawal rate crawling closer to your goal, somewhere below 10\%. Those last few percentage points are the hardest. Just keep this in mind and know that, as long as this number is dropping, you are making progress.

Your target withdrawal rate is only a guess about what the future holds. I probably don't need to tell you that the future usually turns out a lot different than we expect. We can plan the best we can, and use reasonable guesses to facilitate that, but you should remember that you're only guessing. You have to be ready to adjust your plans as things change.

The withdrawal rate probabilities are most heavily affected by what happens in the first two years of withdrawing money. These are big growth years that set the stage for the rest of your lifetime of withdrawals. If you get a lot of growth in these first years, then you'll end up with more money than the withdrawal rate estimated. If you lose money in these first years, you'll be more likely to end up with just enough or not enough money than the withdrawal rate predicted.

Plan the best you can. Pad your expenses. Keep a low withdrawal rate. But don't be paranoid. At some point, you need to trust that you've planned the best you could. If you don't, you risk becoming a miser. Remember, the point of this is to have a life of freedom and joy, not paranoid slavery to your own fears and insecurities.

\section{Your new paycheck}
One common feature of jobs that people think goes away with retirement is the almighty paycheck. Not so. Retired people get paychecks just like everyone else, although they may look a little different. In retirement, ``paychecks'' are called ``withdrawals.''

Once you start taking withdrawals, your ``income'' is simply your target withdrawal rate multiplied by your liquid net worth. For example, if your withdrawal rate is 4\%, and your liquid net worth is \$750,000, then your annual income is \$30,000/year ($4\% \times \$750,000$).

You should have a cash account of some kind to take withdrawals from throughout the year. This could be your checking account, savings account, money market fund, or laddered CDs. Laddering CDs means that you have one Certificate of Deposit for each month. One CD would mature each month, which gets deposited into your checking account. So, you'd have a 1-month CD, a 2-month CD, a 3-month CD, a 4-month CD, etc., up to a 12-month CD. Each year, you'd open up 12 more, or you could just open a new 12-month CD every month. The idea is that since you know you won't need that money until the maturity date, you can take advantage of the higher rates CDs theoretically offer.

However, I've found that CDs don't offer very high rates. I've always been able to find a savings account that offers a higher rate than even a 1 year CD. So, why bother with all the complexity of laddered CDs when I can get a higher rate in a simple savings account? But do your own research and do what makes sense for you.

If you choose to use a savings account, what you could do is put 1 year of living expenses in a savings account, or maybe a little less, since you'll be earning interest on it throughout the year. Each month, you could make deposits into your checking account from your savings account, and that would be your paycheck. Each year or so, you should re-balance your portfolio. You will be selling some of your assets and buying others. That's the time to replenish your savings account or CDs.

Don't worry if you run out of cash before the year is up. It's fine to sell assets that have appreciated higher than your target allocation. You should also have a good amount of short-term, investment-grade bonds in your portfolio, which are much less volatile than other investments. They aren't quite as safe and stable as a savings account, but it doesn't hurt to pull from it occasionally when you need the cash.

\section{Getting raises}
It's impossible to know ahead of time exactly what your withdrawal rate should have been. The best you can do is look at past trends and estimate a target that would have kept you relatively safe in those scenarios. But the future \emph{will} be different than those scenarios, which means that your withdrawal rate \emph{will be wrong.} The idea is to choose a conservative enough withdrawal rate that the future will be even brighter than the scenarios you've planned for, and you will end up with even more money than you expected. In other words, your boss gives you a raise!

The traditional withdrawal method is to simply take your liquid net worth on the day you start withdrawing from it, and withdraw only 4\% of \emph{that} each year. No matter what your investments do or what your withdrawal rate becomes, you keep your spending even. This works fine except for periods of high volatility and high inflation.

After you've been taking withdrawals, you may find that your liquid net worth grows over time, and your withdrawal rate is dropping. Some of this is to be expected, since the stock market can be very volatile, so your money may shrink again, causing a spike in your withdrawal rate that restores it to its target, or goes over the target for a while. But if your liquid net worth continues to grow and your withdrawal rate keeps dropping, at some point it's good to assume that your target withdrawal rate was too low, and you're free to spend some more. In other words, you get a big raise after many years of conservative spending. I don't use this method for that reason. I'd rather get lots of smaller raises along the way rather all at once down the road.

On the other hand, if you notice your liquid net worth shrinking, and your withdrawal rate climbing, after a while that could be a sign that the initial withdrawal rate you chose was too high. This is dangerous, because at some point, you'll run out of money. There are only two ways to fix this: spend less or earn more. It will be much harder after several years of over-spending. That's another reason I don't use this method. I'd rather make smaller adjustments as they prove necessary, using a buffer to protect against volatility putting me in a constant state of feast or famine.

The other methods are more dynamic, and involve tracking your expenses to your withdrawal rate. This is difficult because, between surprise expenses and market volatility, your withdrawal rates will jump around like a pogo stick. How can you know what is safe to withdrawal if your withdrawal rate won't stay still? There are a few approaches.

One is to have a buffer. Whenever your withdrawal rate drops below 4\%, set aside a portion of your ``raise.'' Whenever the withdrawal rate jumps over 4\%, use up the money you set aside in previous months to bring it back to 4\%.

Another method, suggested in the book \emph{Work Less, Live More,} is called the ``95\% rule.'' Whenever your withdrawal rate goes over 4\%, drop your expenses as much as 95\% of what you spent last year. The author found that it's fairly safe to spend that much, even during bear markets.

The method I use is to pick a range of acceptable withdrawal rates, and simply ignore the volatility within that range. I usually keep withdrawal rate below 4\%, so I don't worry about it if my withdrawal rate occasionally gets as high as 4.5\%. 5\% is my panic button. That's when it's time to make significant changes. This is pretty conservative, as even 5\% can be a somewhat safe withdrawal rate for diversified investments.

One monkey wrench in your planning is inflation. I've talked about inflation risks in Chapter 3, but I didn't explain how it will effect your planning. \emph{Your Money or Your Life} suggests you simply ignore inflation risk entirely, which would be convenient since it would simplify your planning. However, in this book, I don't suggest you ignore this risk, which means you need to plan for it.

Inflation is the drop in value of your money over time. In five years, you will be able to buy less with the same money that you have now. If you didn't have investments that outpace inflation, then it would be to your advantage to spend that money as quickly as possible, rather than investing it for the future. So, if you're trying to figure out how much money you'll have and how much you'll need at some point in the future, which is necessary for financial independence planning, you'll need to account for inflation.

The problem is, inflation is hard to predict, and therefore hard to plan for. Some Chicken Little economists warn of stagflation or hyperinflation since our economy no longer depends on a gold standard. Others say the dangers of inflation is exaggerated. It's up to you whose opinion you trust the most. All I recommend is that you not ignore inflation risk entirely, and that you try to plan for it the best you can.

All withdrawal rate calculators take inflation into account, and I believe most of them use the \emph{Consumer Price Index,} or \emph{CPI,} to measure it. That's why the withdrawal rates they give tend to be low versus the growth you can expect from your investments. If you use the traditional method, then once you start taking withdrawals, you need to increase the amount you are allowed to take each year to account for inflation. So, this is another way your ``boss'' gives you a raise. You can assume this to be 3\% per year, which is a conservative estimate of inflation. In the example I gave earlier of a liquid net worth of \$750,000 and an annual income of \$30,000, your raise in the first year would be \$900, bringing your income up to \$30,900. The second year, it will be \$927, bringing your income up to \$31,827. And so on.

There is another form of inflation that is often ignored, which I call \emph{age inflation.} As we age, in addition to prices, our own expenses also change. In most cases, they drop. We've already accumulated a lot of the things we want, we've finished paying off our mortgages, the kids have moved out and gone through college, and we can move to a smaller home in a less expensive area. You may want to plan for these changes. But there is one way that we can expect our expenses to go up as we age, and that is in our health care.

\section{Health insurance}
\emph{Health care} is deceptively named. It's actually illness care. Aside from annual exams, most people don't actually use it unless they get sick. Health care has more to do with your lifestyle than with your doctor. Caring for your health is mostly up to you, by how you eat, sleep, work, and how you spend your time. The best way to save on illness care costs is to decrease your need for it, which mostly has to do with how you live, although genetics and environment are also factors.

\emph{Health insurance} is also deceptively named. Most health insurance is actually health maintenance contracts. Insurance covers you in rare cases of financial hardship. So, true health insurance wouldn't actually come into play unless you become seriously ill or injured. Such plans do exist, and they have extremely low premiums. Look for health insurance with high deductibles and low premiums, and then do what you can to stay healthy so you won't need it. HSAs, which I mentioned briefly in Chapter 3, come with such insurance.

It's difficult to predict what will happen with health care in the future. Health care costs are out of control. Liberals argue for government-sponsored health care for all, while conservatives think the government has its hands in too much of the economy as it is. All that is clear is that health care costs are ridiculous, the rate at which it's going up is not sustainable, and at the time of this writing, there is still no complete solution.

If you are getting health insurance from your employer, then that is a hidden part of your total employment compensation package. This is income that will go away when you leave your job, so it's income that you will need to make sure you provide for yourself. I recommend that you talk to an insurance broker and estimate how much you'll need to pay for health care once you're financially independent, and include that cost in your expenses for your withdrawal rate calculation.

There is a legally mandated health insurance called \emph{COBRA,} which is really just your former employers' health insurance that you pay for instead of your former employer. COBRA is extremely expensive. Most individual health insurance is much cheaper. However, there is a law called \emph{HIPAA,} which says that insurance carriers can't hike up their rates on you as long as you meet certain conditions, and one of those conditions is that you have continuous coverage. Another condition is that you must use COBRA until it expires, so that would be a reason to use COBRA.

In addition to planning for health insurance in your projected expenses for the purpose of your withdrawal rate, I also recommend that you set aside a certain portion of your net worth for a rise in health care costs and emergencies over your lifetime. I call this \emph{health insurance insurance.} Don't include this portion in the net worth for the calculation of your withdrawal rate. Treat it as money you can't use for funding your financial independence. How much you set aside is up to you. You could guess, or you could talk with an insurance broker and make some projections. When I did this, I decided to plan for my premiums to go up 35\% every five years, in addition to the standard inflation rate. Just make sure that when your health care costs go up, or emergencies come up, you deduct the principal specified by your withdrawal rate for these costs from the portion you set aside for this purpose.

Since health insurance is so completely broken and even corrupt in some ways, you face the risk that your health insurance provider will refuse to pay your claims for bizarre reasons, and their well-paid lawyers will make sure they can get away with it. Additionally, if you do get seriously ill, you will suddenly start paying the full deductible each month, which will cause your expenses to go up. This isn't just a problem with health care, but any catastrophe that comes up that insurance doesn't cover. And it isn't just a problem for people who aren't working. Having a paycheck each month doesn't mean that that paycheck will be enough to pay for any catastrophes that come up. You must set aside money for that.

Insurance doesn't cover you for every possible risk. You should have an extra cushion in your savings that you set aside for emergencies. Either don't include this in your withdrawal rate calculations, or change your target withdrawal rate to account for it. Some people may think it's better to just get lower deductible insurance, but if you think about it, that's actually the same thing except that you're giving that extra money to an insurance company for one specific purpose, rather than keeping it for yourself, to use for any purpose. Consider how much higher your monthly premium would be if you got insurance with a lower deductible. Multiply that by 300, and that is the extra cushion you need to set aside. You could either dedicate that to paying your insurance company, or you could keep it for yourself.

It's impossible to predict and plan for every contingency, but health care is one of the likeliest culprits of raining on your financial independence parade, if you don't plan for it well enough. Once you're old enough to qualify for Medicare, your health insurance costs will drop again. However, Medicare is limited, so you should still plan to have health insurance, which is called Medi-gap insurance.

\newpage
\section{Resources}
\begin{itemize}
\item \textbf{\url{http://www.passionsaving.com/retirement-calculator.html}} -- This is the best withdrawal rate calculator I've found.

\item \textbf{\url{http://www.ehealthinsurance.com/}} -- This is an online health insurance broker that is very easy to use and compare providers. 
\end{itemize}
