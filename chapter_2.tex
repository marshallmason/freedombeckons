\chapter{Frugal Does Not Mean Stingy}
\fancyhead[CO]{\emph{Frugal Does Not Mean Stingy}}
\fancyhead[CE]{Freedom Beckons}

Money is a touchy subject. It's taboo to talk about how much money we earn, but it's fine to talk about how much we spend. For example, it's easy to ask how much someone paid for their new car, but it's not so easy to ask about their annual salary. In other ways, we're more sensitive about spending than earning---it's fine to talk about what we do for a living, but our attitudes toward spending are often labeled with judgmental words like \emph{miserly, selfish, stingy,} and \emph{greedy.} Even positive words like \emph{generous} are still judgments of someone's character.

\section{Generosity}
Frugality doesn't necessarily mean giving less to others. It often means giving more. Eq\-uating frugality with being less generous is a common misnomer.

One of the best known portrayals of stinginess in popular culture is the story, \emph{A Christmas Carol} by Charles Dickens, and its many movie renditions. It exaggerates all the worst aspects of stinginess---selfishness, greed, and a blatant disregard for the suffering of others. The main character, Ebenezer Scrooge, has an epiphany one night in which he discovers the true costs of his stinginess. This leads him to become wasteful, spending lavishly on others with no regard to cost. This kind of portrayal confuses people about the true meaning of frugality. It presents wasteful spending as the antithesis to and the remedy for stinginess and frugality, which are viewed as synonomous.

I believe a better example of frugality is Warren Buffett, a famous investor and business owner, and one of the richest people on the planet. He lives in a small house, drives a modest car, and shops at thrift stores. His investing philosophy is also based on frugality. He's an advocate for low-cost investing. He made his billions by buying and holding well-managed but underpriced companies. Then, after this helped him become one of the richest people in the world, he donated the majority of his wealth to a foundation that works to reduce global poverty and improve health care in developing countries.

This is one way frugal people can be more generous---by how they allocate their spending. They spend less money on frivolous personal items that don't bring much satisfaction so they may have more to donate to causes they believe in. Frugality can be motivated by a stark awareness of the suffering of others.

Giving money can be a powerful form of generosity, but there are other forms of generosity as well. You can also give your time. Giving time is generous if it's quality time that helps others in meaningful ways. It can be more gratifying to donate your time than money. It's more personal, and often more necessary. Many well-funded organizations are strapped for manpower. 

Freedom itself can be a form a generosity. Since some needs are more urgent than others, I've found it invaluable to have the freedom to drop everything I'm doing and offer myself to someone for as long as they need me. This is the most rewarding form of generosity for me.

Gift-giving is another form of generosity that can use some re-thinking. I only give a gift if I can do so completely from the heart, I know the person will benefit from it, and it fits my budget. When those conditions are met, it seems silly to wait for a special holiday. Why not give the gift when I feel inspired to, rather than wait for society to sanction it?

\section{Frugality}
Stinginess means detesting the spending of money. Misers save money for the sake of saving money, hoarding it for themselves. This often stems from fear, insecurity, or greed. Spending or saving money always entails values and trade-offs, but for misers, these important aspects of money are often obscured by their emotions.

Everything we buy is a trade-off we've chosen to make. We may choose to spend more for something of higher quality or a recognizable brand name, for example. We may choose to shop down the street or we may choose to shop in the next town. For every purchase we choose to make, we have implicitly chosen to not make countless other purchases with that money. Whenever we spend money, we always give up other things in order to get something else that we value more, even if we're not aware of it.

The choice to buy or not to buy is also a trade-off---the choice to work and spend rather than not work and not spend. The money you spend has to come from somewhere. The more you spend, the more you have to work, so the less you spend, the less you ultimately have to work. The goal of financial independence is to gain the freedom to work less, or for less money, which means spending less.

Consider how much companies spend on advertising each year, and consider what form most of that marketing takes. Ads often target our psychological weaknesses, particularly our emotional reasoning. These ads work so well because people are unaware of their motivations for buying or the trade-offs they're making. Most marketing thrives on ignorance.

The goal of frugality is to combat this ignorance, so that all of your spending decisions are the right ones for you. The point is to bring your spending in alignment with your highest values, not necessarily to decrease the amount. It usually results in spending less money, but it may also mean spending \emph{more} money in certain areas.

Frugality is about awareness. It doesn't mean worrying about every penny you spend. Frugality is about knowing exactly what you're buying, why you're buying it, and what it costs you in real life terms. It's not an act of minimizing, but of \emph{optimizing.}

If you become aware of what you really need, you can seek ways to address those needs more directly, rather than indirectly through buying something. Buying things rarely gives us what the ads imply, like status, approval, lovers, and happiness, so we end up buying more and more in our on-going pursuit of happiness. Awareness can break that cycle. Frugality isn't about sacrificing happiness for money. \emph{It's about maximizing both.}

In this chapter, there will be a lot of math. People either love math or they hate it. Most of them hate it. If you're a math geek, you'll love this, so enjoy. If you don't like math, please don't let these equations and calculations intimidate you. I'm just trying to make a point, so please just focus on that point.

\section{Life exchange rate}
Time is all any of us have. It is a finite resource that we may spend in any way we choose. Your life consists of the precious time you have on this planet. Mathematically speaking, life = time.

A job is a contract to exchange your time for money. The more time you spend working, the more money you can earn. This creates an equivalence between time and money: time = money, hence the phrase, ``time is money'' among business people.

Therefore, there is a one-to-one connection between your life---the precious time you have on this planet---and the money you spend. Life = time and time = money, so life = money. This may sound a little strange, as if we somehow live for money. During the hours you spend at work, this is literally true. That portion of your life is devoted to money. That doesn't mean you can't enjoy your job, just that its main purpose is to earn money.

When you contract to sell something, that which you've sold no longer belongs to you. You may have some choices of jobs, or how you do those jobs, but once you've agreed to make this exchange, the choice about whether to spend your time doing that job or doing something else is one you no longer have. You could walk out, but the consequences may be severe. The whole point of taking the job was to pay for the things you need in life. Without money, you starve. In the best case, you will just enter into another contract with someone else. Ultimately, that piece of your life now belongs to someone else.

Money is abstract and has many emotional connotations, which makes it hard to identify with. Since life = money, we can treat money and life as interchangeable. Just like money, you spend life, wisely or unwisely. It's something you probably cherish and want to maximize. You trade your life for money, and like any conversion between interchangeable currencies, there is an exchange rate. You can quantify exactly how much of your life you're spending whenever you spend money: your \emph{life exchange rate.}

Start with your net annual wage. Deduct from this all work-related costs: work clothes, commuting costs, education, lunches, office supplies, day care costs, union dues, etc. Then add in any unpaid benefits you receive, such as health insurance or free gym membership. Make sure these are all in annual terms.

Now estimate how much time you spend on your job each year, working, commuting, preparing for work, unwinding from work, keeping up with your field, education, sleepless nights, time with doctors for physical or mental work-related illnesses, networking, etc. Divide your net income above by the total hours per year you spend for your job. The result is your life exchange rate.

For example, let's suppose you earn \$50,000 per year. This is called your \emph{gross annual wage.} To get your \emph{net annual wage,} you must subtract taxes, FICA, etc. These are usually deducted from your paycheck before you even see it, so an easy way to do this is to just add up all your paychecks for a year. For this example, let's suppose your net annual wage is \$42,000 per year.

\begin{table}[ht]
\caption{Work-related expenses}
\label{table:expenses}
\centering
\begin{tabular}{l l}
\hline\hline
Clothing & \$600 \\
Commuting & \$4,000 \\
Education & \$400 \\
Lunches & \$1,200 \\
Day care & \$12,000 \\
\hline
Total & \$18,200 \\
\end{tabular}
\end{table}

Next, add up all of your work-related costs. Estimate these, if necessary. Use annual values for everything. So, if you go shopping for clothes every six months, you would need to double that to determine how much you spend annually.

For some values, you will need to separate personal use and work use. For example, if you use your car to commute and for pleasure, you'll have to figure out which portion is for work and only count that. Some values are only a premium of what you'd have to pay if you weren't working, and that premium is all you should count here. So, if you get lunch at a restaurant, you only need to count the difference in cost between the restaurant and cooking at home.

Let's suppose your work-related expenses are as in Table \ref{table:expenses}.

Next, add up all of the benefits you receive from your job that aren't included in your paycheck. Let's suppose your work-related benefits are as in Table \ref{table:benefits}.

\begin{table}[ht]
\caption{Work-related benefits}
\label{table:benefits}
\centering
\begin{tabular}{l l}
\hline\hline
Health care & \$600 \\
Gym membership & \$300 \\
\hline
Total & \$900 \\
\end{tabular}
\end{table}

Next, determine the total time you spend on your job per year, not just hours worked. Let's suppose the annual hours you spend on your job are as in Table \ref{table:time}.

\begin{table}[ht]
\caption{Work-related time}
\label{table:time}
\centering
\begin{tabular}{l l}
\hline\hline
Working & 2,000 hours \\
Commuting & 188 hours \\
Preparation & 125 hours \\
Unwinding & 125 hours \\
Education & 30 hours \\
\hline
Total & 2,468 hours \\
\end{tabular}
\end{table}

Finally, we can use these three totals to calculate your life exchange rate:

\begin{equation}
	\frac{\$42,000 - \$18,200 + \$900}{2,468} = \$10/hour
\end{equation}

Translating the money you spend into life can give some perspective on what trade-offs you're really making. Whenever you're considering spending money, try dividing its cost by this number. For example, suppose you're considering going out to eat rather than cooking for yourself. The restaurant you have in mind would cost \$15. You figure you can cook for yourself for about \$5. So the premium you pay by making this choice is \$10. In our example, we've calculated that your life exchange rate is \$10/hour, so eating at this restaurant costs one hour of your life.

Don't think of that purchase as \$10. That's too abstract. It obscures its true cost. Instead, as you sit down to eat for the next hour, imagine that you had to spend an extra hour working for this privilege. Is it worth it? Would you rather work for an hour and eat at the restaurant, or have the option to not work for an hour of your life? That's the trade-off you have to consider.

\section{Opportunity cost}
The life exchange rate only accounts for the trade-off between time and money. It doesn't account for the trade-off between spending and investing. In investing parlance, this is called \emph{opportunity cost.} If you invest money rather than spend it, then each year it earns more money that you would otherwise have had to earn from your job.

To accurately calculate your opportunity cost, you need to use the compound interest formula, which is defined as follows:

\begin{equation}
	A = P(1 + \frac{r}{n})^nt
\end{equation}

where A is the amount accumulated, P is the amount invested, r is the interest rate, n is the compound frequency, and t is number of periods.

There are also plenty of compound interest calculators online, one of which is included in the Resources at the end of this chapter.

An easy way to approximate the compound interest formula is the \emph{Rule of 72.} If you divide 72 by an annual return rate times 100, you get the approximate number of years it would take for your initial capital to double. So, if you expect to earn 8\% from your investments, then it would take 72 $\div$ (0.08 $\times$ 100) = 9 years for it to double. Knowing your money doubles every 9 years, it's pretty easy to get a sense for your opportunity cost.

You also need to have some sense for how long you plan you invest. In other words, how many years before you expect to achieve financial independence? After reading Chapter 1, you should have some vague sense of this. For this example, let's suppose you expect to invest for 10 years.

According to the Rule of 72, money invested at the beginning would be worth slightly more than double by the time you reach your goal. That means all the money you spend at the beginning costs twice as many hours of your life than is suggested by the life exchange rate alone. In order to factor in the opportunity cost of spending money rather than investing it, everything costs twice as much. Now, that hour at the restaurant doesn't just require one hour of working, but two hours.

\section{Spending habits}
Spending choices don't happen in isolation. They are usually part of an overall \emph{spending habit,} in which the same spending choice is made repeatedly over time. So far, we have only considered individual purchases. Next we need to incorporate the overall spending patterns that individual purchases tend to be a part of. Small changes to these spending habits can reap huge rewards over time, so it's important to gain an awareness of them as well.

Since your goal is financial independence, what you're trying to do is invest a certain amount which will produce money at regular intervals to meet all of your spending habits. For each spending habit, you must invest a certain amount which will accomodate that habit.

In Chapter 5, I will explain how to calculate this yourself, but for now, a reasonable assumption is \$25 per year. In other words, for every \$1 you spend per year, you must have \$25 invested to accomodate that habit indefinitely. For monthly expenses, simply multiply by 12 months, which is \$300. For weekly expenses, multiply by 52 weeks, which is \$1,300.

Whenever you want to spend money, ask yourself how often you intend to spend it. Using the same example, let's say you'd like to eat at a restaurant once a week. $\$10 \times \$1,300 = \$13,000$. You would need to earn and save \$13,000 if you would like to eat at a restaurant every week. If your life exchange rate is \$10, as in the previous example, then this restaurant habit costs you 1,300 hours of your life. If work consumes 50 hours per week\footnote{Remember, this isn't just time spent working, but all of the time you devote to your job, including commuting and preparing for work.}, it would take you about $6\frac{1}{2}$ months to earn this.

We can also calculate the opportunity cost of spending habits. Since we're dealing with an on-going cost rather than a one-time cost, we can't use the Rule of 72. We need the compound interest formula or compound interest calculator described in the previous section. Let's say you want to achieve financial independence in 10 years, as in the previous example, and you expect your investments to earn 8\%, after inflation. Your initial amount is \$0 and your annual investment is \$520, the annual cost of your restaurant habit.

The calculator says you'd have about \$8,136 if you'd have invested this money rather than spent it for 10 years. This includes both the amount you invested and the interest it earned. The opportunity cost, the amount that you had to give up in order to eat at this restaurant every week, is only the interest portion. So, we need to subtract out the amount we invested: $\$8,136 - \$520/year \times 10 years = \$2,936$. This is the cost of spending the money rather than investing it. If you had invested it rather than spending it, you wouldn't need to work in order to accumulate this three grand. It would have literally earned itself.

Using your life exchange rate---let's use \$10 per hour as the example---it would take you about 294 hours to earn the money that could have been earned for you, or almost six weeks at work. Add this to the 1,300 hours already calculated for the habit itself. In total, the true cost of a weekly habit of eating at a restaurant is 1,594 hours of your life, nearly eight months of working full time. Did you have any idea eating at a restaurant was so expensive? And that's only one relatively small spending habit. Imagine how everything adds up when you do this for all of your spending habits.

\section{Putting it all together}
I've made several assumptions here. I chose 10 years for the time horizon; I used \$10 per hour as the life exchange rate; 8\% was the assumed the return rate; 50 hours per week was used to determine how many months you'd have to spend at work. I also only gave one example, the weekly restaurant habit.

These numbers will be different for each person. The life exchange rate and time spent at work depends on what you do for work. The 8\% figure is based on average returns, after inflation, taxes, and expenses.\footnote{This is only a guess, useful for our purposes here. Predicting this number exactly is impossible, and guessing is frought with many assumptions. For our purposes here, it's fine to use 8\%.}

The numbers will also be different for each spending habit. It's possible to run through this routine for every spending habit you have, but it would be very tedious and unnecessary. It's also not very feasible to whip out your financial calculator whenever you're out shopping.

One solution is to plug them into a spreadsheet program. Enter your own values in the cells and use the built-in compound interest calculator. Any time you want to determine how much of your life it will cost you for any given spending habit, you can simply change a couple of cells in a spreadsheet. But even this isn't really necessary. The only point of this exercise is to give you a sense for the true costs of your spending habits. You won't need to calculate the cost of very many spending habits before you'll start to get a feel for the kinds of purchases that cost the most.

I rarely needed to calculate the life cost to know whether a spending habit would or wouldn't be worth it. This was especially true for those spending habits I guessed would \emph{not} be worth it---the risk of being wrong was zero. When I guessed small expenses \emph{would} be worth it, I could always double-check later with a real calculation, and then stop the habit if necessary.

For example, one day I might be in the mood for a soda. I don't think twice about tiny expenses like this. A few days later, I'd grab another soda, and then again the next day. Then I'd realize this was a spending habit. I knew just how much impact even such a small spending habit can have, so I'd check to see how much this was really costing me. At \$1 a day, that's \$30 per month. Using the hypothetical assumptions above, that would cost almost 7 months of working to accomodate such a trivial habit. I'd ask myself, would I rather have a soda everyday, or reclaim 7 months of my life? When the trade-off was made crystal clear like this, the choice was usually easy, and I made it without regrets. Of course, if I knew I \emph{really loved soda}, then I would happily choose the spending habit, again without regrets.

If you're good at simple arithmetic, you could also memorize the life cost of a few key spending habits, and then twist that number around for other habits. For example, we've already determined that spending \$10 per week costs 1,594 hours of your life, or eight months of working. If instead you're considering spending \$40 per month, that's the same thing as \$10 per week, so that too will cost you eight months. Or let's say you're thinking about spending \$5 per week---you can just cut it in half, so it would cost you two months. If you're considering spending \$50 per month, that's just a bit over \$40 per month, so you know it will cost you a bit over four months. Precision is unimportant. You're just trying to get a general sense of the trade-offs.

Here's a summary of the full process to calculate the life cost of a spending habit:

\begin{enumerate}
\item Normalize to annual costs: multiply monthly costs by 12, weekly costs by 52, and daily costs by 365.
\item Multiply by 25. This is cost of the spending habit.
\item Use a compound interest calculator to calculate how much you'd have if you'd invested this money instead of spending it.
\begin{itemize}
\item Use 0 for the initial cost.
\item Use the annual cost from step 1 for the annual investment.
\item Use the time between now and your target date for financial independence for the growth time.
\item Use 8\% for the growth rate, or whatever you think is appropriate.
\end{itemize}
\item Subtract from this the annual cost from step 1 times the number of years between now and your target date from step 4. This is your \emph{opportunity cost.}
\item Add this to the cost of the spending habit from step 2. This is the total cost.
\item Divide this by your \emph{life exchange rate} to get the life cost.
\item Optionally, divide this by the number of hours you work per month to determine how many months you would need to work to accomodate this habit.
\end{enumerate}

Here's one more example, using different numbers. Let's say you're considering subscribing to a premium web service, such as Netflix or Amazon Prime. It costs \$10 per month, a pretty typical cost of such a service at the time I wrote this. Let's use some different numbers. Let's say you want to achieve financial independence 13 years from now. You estimate the growth rate of your investments will be 7\%. Your life exchange rate is \$15 per hour, and you work 50 hours per week.

\begin{enumerate}
\item The annual cost is \$10 $\times$ 12 months = \$120.
\item The spending habit costs $\$120 \times 25 = \$3,000$.
\item The money you'd have if you'd invested this instead, from the compound interest calculator, is \$2,586.
\item The opportunity cost is $\$2,586 - \$120 \times 13$ years = \$1,026.
\item The total cost is $\$3,000 + \$1,026 = \$4,026$.
\item The life cost is \$4,026 $\div$ \$15 per hour = 268 hours.
\item It would take you 268 $\div$ 50 hours per week = 5.36 weeks to pay for this web service.
\end{enumerate}

\section{A new perspective on spending}
Every time I pulled out my wallet, I asked myself how much of my life I'd be spending. Seemingly small costs, made frequently, added up in ways that often astonished me. It gave me a whole new perspective on my spending. Over time, it became second nature for me to think about spending in this way. People would be surprised that I would easily drop \$50 on a DVD player, but refuse to spend a buck for a soda. \$50 every 10 or 20 years is much cheaper than a daily soda habit.

Once you know the life costs of your spending, compare that to how much fulfillment, satisfaction, and value you're getting, and how much it's in line with your values and life purpose. If the life cost is low, the fulfillment and satisfaction are high, and it's very much in line with your values and life purpose, then spend it! If not, then celebrate your choice for something greater rather than something smaller. This is the kind of thought that should go into every purchase you make.

I was grateful for every dollar I didn't spend. Sometimes I would need to do some calculations. Especially at first, I was shocked by how much things were really costing me. Choosing not to spend that money felt so rewarding because I knew I was getting something so much better instead. When I did choose to spend the money, it also felt very rewarding for the same reason. I made the calculation and there was no question that it was worth every hour of life it cost me. I knew I was getting something so much better in return for the life I spent to buy it.

My guitar is a great example. After playing cheap guitars all my life, I finally bought an expensive, high-quality guitar. I didn't wring my hands over it. I had no reason to worry about my choice, and I had no regrets because I'd quantified exactly how much of my life I was giving up for it, and I knew without a doubt that this instrument that I would own for decades, would enjoy playing every day, and was in line with values, was well worth the life cost.

The trade-offs you make are your choice. I only urge that you be conscious of this choice. Calculate how much life you're trading for what you spend. If you find that it's not worth it, then don't spend it. But if you find that it is worth it, then you should spend it, or spend \emph{even more} if that makes sense.

\section{Budgets}
Budgets are like the financial equivalent of diets. Instead of counting calories, you count dollars. Budgeters decide each month how much they're willing to spend in various categories, and then try not to exceed that amount each month. I recommend that you not budget in this way, unless you know it works for you. The reason is that it comes from a mindset of sacrifice. This is not a very motivating concept, and may turn purchases into ``forbidden fruits,'' which only makes you want them more. But the worst part is how discouraging it is to live with an attitude of sacrifice. I find \emph{abundance} to be a far more motivating concept.

If you start thinking in terms of life instead of dollars, you can become more conscious of what choices you're actually making. You're more likely to make the right choices for yourself, based on your own values. So there's really no reason to budget, or to sacrifice. Every choice you make, to spend or not to spend, or how you spend, is a choice of abundance---you're \emph{always} choosing something greater over something smaller. It's a matter of shifting your perspective and attitude. No more budgets and no more sacrifice! Isn't that a relief?

That doesn't mean you shouldn't keep records or make plans. There's no better way to be fully conscious of your trade-offs than to have thorough records of the choices you're making, which I'll discuss in Chapter 4. I'm only suggesting that you base your spending choices, not on preconceived limits, but on a moment-to-moment awareness of the abundance of life and your desire to get the maximum value from that immense, but nonetheless finite, abundance.

\section{Easy ways to cut costs}
There's a concept that seems to be true for a lot of human activity, called the \emph{Pareto principle,} or \emph{80/20 rule}: 80\% of the effects come from 20\% of the causes. This ratio has been observed in many spheres, especially economics and business. For example, 80\% of sales tends to come from 20\% of the clients, and 80\% of productivity tends to come from 20\% of the employees. In our case, the Pareto principle implies that 80\% of the money you spend is for 20\% of your expenses. You'll see the most improvement by focusing your efforts on optimizing that 20\%.

You might be amazed to find that, just by poking around a little, you can cut costs dramatically, with little or no impact on your quality of life whatsoever. Sometimes you may find cheaper ways to get the same exact things. Other times you may discover that the trade-offs you're making aren't appropriate, so you decide to scale back on an expense.

Here is a list of examples from my own experiences with this. Inevitably, the spending choices you make will be different from mine or anyone else's. I cannot tell you which expenses you should re-evaluate. All I can do is provide some general tips I've picked up over the years. Some or even all of these may not be relevant for you, but it's a useful exercise nonetheless because it serves as a good demonstration of the process of frugality. In that spirit, here are some ways I've found to significantly cut costs:

\textbf{Subscriptions} --- I'm extremely cautious about subscription services. These are used to lock consumers into an effortless, long-term consumption habit, with a built-in loyalty to the service provider. This can be deceptive about the long-term costs of that habit. By focusing only on the monthly cost, rather than the total cost, things may appear cheaper than they actually are. By deducting the cost automatically from my credit card, the conscious process that I engage in when I spend money is circumvented. Subscriptions are common because of the ways they benefit the businesses selling them. Some services are no longer available on anything but a subscription basis, so you may be stuck with them. Just be cautious and seek alternatives when possible. When you stop deriving value from them, cancel them immediately.

\textbf{Cell phones} --- People think I'm a luddite, but I abhor smartphones, holding out to buy one until only recently. The iPhone in particular seems like an over-priced fashion statement. However, the cell phone industry is brimming with competition, and therefore ways to cut costs. Some cell phones cost less than land lines, while others cost hundreds of dollars a month. It's also possible nowadays to do without phones altogether, and instead use web services like Google Voice or Skype.

\textbf{Internet banks} --- I use an internet bank for my checking and savings accounts. These banks don't have the high costs of branch offices or the large marketing budgets of the big banks. They offer free checking accounts and pay a small amount of interest. Some even provide postage-paid deposit envelopes and reimburse ATM fees.

\textbf{Online shopping} --- Another great way to use the internet to save money is to shop online. It takes a lot of time and gas to shop around in brick-and-morter shops, but it's quick and easy to do it online. I often find that I can save up to 50\% on some very expensive items, even after paying shipping. If I can't find a good price online, I haven't wasted much time, and I can go ahead and buy it at a local shop.

Even if you're familiar with online shopping, there still may be certain things you just never think to check online for. I used to pay my vet clinic \$100 for a 7-month supply of flea protection for my cats. That's \$14.29 a month. I'd done this for years before it occurred to me to check online. Within just a few minutes, I found a 12-month supply for \$81. Shipping costs were \$15, but if I bought at least \$100 worth, shipping was free. So I bought two years worth. That's only \$6.75 a month, a savings of \$7.54 per month for the same exact product.

There are some downsides to online shopping. The biggest downside is shipping costs, which can be prohibitive. In some cases, the shipping can double the price, and in other cases, the shipping is free, so you really have to check this. Another downside is safety. When you buy something at the store down the street, you can easily return it by taking it back to the store. When you shop online, you have to ship it back to return it. It's also possible to get ripped off when you shop online, but this is unlikely if you shop at reputable websites.

I tend to shop online for small, expensive items. The shipping costs for small items are usually insignificant compared to the cost savings.

Another way the internet has made it easy to save money is with coupons. Traditional coupons require so much time and effort that the trade-off is hardly worth it in most cases. The internet takes away all of this effort. It's a simple matter of typing what you're looking for into a search engine and adding the word ``coupon.'' If there are results, you can quickly evaluate how much you can save. If there are no results, then you haven't lost much time or effort.

\textbf{Insurance} --- Insurance is expensive and can vary widely. Insurance policies are rated by an agency called A.M. Best. A high rating, such as B+, B++, A-, A, A+, and A++, means the insurance company can be trusted to pay claims. It accounts for factors such as their balance sheet and operating performance, relative to their peers.

As long as they have a strong A.M. Best rating, all insurance companies basically offer the same thing. I found that shopping around for insurance could save me over a hundred dollars a year. Companies that don't have a big brand name tend to be cheaper even though they have the same or higher A.M. Best rating. Their customers benefit from their smaller marketing budgets.

It also helps to understand the exact details of insurance policies, which can vary depending on the type of policy. For health insurance, I learned about copayments, lifetime maximum, deductibles, premiums, and coinsurance, for both doctor visits and prescriptions. For auto insurance, I had to learn what liability, collision, comprehensive, uninsured motorist, medical, and personal injury coverage means. Until I understood how the kind of insurance policy I was buying worked, I often paid a lot of money for coverage I didn't need.

For example, I realized that I already had health insurance, so I didn't need medical coverage in my auto insurance. As my car aged and I had enough money in the bank to pay for it myself, I no longer needed collision insurance. I've found liability insurance to be the most important part of auto insurance. If I was the only person involved in an accident, my costs are limited to the value of my car and medical bills. But most accidents involve many people, and this can multiply the costs of the accident significantly.

I'm always willing to pay for large costs out-of-pocket. This allows me to have high deductibles and low premiums.

The insurance industry operates on paying less in claims than they make in premiums. Insurance companies hire experts who specialize in making such predictions with a high degree of accuracy. I don't ``bet against the house'' when it comes to insurance.

\textbf{Health care} --- I believe the best way to save on health care is to not get sick. This may sound smug, but I'm serious. I try to eat fairly healthy, drink plenty of water, exercise regularly, minimize stress, drink alcohol in moderation, avoid pollution, wear sunscreen, get plenty of rest, and not smoke. These became easier to do when I wasn't spending so much time working.

I'm very proactive about my health. I don't wait until I get sick before I start thinking about my health. I get regular physical exams. I do my own research in addition to consulting doctors. I go above and beyond the doctor's prescribed treatment. I try to treat my health as the most precious thing I have, because it is.

Some of the biggest cost savings I've had in this category came from just asking around. I've discovered how important it is to tell my doctors that I'm trying to keep my costs down. When my physician heard this, he skipped the most expensive and least beneficial tests, and ordered some of the others less frequently. In another case, I complained to a doctor about the high cost of the medications she prescribed. I didn't need a high dosage, and she had plenty of free samples, so she just sent me home with a bag of those.

I've also struggled to find a high quality, inexpensive dentist. I've tried HMO dental insurance and discovered that it is worthless. The low quality dentists the insurance required made up for their loss in earnings with aggressive and expensive treatment plans, much of which wasn't even covered by the insurance. In frustration, I went back to a high quality dentist and paid cash. I actually saved money this way because she was honest in her treatment plans and did good work that did not require expensive follow-ups. 

But still, I was spending a lot of money, so I kept asking around. Someone suggested I can still see the quality dentist for check-ups and dental work, but get free cleanings at the local community college in exchange for being a training subject for student dental hygienists. The trade-off is that the appointments take a lot more time than a normal cleaning. They have to be extremely thorough, and all their work must be checked by several teachers. I appreciated this thoroughness and, being retired, I had the time to spare. Also, it was kind of fun being their guinea pig. I liked that I was contributing to their education.

Health insurance is a very large and important topic, so I will cover it separately in Chapter 5.

\textbf{Transportation}
This is one of the highest costs for most people in developed countries, especially for those who own cars. The main alternatives to driving are bicycling and public transportation. The feasibility of these depend on where you live. Most of the expenses of owning a car---insurance, registration, maintenance, and depreciation---have to be paid even if you don't drive it. Where I live, a car is necessary and public transportation is expensive. But if I were to move, I would seriously look into public transportation options. It varies widely from place to place.

Bicycling is extremely cost effective. You only have the cost of the bicycle itself and occasional maintenence, both of which are inexpensive. For short trips, it doesn't take that much more time than driving, and you get exercise along the way. If you enjoy bicycling, this is a no-brainer.

The best ways I've found to keep driving costs down is to own an inexpensive car, drive safely, shop around for maintenence costs, and drive as seldom as possible. I own a quality car that has lasted me a long time, and I plan to keep it for a lot longer because I take good care of it and maintain it proactively.

I take my car to the dealer for regular maintenence. They charge about the same price as other shops for basic maintenence such as oil changes. Whenever they recommend any kind of high cost work, I shop around. I have found the difference in cost to be quite drastic in most cases. For example, they wanted to charge \$600 to replace my tires. I went to another shop and paid only \$260.

When I started my financial independence goal, my unsafe driving led to a car accident and a ticket. The accident made my insurance premiums go way up, and the ticket was also quite expensive. When I was looking at my biggest expenses, I saw my high insurance rates, repair costs, and ticket as a large portion of my transportation costs. All of this, I realized, was largely preventable.

Initially I started driving more safely to save money, but becoming more conscious of safety made me realize just how important it is anyway. Driving is one of the riskiest behaviors we engage in, I realized. In America in 2013 alone, there were over 30,000 deaths caused by car crashes.\cite{fatality-facts} This is 10 times the number of people who died in the 9/11 attacks.\cite{wikipedia-9-11} If I were to scale my fears appropriately, I figured, I'd be 10 times more afraid of driving than I am of an annual, large-scale terrorist attack.

And that's just fatalities. Far more common are accidents causing major injuries, disabilities, and property damage, which happens 2.35 million times a year! It's estimated this costs Americans \$230.6 billion annually.\cite{asirt-crash-statistics} This money isn't spent on financial independence or even increasing our quality of life, I realized. It just felt like a tragic waste.

That isn't to say I felt paralyzed with fear when I got behind the wheel. I just took safety very seriously. I drive slowly. I stop completely at every stop sign. I use my blinker incessantly, always keep my eyes moving, and avoid unnecessary lane changes. I completely avoid distracting myself with cell phones, food, or maps while I drive.

Over time, my insurance rates dropped, and I never had another accident or got a ticket again.

\textbf{Software} --- I don't pay for software. There is a ton of quality, free, open source software available, which I've found to be more stable and secure. It also gives me a lot more freedom in what I can do with my computer. It works well on older hardware, allowing me to get more value from my computer. If I were to buy a new computer, with all the necessary software, upgrades, and anti-virus protection, the costs can go over \$1,000, about every five years.

Instead, I buy a used, out-dated computer for about \$200. I install Linux and a huge host of software for free and run it for over 10 years. It never crashes or gets viruses, and it's much faster on older hardware than expensive operating systems are on newer computers. This saves me hundreds of dollars a year.

\textbf{Heating and cooling} --- In exploring ways to cut my utility costs, I've discovered that I have a comfort zone, a range of temperatures in which I'm comfortable. I experimented with this comfort zone to determine the exact temperatures. During the winter, I keep the temperature at the low end of that range, and during the summer, I keep it at the high end. I learned how my thermostat works, and made sure it shuts off when I don't need it. Another way I've saved money on temperature control is to get good insulation and dual-paned windows.

\textbf{Food} --- I eat a lot of quick, inexpensive meals. This costs a fraction of what I pay at a restaurant and without taxes or tips. I buy generic brands, compare prices, eat simple foods, and save the extravagant meals for special occasions.

\textbf{Milk} --- Ever since I was a kid, few things in this world brought me as much joy as a simple glass of chocolate milk. It just makes my life better. It's my comfort food and my favorite snack. When I decided to become financially independent, I knew I'd always make room in my spending for plenty of chocolate milk.

Then I became lactose intolerant and I had to buy lactose-free milk, which quadrupled the cost. I was already buying large quantities, so I suddenly found myself spending \$65 a month for milk. I searched for years to find a way to lower these expenses, and eventually discovered that I can easily make my own lactose-free milk. I just needed to put drops of lactase enzyme and wait 24 hours. I tried it, and it worked. The cost of milk plus the lactase drops together is only \$45 a month. That's \$20 per month savings with no change to my lifestyle.

\textbf{Investments} --- I keep my investment costs low. I discuss this more in Chapter 3. It's very common for investors to throw away a lot of money for the same product (or quite often, an even worse product) and be completely unaware of it.

\textbf{Entertainment} --- I tried hard to re-think my idea of fun. There are a lot of community events and free concerts in most cities. As a musician, I know that many of us love to play great music for attentive audiences, completely free of charge. I also like to stay at home, read library books, watch movies, play games, or spend time with people I care about.

\textbf{Music} --- A lot of independent musicians like to share their music online for free. I look for music that is licensed by Creative Commons, which makes it legal to download, copy, and share with my friends. I've found some extremely talented artists this way.

\textbf{Books and movies} --- I'm a big fan of my local library. I've been amazed by all the services they offer for free. My library carries all the latest movies and television shows and offers a service that will fetch whatever books I need from any of dozens of other libraries, all free of charge. They also offer free internet access and very inexpensive printing and photocopies. This way, I don't need to own a printer or scanner.

\textbf{Spending habits} --- I try very hard to stay ``behind the curve.'' I never buy the latest and greatest. Newer is not necessarily better. Keeping up with the trends is extremely expensive. Aside from the higher cost of frequently buying new things when the demand and cost are both highest, the new things also tend to be less reliable and take a lot of time to figure out how to use. I maintain and use up what I buy. I only buy something new when the old thing isn't useful anymore. When I'm ready for something new, I try to buy the old technology. For example, I didn't switch from VHS to DVD until Blu-Ray came out.

\textbf{Wishlist} --- I actively maintain a list of things I have some desire to buy. When I want to buy something, I just add it to the list. Then I wait. It could be some definite amount of time or some financial goal that I want to reach. This tempers my impulses and helps me overcome my desire for instant gratification. Sometimes, I put something on the list only to find later that I really didn't want it after all. Just the act of writing things down can help alleviate my desire for it, because I've acknowledged my desire and given myself some target for when I'll get it.

Something I really enjoy about my wishlist is that it creates what I call the ``Christmas effect.'' The anticipation during the time before opening a gift is so often more delicious than the gift itself. I like to savor this excitement for a while, even when I know I can afford to just buy it.

\textbf{Taxes} --- I've found that the best ways to save on taxes is to keep my spending down, use tax shelters as much as possible, buy and hold my investments, and buy a home. Using these techniques, I have been able to bring my income taxes down to zero. I talk more about taxes in Chapter 3.

\textbf{Housing} --- This is the largest cost most people have. We pay by location and square footage. Square footage depends partly on how many people are living together but mostly on how much stuff we have. I own very little and live in tiny room, renting out two other rooms to others.

The average American house is 2,349 square feet, more than twice the average size in the 1950's.\cite{npr-dream-house} Our bodies are less than 20 square feet. What is the other 2,329 square feet for? Our stuff. The less stuff we own, the less space we need. The less space we have for our stuff, the less cleaning and maintenance we need to do. The less time and money we spend on housing, the more of our life we get to use for other things.

But the biggest factor in housing costs is location, especially the quality of the local school district, crime rate, employment levels, and access to transportation and facilities. You pay a premium for each of these that your local area offers, so it can help to live at location that meets all of your needs but nothing more. I don't have kids, so I don't need a good school district, for example.

\textbf{Credit cards} --- I use credit cards as much as possible. This is probably the opposite of what you hear in most financial books.

Credit card companies profit from people's lack of discipline. Their interest rates can run as high as 20\%-30\%. One credit card charges 79.9\%, and this is actually legal.\cite{creditcards-high-interest} These companies profit handsomely from people who carry a balance. However, I've discovered that if I can be careful in my spending, then I can profit from the credit card companies instead.

I use a credit card that has no annual fee and gives cash rewards. I use this card for everything I buy. Then I pay off the balance in full each month. When I do it this way, I pay no fees or interest. Using the cash rewards, I cut all of my costs by 1\%-3\%. That's free money, just by playing the credit card companies' game and winning.

Of course, there's a huge difference between 1\%-3\% and 20\%-30\%. The cost of failure is much higher than the reward for winning. So don't play this game if you're not certain you can win. If you have trouble paying your balances each month, this is a very bad move. Not paying even one statement in full can cost you heavily in interest fees that month. After this, many credit cards continue charging fees in the future, even if you do pay off your balance every month after that. If you already have credit card debt, this also isn't a good choice. I talk more about debt in Chapter 4 and Chapter 6.

\textbf{Special occasions} --- Holidays, weddings, and funerals are occasions when even some of the most frugal behave as though money was no object. These occasions are when people are most susceptible to the fear of being perceived as ``cheap.'' This led to specialized industries that intentionally hide their true costs to exploit this fear.

Some holidays, such as Valentine's Day and Christmas, started off very simple and frugal. Valentine's Day used to be St. Valentine's Day, the the honoring of a legendary Catholic saint who was imprisoned for performing forbidden weddings. It wasn't commercialized until the 19th century.\cite{valentines-day} Christmas was an annual feast in celebration of the birth of Jesus Christ, inspired by pagan festivals popular at that time of year. It has always had a commercial element, but it didn't really heat up until F.W. Woolworth Company and Mongomery Ward discovered this gold mine in the 19th and 20th centuries, inventing much of the modern mythology of Christmas.\cite{remembering-woolworths}

Nowadays, gifts are often used, not to express love or gratitude, but to impress people. Sometimes, they're used for both purposes, and you'll hear justifications such as, ``nothing but the best for my baby.'' As if wasting money is somehow beneficial to the loved one. When trade-offs are totally overlooked like this, people end up making bizarre choices, such as taking a second mortgage on their house to pay for a diamond ring.

I try to become \emph{more} frugal on special occasions, not less. On holidays, such as Valentine's Day and Christmas, I don't buy gifts. If I feel moved to buy a gift for someone, I want this to be motivated purely by my generosity and love, not guilt or duty. I don't want to give people the gift of guilt.

I've made the choice that works best for me, but if you love these holidays and take great joy in gift-giving on these occasions, then you should do it. Still, I do recommend you be very careful about weddings and funerals. These are extremely emotional times for people, and trying to be frugal feels like it cheapens these occasions. There is no shortage of hucksters ready to cash in on the desire people have to put love before money, overcharging their clients and hiding the costs from them until the end.

\textbf{DIY} --- ``Do-it-yourself,'' or DIY, is a totally different attitude toward how you fulfill your needs. In a modern capitalist society, there is a strong commercial motivation for people to meet their needs through the acquisition of pre-packaged goods. So many goods are easily accessible and often cheap that it just makes sense to fulfill our needs in this way. But if your goal is frugality, it's helpful to remember that this wasn't always the way things were, and still isn't in many parts of the world. It's still possible to meet our own needs through creativity, ingenuity, and adaptation.

There's a whole subculture that supports this, including workshops, conventions, and meetup groups. Sometimes they're called ``makers.'' They're people who get a thrill out of solving their own problems, making many of the things they need. They repair their own cars, grow their own food, sew their own clothes, and engineer all sorts of clever solutions to everyday problems. The cost savings can be enormous, and no list of how to cut costs is complete without it. I'm not a maker, but I find this general attitude very helpful and inspiring. I make an effort to think twice before I rush out to buy something, asking myself if a DIY solution might make sense.

These are all ways of cutting costs that I have found. A thorough list would be endless, and there have been many books written about it. The point here is to give you a little taste of some things that have worked for me personally. Most of the examples I gave are just about being smart in your spending. I gave a few specific cases that I had a dollar figure for how much I'd saved. Many of these were spending habits that added up to thousands of dollars that I didn't need to earn and save in order to maintain. There were dozens, maybe even hundreds of these little things that I uncovered when I looked for them. They all add up, and have allowed me to redirect that money to my investments, which then had more time to appreciate in value.

\section{Voluntary simplicity}
Frugality is obviously not a new idea. The miserly form of frugality has a long history, which is why people think frugality means being stingy, and why I had to spend so much time debunking this. After the wasteful 1980's, frugality started making a comeback. People find that it can actually lead to a life of abundance, rather than the miserly life of deprivation. There has been a growing movement in the last few decades of people who feel overwhelmed by our consumer culture and are seeking a new lifestyle and community of others who share their values. It's called \emph{voluntary simplicity.}

Duane Elgin, who wrote one of the first books on this, defines voluntary simplicity as ``a lifestyle that is outwardly simple but inwardly rich.''\cite{voluntary-simplicity} This inner richness involves more than just spending less money and having fewer things. It's about having an awareness of our values and life purpose, and making sure that all of our choices are in line with those.

There are several books and websites devoted to voluntary simplicity as well as discussion groups for people to find support. A community of like-minded people is very helpful for such a significant lifestyle shift. See the Resources section for more information.

\newpage
\section{Resources}
\begin{itemize}
\item \textbf{\emph{The Complete Tightwad Gazette} by Amy Dacyczyn.} This is a collection of every issue of \emph{The Tightwad Gazette,} a newsletter that promoted thrift as a lifestyle, and provided thousands of tips for saving money without sacrificing quality of life. Dacyczyn was a master at frugality, emphasizing DIY within the context of raising a large family. Some of her tips may seem out-dated or irrelevant to you. She constantly reminds the reader that the tips themselves aren't the point, but the attitude and thought process they demonstrate.

\item \textbf{\url{http://www.bankrate.com/}} --- An excellent tool for finding and comparing financial institutions. You can use this to find the best rates for internet checking, credit cards, mortgages, savings accounts, and more.

\item \textbf{\url{http://www.linuxmint.com/}} --- The website for Linux Mint, a very easy and popular Linux distribution. I recommend the Xfce edition.

\item \textbf{\url{http://www.jamendo.com/}} --- A music service for Creative Commons artists. Use this to find free, excellent music.

\item \textbf{\url{http://www.simplicitycollective.com/}} --- Articles, blog, and forum devoted to the voluntary simplicity lifestyle.

\item \textbf{\url{http://www.moneychimp.com/calculator/compound\_interest\_calculator.htm}} --- A good compound interest calculator.
\end{itemize}
