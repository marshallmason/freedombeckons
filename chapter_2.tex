\chapter{Frugal Does Not Mean Stingy}

Money is a touchy subject. It's taboo to talk about how much money we earn, but it's fine to talk about how much we spend. For example, it's easy to ask how much someone paid for their new car, but it's not so easy to ask about their annual salary. In other ways, we're more sensitive about spending than earning: it's fine to talk about what we do for a living, but our attitudes toward spending are often labeled with judgmental words like \emph{miserly, selfish, stingy,} and \emph{greedy.} Even positive words like \emph{generous} are still judgments of someone's character.

\section{Generosity}
Frugality doesn't mean giving less to others. It often means giving more. This is the most important difference between frugality and stinginess. Eq\-uating frugality with being less generous is a common misnomer.

One of the most celebrated portrayals of stinginess in popular culture is the story, \emph{A Christmas Carol} by Charles Dickens, and its many movie renditions. It exaggerates all the worst aspects of stinginess---selfishness, greed, and a blatant disregard for the suffering of others. The main character, Ebenezer Scrooge, has an epiphany one night in which he discovers the true costs of his stinginess. This leads him to become wasteful, spending lavishly on others with no regard to cost.

This kind of portrayal confuses people about the true meaning of frugality. It presents wasteful spending as the antithesis to and the remedy for stinginess and frugality, which are portrayed as synonomous. Ironically, since the publication of \emph{A Christmas Carol}, Christmas has transformed into a wasteful, greedy holiday. Much of the generosity and spirit of brotherhood seen in this story has been replaced with stress and shopping.

By contrast, my favorite example of frugality is Warren Buffett, a famous investor and business owner, and one of the richest people on the planet. He lives in a modest house, drives a modest car, and shops at thrift stores. His entire investing philosophy is based on frugality. He's an advocate for low-cost investing. He made his billions by buying and holding well-managed but underpriced companies. He donated 85\% of his wealth to a foundation that works to reduce global poverty and improve healthcare in developing countries.

This is one way frugal people can be more generous---by how they allocate their spending. They spend less money on frivolous personal items that don't bring much satisfaction so they may have more to donate to causes they believe in. Often, frugality is motivated by a stark awareness of the suffering of others.

When people talk about generosity, they usually mean how much money they're willing to spend on other people. Money is convenient because it is easy to measure. Giving money can be a powerful form of generosity, but there are other forms of generosity.

You can also give your time. Giving time is generous if it's quality time that helps others in meaningful ways. It can be more powerful to donate your time than money. It's more personal, and often more necessary. Many well-funded organizations are strapped for manpower. 

Freedom itself can be a form a generosity. Since some needs are more urgent than others, I've found it invaluable to have the freedom to drop everything I'm doing and offer myself to someone for as long as they need me. This is the most rewarding form of generosity for me. Giving money to a faceless organization almost feels cheap by comparison.

Gift-giving is another form of generosity that can use some re-thinking. I rarely buy gifts, especially on Christmas and birthdays. I only give a gift if I can do so completely from the heart, I know the person will benefit from it, and it fits my budget. When those conditions are met, it seems silly to wait for a special holiday. Why not give the gift when I feel inspired to, rather than wait for society to sanction it? Some people truly enjoy giving and receiving gifts, and they love holidays. I can see how wonderful that is for them, but buying stuff is not the way I show love.

\section{Frugality}
Stinginess means detesting the spending of money. Misers save money for the sake of saving money, hoarding it for themselves. This often stems from fear, insecurity, or greed. Spending or saving money always entails values and trade-offs, but for misers, these important aspects of money are often obscured by their emotions.

Spending money can be like voting. How you spend your money is a statement of your values. Someone can tell a lot about you by looking at your bank statement or credit card statement. Your spending habits indicate which things you value enough to trade for money. Everything on that list is something you've chosen among many options, each catering to different values: taste, cost, trust, safety, brand name recognition, compassion, sustainability, etc.

Everything we buy is a trade-off we've chosen to make. We may choose to spend more for something of higher quality or a recognizable brand name, for example. We may choose to shop down the street or we may choose to shop in the next town. For every purchase we choose to make, we have implicitly chosen to not make countless other purchases with that money. Whenever we spend money, we always give up other things in order to get something else that we value more, even if we're not aware of it.

The choice to buy or not to buy is also a trade-off---the choice to work and spend rather than not work and not spend. The money you spend has to come from somewhere. The more you spend, the more you have to work, so the less you spend, the less you ultimately have to work. The goal of financial independence is gaining the freedom to work less, or for less money, which means spending less.

Consider how much companies spend on marketing each year, and consider what form most of that marketing takes. Ads often target our psychological weaknesses, particularly our emotional reasoning. These ads work so well because people are unaware of their motivations for buying or the trade-offs they're making. Most marketing thrives on ignorance.

The goal of frugality is to combat this ignorance, so that all your spending decisions are the right ones for you. This usually means spending less money, but it may also mean spending more money in certain areas.

Frugality is about awareness. Frugality doesn't mean worrying about every penny you spend, or sacrificing your happiness for money. That's what misers do. Frugality is about knowing exactly what you're buying, why you're buying it, and what it costs you in real life terms. It's not an act of minimizing, but of optimizing.

If you become aware of what you really need, you can seek ways to address those needs more directly, rather than indirectly through buying something. Buying things rarely gives us what the ads imply, like status, approval, lovers, and happiness, so we end up buying more and more in our on-going pursuit of happiness. Awareness can break that cycle. Frugality isn't about sacrificing happiness for money. \emph{It's about maximizing both.}

\section{Life exchange rate}
You have to trade some of your time for money, so there is a one-to-one connection between the money you spend and your life, the precious time you have on this planet. That doesn't mean you're dead while you're working, although some people feel that way. It just means the time doesn't belong to you. You've sold that piece of your life to someone else. To some extent, you can choose what kind of work you do, but while you're working, you don't get to choose anything else because that choice no longer belongs to you.

Money is abstract and has many emotional connotations. This makes it harder to connect with. But since you can literally treat money and life as interchangeable, let's talk about life instead.

Just like money, you spend life, wisely or unwisely. It's something you probably cherish and want to maximize. You trade your life for money, and like any conversion between interchangeable currencies, there is an exchange rate. You can quantify exactly how much of your life you're spending whenever you spend money: your \emph{life exchange rate.}

Start with your gross annual wage. Deduct from this all work-related costs: income taxes, FICA, work clothes, commuting costs, education, lunches you buy because you're in a hurry, office supplies, day care costs, union dues, etc. Then add in any unpaid benefits you receive, such as health insurance and free gym memberships. Make sure these are all in annual terms. For example, if you estimate a health membership would cost you \$25 per month, multiply \$25 by 12 months.

Now calculate how much time you spend on your job each year, working, commuting, preparing for work, unwinding from work, keeping up with your field, education, sleepless nights, time with doctors for physical or mental work-related illnesses, networking, etc. Divide your net income above by the total hours per year you spend for your job. The result is your life exchange rate.

Translating the money you spend into life can give some perspective on what trade-offs you're really making when you spend money. Whenever you're considering spending money, try dividing its cost by this number. For example, let's say your life exchange rate is \$20 per hour. You're considering going out to eat rather than cooking for yourself. The restaurant you have in mind would cost \$15. You figure you can cook for yourself for about \$5. So the premium you pay by making this choice is \$10. Divide this by \$20 per hour, and you'll get a half an hour.

Don't think of that purchase as \$10. That's too abstract. It obscures its true cost. Instead, think of it as 30 minutes of your life. Is it worth it? Would you rather work for 30 minutes and eat at the restaurant, or have the option to not work for 30 minutes? That's the trade-off you have to consider.

\section{Opportunity cost}
The life exchange rate only accounts for the trade-off between time and money. It doesn't account for the trade-off between spending and investing. In investing parlance, this is called \emph{opportunity cost.} If you invest money rather than spend it, then each year it earns more money that you would have otherwise had to earn from your job.

There is a handy rule-of-thumb called the \emph{Rule of 72.} If you divide 72 by an annual return rate (expressed as a whole number), you get the approximate number of years it would take for your initial capital to double. If you expect to earn 8\% from your investments, then it would take 72 $\div$ 8 = 9 years for it to double.

I'll talk more about investing in the next chapter. The Rule of 72 is an investing concept, but I introduce it here because it is helpful in gaining an awareness of the trade-off between spending and investing.

How soon would you like to achieve financial independence? For me, it was 10 years. According to the Rule of 72, money invested at the beginning would be worth slightly more than double by the time I wanted to reach my goal. That means all the money I spent at the beginning cost twice as many hours of my life than what is suggested by the life exchange rate alone. In order to factor in the opportunity cost of spending money rather than investing it, everything costs twice as much.

Going back to the example of considering whether you should eat at a restaurant, the premium you have to pay to eat there is no longer \$10, but \$20. It doesn't just cost you 30 minutes of your life, but a full hour.

\section{Spending habits}
Spending choices are usually part of an overall \emph{spending habit.} Small changes to spending habits can reap huge rewards over time, so it's also important to gain an awareness of them.

Here's another useful rule-of-thumb: for every dollar you spend each year, you must earn and save \$25 in order to be financially independent. This is called the \emph{Rule of 25.} You can also multiply this by 12 to give you a \emph{Rule of 300} for monthly expenses. For every dollar you spend each month, you must earn and save \$300. In Chapter 5, I'll explain the significance of this and how it was arrived at. For now, let's see how it can bring more awareness to your trade-offs.

Whenever you want to spend money, ask yourself how often you expect to buy it. Using our previous example, let's say you'd like to eat at a restaurant once a week. That's 52 times per year. At \$10 per meal, you'll be spending \$520 per year. Using the Rule of 25, you'll need to earn and save \$13,000 to cover this cost in order to be financially independent. If your life exchange rate is \$20, as in the previous example, then this restaurant habit costs you 650 hours of your life. If work consumes 50 hours per week,\footnote{Remember, this isn't just worked time, but all time you devote to your job, including commuting, preparing for work, and unwinding after work.} it would take you about $3\frac{1}{4}$ months to earn this.

We can also calculate the opportunity cost of spending habits. Since we're dealing with an on-going cost rather than a one-time cost, we can't use the Rule of 72. Instead, we need a \emph{compound interest calculator.}\footnote{These are pretty easy to find. Most spreadsheets have one, as well as many websites. I've included one in the Resources section.} This takes an initial amount, an annual investment, an interest rate, and a number of years, and tells you how much money you'd have saved by the end, accounting for the interest you'd have accrued along the way.

Let's use a compound interest calculator to determine the opportunity cost of this restaurant habit. Let's say you want to achieve financial independence in 10 years, as in the previous example, and that you expect your investments to earn 8\%, after inflation. Your initial amount is \$0 and your annual investment is \$520, the annual cost of your restaurant habit. The calculator says you'd have about \$8,136 if you'd have invested this money rather than spent it for 10 years. The interest portion of this is \$2,936 ($\$8,136 - \$520 \times 10$). This is the cost of spending the money rather than saving it. If you had invested it rather than spending it, you wouldn't need to work in order to accumulate this three grand. It would have literally earned itself.

Using your life exchange rate, again using \$20 per hour, it would take you about 147 hours to earn the money that could have been earned for you, or almost three and a half weeks at work. Add this to the 650 hours already calculated for the habit itself. In total, the true cost of a weekly habit of eating at a restaurant is 797 hours of your life, nearly four months of working full time.

\section{Doing your own calculations}
I made several assumptions here. I chose 10 years for the time horizon; I used \$20 per hour as the life exchange rate; 8\% was the assumed the return rate; 50 hours per week was used to determine how many months you'd have to spend at work. I also did the calculation for only one expense, the weekly restaurant habit.

These numbers will be different for each person. The life exchange rate and time spent at work you've already calculated earlier. The 8\% figure is based on average returns, after inflation, taxes, and expenses.\footnote{This is only a guess, useful for our purposes here. Predicting this number exactly is impossible, and guessing is frought with many assumptions. I'll discuss this more in Chapter 3. For now, it's fine to use 8\%.}

The numbers will also be different for each spending habit. It's possible to run through this routine for every spending habit you have, but it would be very tedious and unnecessary. It's also not very feasible to whip out your financial calculator and notepad whenever you're out shopping.

One solution is to plug them into a spreadsheet program. Enter your own values in the cells and use the built-in compound interest calculator. Any time you want to determine how much of your life it will cost you for any given spending habit, you can simply change a couple of cells in a spreadsheet.

But even this isn't really necessary. The only point of this exercise is to give you a sense for the true costs of your spending habits. You won't need to calculate the cost of very many spending habits before you'll start to get a feel for the kinds of purchases that cost the most.

I rarely needed to calculate the life cost to know whether a spending habit would or wouldn't be worth it. This was especially true for those spending habits I guessed would \emph{not} be worth it---the risk of being wrong was zero. When I guessed it \emph{would} be worth it, I could always double-check later with a real calculation, and then stop the habit if necessary. It never was.

If you're good at simple arithmetic, you could also memorize the life cost of a few key spending habits, and then twist that number around for other habits. For example, we've already determined that spending \$10 per week costs 797 hours of your life, or four months of working. If instead you're considering spending \$40 per month, that's the same thing as \$10 per week, so that too will cost you four months. Let's say you're thinking about spending \$5 per week---you can just cut it in half, so it would cost you two months. If you're considering spending \$50 per month, that's just a bit over \$40 per month, so you know it will cost you a bit over four months. Precision is unimportant.

Here's a summary of the full process to calculate the life cost of a spending habit:

\begin{enumerate}
\item Normalize to annual costs: multiply monthly costs by 12, weekly costs by 52, and daily costs by 365.
\item Multiply by 25. This is cost of the spending habit. (\emph{Rule of 25})
\item Use a \emph{compound interest calculator} to calculate how much you'd have if you'd invested this money instead of spending it.
\begin{itemize}
\item Use 0 for the initial cost.
\item Use the annual cost from step 1 for the annual investment.
\item Use the time between now and your target date for financial independence for the growth time.
\item Use 8\% for the growth rate, or whatever you decide after reading Chapter 3.
\end{itemize}
\item Subtract from this the annual cost from step 1 times the number of years between now and your target date from step 4. This is your \emph{opportunity cost.}
\item Add this to the cost of the spending habit from step 2. This is the total cost.
\item Divide this by your \emph{life exchange rate} to get the life cost.
\item Optionally, divide this by the number of hours you work per month to determine how many months you would need to work to accomodate this habit.
\end{enumerate}

Here's one more example, using different numbers. Let's say you're considering subscribing to a premium web service, such as Netflix or Amazon Prime. It costs \$10 per month, a pretty typical cost of such a service at the time I wrote this. Let's use some different numbers. Let's say you want to achieve financial independence 13 years from now. You estimate the growth rate of your investments will be 7\%. Your life exchange rate is \$25 per hour, and you work 50 hours per week.

\begin{enumerate}
\item The annual cost is \$10 $\times$ 12 months = \$120.
\item The spending habit costs $\$120 \times 25 = \$3,000$.
\item The money you'd have if you invested this instead, from the compound interest calculator, is \$2,586.
\item The opportunity cost is $\$2,586 - \$120 \times 13$ years = \$1,026.
\item The total cost is $\$3,000 + \$1,026 = \$4,026$.
\item The life cost is \$4,026 $\div$ \$25 per hour = 161 hours.
\item It would take you 161 $\div$ 50 hours per week = 3.22 weeks to pay for this web service.
\end{enumerate}

Again, you can twist this around in many ways to help you estimate the total cost of other purchases. If you're considering a habit of \$20 per month instead of \$10, then just double the total cost: $6\frac{1}{2}$~weeks, or about $1\frac{1}{2}$~months. If you want to spend \$5 per month, just cut it in half: just over $1\frac{1}{2}$~weeks.

\section{A new perspective on spending}
Every time I pulled out my wallet, I asked myself how much of my life I'd be spending. Seemingly small costs, made frequently, added up in ways that often astonished me. It gave me a whole new perspective on my spending. Over time, it became second nature for me to think about spending in this way. People would be surprised that I would easily drop \$50 on a DVD player, but cringe at spending \$0.75 for a soda. \$50 every few years is much cheaper than a daily soda habit.

Knowing the life costs of your spending, compare that to how much fulfillment, satisfaction, and value you're getting, and how much it's in line with your values and life purpose. If the life cost is low, the fulfillment and satisfaction are high, and it's very much in line with your values and life purpose, then spend it! If not, then celebrate your choice for something greater rather than something smaller. This is the kind of thought that should go into every purchase you make.

I was grateful for every dollar I didn't spend. Sometimes I would need to do some calculations. Especially at first, I was shocked by how much it was really costing me. Choosing not to spend that money felt so rewarding because I knew I was getting something so much better in return.

Of course, sometimes I made this calculation and there was no question that it was worth every hour of life it cost me. My guitar is a great example. After playing cheap guitars all my life, I finally bought an expensive, high-quality guitar. I didn't wring my hands over it like misers would. I had no reason to worry about my choice, and I had no regrets because I'd quantified exactly how much of my life I was giving up for it, and I knew without a doubt that this instrument that I would own for decades, would enjoy playing every single day, and was in line with values by helping me bring music to the world, was well worth the life cost. I still play that very guitar every day.

The trade-offs you make are your choice. I only urge that you be conscious of this choice. Calculate how much life you're trading for what you spend. If you find that it's not worth it, then don't spend it. But if you find that it is worth it, then you should spend it, or spend \emph{even more} where it makes sense. How's that for not being stingy?

\section{Abundance}
Budgets are like the financial equivalent of diets. Instead of counting calories, you count dollars. Budgeters decide each month how much they're allowed to spend in various categories, and then try not to exceed that amount each month. I recommend that you not budget in this way, unless you know it works for you. The reason is that it comes from a mindset of sacrifice. This is not a very motivating concept, and may turn purchases into ``forbidden fruits,'' which only makes you want them more. But the worst part is how miserable it is to live life with an attitude of sacrifice.

I find \emph{abundance} to be a far more motivating concept. Imagine how many opportunities and choices you're blessed with in life, especially if you live in a free country.

As I write this, I'm in my 30's. My life expectancy is in my 70's. That's 40 years. There are 365 days in a year and 86,400 seconds in a day. That's 1,261,440,000 seconds. That's an enormous abundance of choices I have open to me.

Every moment, every waking second of every day, we're making choices, whether we're aware of it or not. The only choice we don't have is the choice to not make a choice. You could make choices that squander a lot of those precious seconds, or you could make every second count. You could trade in a small portion of those seconds for something that will bring immense joy. What a great trade that would be. Or you might decide that the portion you'd trade is too large, and the joy you'd get from the trade too small, so you decide to keep those precious seconds for yourself, to trade in later for some activity that will bring immense joy. That would be a great trade as well.

The only poor choices are the ones that don't bring you the maximum joy for the precious seconds you trade for it. If you can start thinking in terms of life instead of dollars, you can become more conscious of what trade you're actually making. You're more likely to make the right choices for yourself, based on your own values. So there's really no reason to budget, or to sacrifice. Every choice you make, to spend or not to spend, or how you spend, is a choice of abundance. It's only a matter of shifting your perspective and attitude. No more budgets and no more sacrifice! Isn't that a relief?

That doesn't mean you shouldn't keep records or make plans. There's no better way to be fully conscious of your trade-offs than to have thorough records of the choices you're making, which I'll discuss in Chapter 4. I'm only suggesting that you base your spending choices, not on preconceived limits, but on a moment-to-moment awareness of the abundance of life and your desire to get the maximum value from that immense, but nonetheless finite, abundance.

What this is really about is optimizing your life, not a chore in which you hoard every penny you have. If it feels like a chore, you're probably doing it wrong.

\section{Easy ways to cut costs}
The records you keep will give you an idea of which costs to focus on. You'll get a sense for the ``hot spots,'' those expenses that are easiest to cut while bringing you the greatest gains. There's a concept that seems to be true for most human activity, called the \emph{Pareto principle,} or \emph{80/20 rule}: 80\% of the effects come from 20\% of the causes. Those are the expenses you want to focus on.

Try it. You might be amazed to find that, just by poking around a little, you can cut costs dramatically, with little or no impact on your quality of life whatsoever. Sometimes you may find cheaper ways to get the same exact things. Other times you may discover that the trade-offs you're making aren't appropriate, so you scale back on an expense in order to maximize something else you value more.

Inevitably, the spending choices you make will be different than mine or anyone else's. For that reason, I cannot tell you which expenses you should re-evaluate. All I can do is provide some general tips I've picked up over the years. Some or even all of these may not be relevant for you, but it's a useful exercise nonetheless because it serves as a good demonstration of the process of frugality. In that spirit, here are some ways I've found to significantly cut costs:

\textbf{Subscriptions} -- Be cautious about subscription services. These are used to lock consumers into an effortless, long-term consumption habit, with a built-in loyalty to the service provider, and to deceive you about the long-term costs of that habit. By focusing only on the monthly cost, rather than the total cost, things may appear cheaper than they actually are. By deducting the cost automatically from your credit card, the conscious process that we engage in when we spend money is circumvented. The 25 Rule comes in especially handy for determining the true costs of subscriptions. Subscriptions are common because of the ways they benefit the businesses selling them. Some services are no longer available on anything but a subscription basis, so you may be stuck with them. Just be cautious. Seek alternatives if possible, and when you stop deriving value from them, cancel them immediately.

\textbf{Cell phones} -- This is one of the most egregious form of conspicuous consumption I've seen in recent years, and a perfect example of services unnecessarily set up as subscriptions for the purpose of maximizing profits for the service providers. The iPhone in particular is more glamor than utility. Like all Apple products, what you're really paying for is the brand. This industry is still in its infancy. It seems every company on the planet is marketing their own cell phone. Use this to your advantage. Some cell phones cost less than land lines, while others cost hundreds of dollars a month.

Think seriously about whether you really need a ``smart phone.'' They cost more than simple cell phones and are much more likely to require an expensive subscription contract. Look for cheap cell phones with a pay-as-you-go plan. These have low costs and no contractual obligations. If you absolutely must have a smart phone, then look for a good phone at a good price and hang onto it. Don't get caught up in the unnecessarily accelerated upgrade cycle that comes with owning products in a growing and rapidly-changing industry.

\textbf{Internet banks} -- There are dozens of online banks which offer free checking and free bill pay and pay a small amount of interest. Some may even provide postage paid deposit envelopes and reimburse you for any ATM fees you incur. These banks don't have the high costs of branch offices or the large marketing budgets of the big banks, so they're able to pass the savings onto you.

\textbf{Online shopping} -- Another great way to use the internet to save money is to shop online. It takes a lot of time and gas to shop around in brick-and-morter shops, but it's quick and easy to do it online. Shipping costs can be prohibitive and that must be accounted for. I often find that I can save up to 50\% on some very expensive items, even after paying shipping. If you can't find a good price online, you haven't wasted much time, and you can go ahead and buy it at a local shop.

Even if you're familiar with online shopping, there still may be certain things you just never think to check online for. I used to pay my vet clinic \$100 for a 7-month supply of flea protection for my cats. That's \$14.29 a month. I'd done this for years before it occurred to me to check online. Within just a few minutes, I found a 12-month supply for \$81. Shipping costs were \$15, but if I bought at least \$100 worth, shipping was free. I could get free shipping if I was willing to buy two years worth. That's only \$6.75 a month, a savings of \$7.54 per month for the same exact product.

\textbf{Price book} -- A price book is a list of the cheapest prices you've found for things you buy regularly. It can be a notepad, a file on a PDA or smart phone, or a file on a computer. Whenever you need to buy something, check your price book first. Also note the size of what you buy and where you bought it. The more frequently you buy something, even tiny changes can make a huge difference.

\textbf{Insurance} -- Insurance is expensive and can vary widely. As long as they have a strong A.M. Best rating, all insurance companies offer the same thing. I found that shopping around for insurance could save me over a hundred dollars a year. Companies that don't have a big brand name tend to be cheaper even though they have the same A.M. Best rating. Their customers benefit from their smaller marketing budget.

It also helps to understand the exact details of your insurance policy. Some companies will sell you insurance you don't need. For example, if you have health insurance then you don't need medical coverage in your auto insurance. If you have a cheap car or you have enough money in the bank to pay for it yourself, it might be best to not have collision insurance.

Always buy plenty of liability insurance. If you were the only person involved in an accident, your costs are limited to the value of your car and medical bills. But most accidents involve many people, and this can multiply the costs of the accident significantly.

Having money in the bank can be the best insurance you can buy. Be willing to pay for large costs out-of-pocket. This will allow you to have high deductibles and low premiums. Many people assume that the benefits paid for by higher premium policies are free, especially in their health insurance. It's not. It's extremely expensive. You pay for all of it indirectly, through your premiums.

In some forms of insurance, such as auto insurance, every claim you make will be charged back to you in the future through higher premiums. If you get into a small accident and make a claim on your insurance, your premium will go way up. Only use insurance to cover the unexpected costs of catastrophes, not as a payment plan.

The insurance industry operates on paying less in claims than they make in premiums. Insurance companies hire experts who specialize in making such predictions with a high degree of accuracy. You should never ``bet against the house'' when it comes to insurance. When you pay high deductibles, the costs you pay will be more direct, so you will feel them more. This is exactly what you want if your goal is to be more conscious of your spending trade-offs.

\textbf{Health care} -- The best way to save on health care is to not get sick. This may sound smug, but I'm serious. Many illnesses are preventable. The ways to do this are so well-known as to be clich\'e. Eat healthy, don't over-eat, drink plenty of water, exercise regularly, minimize stress, don't smoke, drink in moderation, avoid pollution, wear sunscreen, and get plenty of rest. These are all easier to do when you don't spend as much of your time at a job.

Be proactive about your health. Don't wait until you get sick before you start thinking about your health. Get regular physical exams. When you do get sick, get opinions from multiple doctors. Do your own research in addition to consulting doctors. Go above and beyond the doctor's prescribed treatment. Always treat your health as the most precious thing you have, because it is.

\textbf{Transportation}
This is one of the highest costs for most people in developed countries, especially for those who own cars. The alternatives to driving are bicycling and public transportation. The feasibility of these depend on where you live. I've found that, unless I choose to not own a car at all, public transportation is too expensive. Most of the expenses of owning a car---insurance, registration, maintenance, and depreciation---have to be paid even if you don't drive it.

However, bicycling is always worth the cost. Even when you account for the extra time it takes to bicycle (which you could even treat as exercise time), bicycling always saves money for short trips. If you enjoy bicycling, this is a no-brainer.

If you drive a car, The only ways I've found to keep driving costs down is to own an inexpensive car, drive safely, and drive as seldom as possible. Shop around and get a good quality, used car that you know will last a long time. Take good care of it and maintain it proactively.

Don't take risks on the road. Driving is a pretty big risk by itself, without adding risky driving behavior. Drive below the speed limit and stop completely at every stop sign. Use your blinker and avoid unnecessary lane changes. Always keep your eyes moving. Avoid distracting yourself with cell phones, food, or maps while you drive. Your insurance costs will stay low if there are no recent tickets or accidents. By avoiding accidents, you also don't have to fix your car, deal with hefty medical bills, or experience unnecessary suffering. Plus, you'll live longer.

\textbf{Software} -- Don't pay for software. There is a ton of quality, free, open source software available, which is more stable and secure, and gives you a lot more freedom with what you can do with your computer. It also works well on older hardware, allowing you to get more value from your computer. When you buy a new computer, with all the necessary software, upgrades, and anti-virus protection, the costs can go over \$1,000, about every five years. Instead, I buy a used, out-dated computer for about \$200, install Linux and a huge host of software for free, run it for over 10 years, and it never crashes or gets viruses. I estimate this saves me \$180 per year, a savings of 90\%.

\textbf{Heating and cooling} -- This is a huge cost for many people. Just a little bit of research and diligence can save you a ton of money and you can still be very comfortable. Most people have a comfort zone, a range of comfortable temperatures. Get to know your comfort zone, the exact temperatures. During the winter, keep the temperature at the low end of that range, and during the summer, keep it at the high end. Learn how your thermostat works, and make sure it shuts off when you don't need it. We have a natural tendency to only mess with things when we're uncomfortable. Your heater or air conditioner might be on past when it needs to be, and you won't notice it because you're comfortable, so you won't change it. Another way to save money on temperature control is to get good insulation and dual-paned windows.

\textbf{Food} -- Learn to cook inexpensive meals. You can make some really yummy food, without much effort, for a fraction of what you pay at a restaurant and without taxes or tips. Buy generic brands. Use a price book at the grocery store. Eat simple foods, and save the extravagant meals for special occasions.

\textbf{Investments} -- Keep your investment costs low. I will discuss this more in Chapter 3. It's very common for investors to throw away a lot of money for the same product (or quite often, an even worse product!) and be completely unaware of it.

\textbf{Entertainment} -- Re-think fun. There are a lot of community events and free concerts in most cities. As a musician, I can assure you that many of us love to play great music for attentive audiences, completely free of charge. Or just stay at home. Read library books, watch movies, play games, or spend time with people you care about.

\textbf{Music} -- A lot of independent musicians also like to share their music online for free. If you find music that is licensed by Creative Commons, it's also legal for you to share it with your friends. Many independent artists are even more talented than famous ones, and since they aren't limited by a rigid ``pop formula'' they also tend to be more unique and diverse. The only downside is that they're not as familiar to you, but they will be once you start listening to them!

\textbf{Books and movies} -- Get to know your local library. I've been amazed by all the services they offer for free. My library carries all the latest movies and television shows and offers a service that will fetch whatever books I need from any of dozens of other libraries, all free of charge. They also offer free internet access and very inexpensive printing and photocopies.

\textbf{Spending habits} -- Many Americans like to have the latest and greatest. It's not because the newer things are better, but because it's exciting to play with new toys. This has huge costs. Aside from the higher cost of frequently buying new things when the demand and cost are both highest, the new things also tend to be less reliable, take a lot of time to figure out how to use, and are less likely to be standards-compliant. Consider changing your buying strategy from ``early and often'' to ``late and seldom.'' Maintain and use up what you buy. Only buy something new when the old thing isn't useful anymore. When you are ready for something new, stay behind the curve and buy something at least one step behind the latest. For example, I didn't switch from VHS to DVD until Blu-Ray came out.

\textbf{Wishlist} -- Maintain a list of things you have some desire to buy. When you want to buy something, add it to the list. Then wait. It could be some definite amount of time or some financial goal that you want to reach. This will temper your impulses and help you overcome your desire for instant gratification. Sometimes, you may find that you put something on the list, only to find later that you really didn't want it after all. Just the act of writing things down can help alleviate your desire for it, because you've acknowledged your desire and given yourself some target for when you'll get it. Sometimes we buy things just to alleviate the anxiety that arises when we tell ourselves we ``shouldn't'' buy them.

\textbf{Taxes} -- One way or another, taxes are incurred on spending, and sometimes you're double-taxed. First, there's income tax. The more you spend, the more you need to earn, and therefore the more you're taxed. Even if you spend less but earn more, you can still lower your taxes by using tax shelters. Therefore, in an indirect way, income tax is actually a form of sales tax. Then you're charged sales tax on the things you buy, except for food and drugs. You need a higher income to afford the sales tax, so you're taxed yet again.

The best ways to save on taxes is to keep your spending down, use tax shelters as much as possible, buy and hold your investments, and buy a home. Using these techniques, I have been able to bring my income taxes down to zero. I'll talk more about taxes in Chapter 3.

\textbf{Housing} -- This is the largest cost most people have. We pay by square footage, which depends partly on how many people are living together but mostly on how much stuff we have. The average American house is 2,349 square feet, more than twice the average size in the 1950's.\cite{npr-dream-house} Our bodies are less than 20 square feet. What is the other 2,329 square feet for? Our stuff. The less stuff we own, the less space we need. The less space we have for our stuff, the less cleaning and maintenance we need to do. The less time and money we spend on housing, the more of our life we get to use for other things.

The biggest factor in housing costs is location, especially the quality of the local school district, crime rate, employment levels, and access to transportation and facilities. You pay a premium for each of these that your local area offers. If you don't have kids, then you don't need a good school district. If you retire early, then you don't need any local jobs.

It also helps to share your home with others. One kitchen, living room, and bathroom can be used by two people almost as easily as by one person.

\textbf{Credit cards} -- Use credit cards as much as possible. This is probably the opposite of what you hear in most financial books. Credit card companies profit from ignorance and lack of discipline. If you can conquer those, which is best done by being conscious of trade-offs, then you can profit from the credit card companies instead. Find a credit card that has no annual fee and gives cash rewards. Use this card for everything you buy. Then pay off the balance in full each month. If you can do this, you'll pay no fees. Using the cash rewards, you can cut all of your costs by 1\% - 3\%. That's free money, just by playing the credit card companies' game and winning.

Don't play this game if you're not certain you can win. If you have trouble paying your balances each month, this is a very bad move. Not paying even one statement in full can cost you heavily in fees that month. After this, many credit cards continue charging fees in the future, even if you do pay off your balance every month after that. If you already have credit card debt, this also isn't a good choice. I'll talk more about debt in Chapter 4 and Chapter 6.

\textbf{Special occasions} -- Holidays, weddings, and funerals are occasions when even some of the most frugal behave as though money was no object. These occasions are when people are most susceptible to the fear of being perceived as ``cheap.'' This led to specialized industries that intentionally hide their true costs to exploit this fear. Their goal is to convince customers that they deserve ``nothing but the best'' for these important occasions.

Some holidays, such as Valentine's Day and Christmas, started off very simple and frugal. As they were commercialized, they became increasingly lavish, through a kind of arm's race in which people tried to out-do one another in their willingness to spend.

Gifts are often used, not to express love or gratitude, but to impress people. Sometimes, they're used for both purposes, and you'll hear justifications such as, ``nothing but the best for my baby.'' As if wasting money is somehow beneficial to the loved one. When trade-offs are totally overlooked like this, people end up making bizarre choices, such as taking a second mortgage on their house to pay for a diamond ring.

I highly recommend you become \emph{more} frugal on these occasions, not less. Instead of a diamond wedding ring, buy your spouse the gift of freedom, and get a cheap imitation. Then you can invest the remainder for your future together, of which the ring is merely a symbol. You and the person wearing it will be the only ones who will know it's not a real diamond.

On holidays, such as Valentine's Day and Christmas, think twice before buying gifts. Ask yourself why you feel compelled to buy gifts on these occasions, rather than any random day of the year. If guilt factors into it, is it really gracious to buy gifts out of guilt? If you don't enjoy shopping for and giving gifts, then don't do it. Find other, more personal ways to honor those you love.

Instead of an expensive funeral, you can honor the deceased better by putting them in a cheap wooden box, and donate the difference to a charity they were passionate about.

\textbf{DIY} -- ``Do-it-yourself,'' or DIY, is a totally different attitude toward how you fulfill your needs. There's a whole culture that supports this, including workshops, conventions, and meetup groups. Sometimes they're called ``makers.'' They're people who get a thrill out of solving their own problems, making many of the things they need. They repair their own cars, grow their own food, sew their own clothes, and engineer all sorts of clever solutions to everyday problems. This isn't something I care for personally, but it fascinates me. The cost savings can be enormous, and no list of how to cut costs is complete without it.

These are all ways of cutting costs that I have found. A thorough list of ways to cut costs would be endless, and there have been many books written about it. The point here is to give you a little taste. Most of the examples I gave are just about being smart in your spending, and don't require any change in your quality of life. I gave a few specific cases that I had a dollar figure for how much I'd saved. Many of these were spending habits that added up to thousands of dollars that I didn't need to earn and save in order to maintain. There were dozens, maybe even hundreds of these little things that I uncovered when I looked for them. These all add up, allowing me to redirect that money to my investments, which then had more time to appreciate in value. This is how it's done.

\section{Voluntary simplicity}
Frugality is obviously not a new idea. The miserly form of frugality has a long history, which is why people think frugality means being stingy, and why I had to spend so much time debunking this. After the wasteful 1980's, frugality started making a comeback. People find that it can actually lead to a life of abundance, rather than the miserly life of deprivation. There has been a growing movement in the last few decades of people who feel overwhelmed by our consumer culture and are seeking a new lifestyle and community of others who share their values. It's called \emph{Voluntary Simplicity.}

Duane Elgin, who wrote one of the first books on this, defines Voluntary Simplicity as ``a lifestyle that is outwardly simple but inwardly rich.''\cite{voluntary-simplicity} This inner richness involves more than just spending less money and having fewer things. It's about having an awareness of our values and life purpose, and making sure that all of our choices are in line with those.

There are many books and websites available on Voluntary Simplicity as well as discussion groups for people to find support. A community of like-minded people is very helpful for such a significant lifestyle shift.

\newpage
\section{Resources}
\begin{itemize}
\item \textbf{\emph{The Complete Tightwad Gazette} by Amy Dacyczyn.} This is a collection of every issue of \emph{The Tightwad Gazette,} a newsletter that promoted thrift as a lifestyle, and provided thousands of tips for saving money without sacrificing quality of life. Dacyczyn was a master at frugality, emphasizing DIY within the context of raising a large family. Some of her tips may seem out-dated or irrelevant to you. She constantly reminds the reader that the tips themselves aren't the point, but the attitude and thought process they demonstrate.

\item \textbf{\url{http://www.bankrate.com/}} -- An excellent tool for finding and comparing financial institutions. You can use this to find the best rates for internet checking, credit cards, mortgages, savings accounts, and more.

\item \textbf{\url{http://www.whylinuxisbetter.net/}} -- Explains why Linux is better than Windows, and how to switch to it.

\item \textbf{\url{http://www.jamendo.com/}} -- A music service for Creative Commons artists. Use this to find free, excellent music.

\item \textbf{\url{http://www.newdream.org/}} -- This is a very popular Voluntary Simplicity website.

\item \textbf{\url{http://www.moneychimp.com/calculator/compound\_interest\_calculator.htm}} -- The compound interest calculator I used in this book.
\end{itemize}
