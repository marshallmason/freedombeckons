\chapter{Preface}
\fancyhead[CO,CE]{\emph{Preface}}
I flunked nearly every class in school. I intentionally caused trouble for most of my teachers. I skipped classes every chance I had. I was suspended from school almost as often as I was in attendance. My parents and teachers assumed I was a ``bad kid'' who might shape up with enough discipline. They figured I wouldn't make it to graduation. Certainly no one dreamed I'd retire in my early 30's. Looking back, I feel like my whole life was leading up to it.

Many people ask me how I did it, and they seriously think I can give them a one-sentence answer. A lot went into this goal. It was a labor of love. It had just as much to do with passion and commitment as with investing and frugality. Motivation is everything for such a huge life dream as this, so before I talk about \emph{how} I did it, I want to explain \emph{why} I did it.

Defiance wasn't in my nature. I was actually very cooperative as a child. I instinctively wanted to do the right thing. I had a lot of energy and curiosity. I would latch onto adults and ask them an endless stream of questions. I had an intuitive sense of ethics and fairness. One of my earliest memories is a feeling of terror when I discovered I'd forgotten to pay for a five cent piece of bubble gum at the store. I walked all the way back to apologize and pay for it.

I've always remembered my preschool teacher as one of the kindest people in the world. She radiated love and embodied compassion. I felt safe and happy with her. What I remember most distinctly was how overjoyed she was. She absolutely adored kids. This is the kind of teacher kids should be accustomed to, but I'm grateful to have had this one.

My parents were fairly invested in the family. They didn't spoil us, but we had lavish Christmases. We went on interesting summer vacations every year. We ate dinner as a family every night. They were usually very warm and caring. I was a cute kid with a lot of energy, and I often felt adored by everyone.

But my life turned very ugly for me, early in my childhood. Things got very bad, and then they got worse.

My father seemed to have two different personalities---one that loved me and one that hated me. Whenever I was being punished, I felt hated. Because of my cooperative nature, I've always found punishment very confusing. It never seemed to fit the crime. I usually didn't even understand what crimes I'd committed, like the time he grounded me for not making my bed, until he realized he'd never actually taught me how. Here's a pattern that repeated in my home: my brother would hit me, and I'd cry. Then my father would come and hit me for crying. He'd send me to my room for the rest of the day, where I would continue sobbing for hours, and be ignored.

This happened often, but one time in particular has always stuck with me. It was earlier in the day, so I was grounded to my room longer than usual. I cried endlessly, all day, and well into the night. My stomach was growling, and I could hear my family sitting down for dinner without me. I wailed even louder, and started yelling that I was hungry. I was still crying when I heard dishes being put away. I told myself I had to keep crying, because if I stopped, they might forget I was there---my pain would become invisible. But it didn't work. The lights were turned off, and eventually all I could hear was the television.

Then I realized something tragic: \emph{nobody cares.} I looked around the room, taking stock of how I felt at that moment. I took a snapshot in my mind. I thought about how lonely and angry I was. I promised myself that I would never forget that moment, or what it was like to be a kid, to feel so trapped and helpless, to be so powerless over my own life. I promised myself that, once I was older, I would never treat kids the way I was treated, and would get to the bottom of this injustice. Of course, I couldn't have articulated it like that at the time, but that was the essence of the promise. For the rest of my life, I never forgot that night, and over time, the promise I made to myself blossomed into an obsession with freedom.

School was no better for me. Ms. W was a monster masquerading as a fifth grade teacher. She would yell at me, I would cry, and then she'd call me a cry baby. She often told me I was stupid, as if reminding me just in case I'd forgotten. When I was sniffling a lot from allergic reactions, she said I was gross. I tried to stop sniffling, but I didn't know how. I figured it was because I was stupid. What I remember most vividly was when she brought me into the bathroom where no one else could hear her, and yelled at me as I cowered in the stall and cried. I just sobbed uncontrollably while she yelled at me to stop crying. ``Grow up! Only babies cry!'' she would yell. I tried to tell my mother about this, but she didn't believe me. The year I spent with this teacher changed me forever and completely. My attitude toward authority figures was transformed by this experience.

My sixth grade teacher found me extremely difficult to work with, and she was visibly frustrated. It seemed impossible to motivate me. She punished me a lot, sometimes she pleaded, and sometimes she tried pep talks. She told me I had a lot of potential, if only I would apply myself. When I started junior high, I made a sincere attempt to be a good kid and a good student, but met with limited success.

I tried hardest in Mr. G's class, which had a strict and detailed system for measuring and boasting success. From the very first day, the kids in the front were obviously favored, and seemed to breeze through the class with lots of attention from the teacher. The smart kids seemed like an exclusive, elite club, a higher caste that I was simply not born into. I sat in the back with the other misfits. I was obviously one of the stupid kids. No matter how hard I tried, I couldn't break out of my caste.

My parents had become increasingly distant around this time. They separated, and nearly divorced, but changed their minds after many trying months. They stayed together after this, but everything changed. They were no longer so invested in the family. No more family dinners. No more summer vacations. No more weekend outings with my father. We moved across the country, and they started a business together. I didn't see them much anymore, and their influence in my life mostly disappeared at this point. They started to feel like strangers whose task in life was to punish me. I thought of them more as prison wardens than parents.

All of my efforts in school felt wasted. There was so little incentive to work hard. I didn't get anything for it, just a letter grade on a piece of paper, which was meaningless to me. Adults seemed to think it was important, so I figured I could use that grade to finally earn the respect of my parents and teachers, but that wasn't forthcoming, despite my best efforts. Besides, I resented that I had to earn their approval by doing these tricks for them. I simply gave up trying.

This was a point in history when the schools had become preoccupied with self-esteem, a contentious obsession that not everyone agreed with. All students were required to be evaluated by a psychologist. The man I met with asked me to give him an analogy of how I felt about myself. I told him I felt like a piece of trash in a trash can that someone hits with their car. He found this disturbing, and decided to schedule an appointment with an on-campus therapist. When I met with her, I felt very safe. I told her everything about fifth grade. She listened to me and believed me. It was such a relief to find an adult I could trust, someone I could talk to whenever I felt lost or scared.

A couple days later, I was called out of class to the principal's office. I had no idea what provoked this, so I was fairly alarmed. The principal sat me down and closed the door. He alternately yelled and hissed at me about how I was privileged and whiny, that I was taking advantage of the system. I had no idea what he was talking about. I just sat there shaking as he laid into me. This lasted for the remainder of the class period. When the bell rang and I was excused, I went directly to the therapist's office, still shaking from the verbal assault. I started telling her what had happened, but she stopped me short. She said that I couldn't meet with her anymore, and she suggested I just try to move on. I felt incredibly betrayed, and more alone than ever.

The lesson from school that really stuck with me was that there are smart kids and there are stupid kids, and no matter how hard I tried, I would always be stupid. Learning seemed to be something that only happens in school, a dreadful, boring process to be avoided at all costs. I also learned that adults are not to be trusted, and they will usually betray me, even when they seemed like my friend.

School felt like a jail, which seemed strange in a country that boasts about freedom and democracy. There was a pecking order of good kids and bad kids, with systems in place for disciplining those who got out of line. I was confined and legally forced to obey rules made by arbitrary authorities, joylessly doing work with no incentive. I was punished for my disobedience. I had very little say in my own well-being and destiny. I was legally forced to attend this institution, and could literally be jailed for refusing. I needed to ask for permission to do even the most basic and private of functions. I was under constant surveillance. A strict schedule was facilitated by bells ringing throughout the day to shuffle me off to the next authority figure, who did as they pleased.

I believed, but could not yet articulate, that no one, anywhere, should stand for such a totalitarian regime. I couldn't fathom adults tolerating this kind of oppression, but they seemed to believe this was the appropriate setting for young people, as if it were some kind of boot camp for a lifetime of compliance and obedience as adults. I didn't see what my youth had to do with it. It seemed very much like age discrimination, and defiance was my only reasonable and ethical reaction.

By high school, I became more skilled and intentional in my defiance. I didn't just give up trying---my goal was to be the anti-student. The teachers wanted cooperative kids who studied and got good grades, so I did the exact opposite. I tried to make their lives as miserable as I could and get the worst possible grades. That way, they would have no power over me, and I made sure they knew that.

One teacher eventually got to the point that she could just look at me when I came in and see that I was ready for battle. She could read it on my face, and would make a preemptive strike, sending me to the principal's office before I even did or said anything. One day she asked if I intended to do this every day, and I said that I absolutely did, so we agreed that I'd just come in each day to pick up my referral slip.

Another teacher had a point system for determining grades, and misbehaving had the effect of deducting points, which I found very exciting. Here was a whole new tool in my quest for the worst possible grades. Before this, the worst I could get was a zero, but this was the first time I could get a negative grade! Unfortunately, he caught onto my game, and refused to deduct my points, no matter how poorly I behaved, so I never reached this crowning achievement.

My parents continued trying to punish me for my increasingly defiant behavior. They sought more creative punishments, and were distraught by how I just seemed to get worse the more they punished me. One day, they tried taking away my guitar, my only solace, to which I responded by not speaking to them aside from the occasional ``fuck you.'' After weeks of this, they took me to a family therapist, to whom I would continue to say my usual epithet. Finally, the therapist seemed to give up, and just started talking about himself, like the problems he was having with his computer. This was something I loved talking about, so I started to open up, and we had a great conversation. When my parents came to pick me up, he told them, ``your son is fine. He just needs his guitar back.''\footnote{I discovered this recently, which explains my entire adolescence: ``Confused and bewildered parents mistakenly hope that punishment will \emph{eventually} bring results, without realizing that they are actually getting nowhere with their methods. The use of punishment only helps the child to develop a greater power of resistance and defiance.'' - Rudolf Dreikurs, M.D.}

I did need my guitar back, but I wasn't fine. I was becoming extremely depressed. I felt desperately lonely, alienated, and unloved. My life was turning into a failure. I thought I was stupid and incapable of doing anything but making life hard for people. I had few friends, and my peers were cruel in their bullying and teasing. The only option that made any sense was suicide. I figured, maybe then people would take me seriously. They'd regret how they treated me. My life would be over, I would no longer be suffering, and everyone would blame themselves for it. Win-win.

I didn't go through with it because I knew there was one thing that I still loved in life: music. If there was one thing, then it was possible there would be more in the future. I wondered if I could hang onto music as my life preserver to get me through in the meantime, and I knew that I could. Music became my salvation and my only reason for living.

I transferred to a continuation school, where the teachers were more accustomed to kids like me. The counselor at this school was none other than Mr. G, my teacher from junior high, who must have thought his quirky brand of tough love was a great service to this wretched band of misfits, gang bangers, pregnant teens, and drug addicts. I had fewer opportunities to fight with the teachers, partly because some of the teachers treated the kids with respect and decency, and partly because most of the learning at this school was self-directed.

We worked at our own pace, and graduated as soon as we completed our high school requirements, which could be earlier or later than our usual graduation dates. The harder we worked, the sooner we graduated. Or we could do nothing at all, and never graduate. It was up to each of us how much or little we wanted to do. We were rarely nagged, and there was no attempt to discipline us for not being ``on task.'' This came as a great relief to me. This system gave me a real incentive to work, and less incentive to irritate the teachers.

However, I still wasn't very motivated, and was frequently suspended from school for increasingly controversial reasons. One time, the police were even called in, but I think that was just theater. My parents seemed to give up on me by this point, and only complained about the constant nuisance of being called at work to tell them that I was suspended yet again. My mother said she dreaded picking up the phone because the school was always calling her with more bad news.

Then I met Mrs. Adams.

She didn't seem very responsive to my misbehavior. She didn't get upset with me, or try to punish me. So I just goofed off and waited for the class to be over. When she gave me a math test, I discovered she'd given me the teacher's edition. My joyous defiance was rekindled once more as I copied all the answers from the back of the book and handed it in.

She thanked me and looked it over. I was quite proud of myself for my little stunt, and stood there confidently, knowing every answer would be correct. She asked if I could show her how I did one of the math problems. I looked at it and said, ``um, I forget.'' I knew I was busted. I was prepared to head to the principal's office for yet another suspension.

She said, ``okay, well, let me show you how I do it,'' and asked me to take a seat. Reluctantly, I sat down and watched as she worked out the math problem. Then she started on the next one, explaining what she was doing as she went along. I sat there uncomfortably, trying to figure out what she wanted from me, whether I was on my way to see the principal, or if I could go back to my desk. But she just kept working on math problems, and didn't seem to be trying to do anything else, so I just sat there and watched.

After a while of this, I was surprised that I was actually understanding what she was doing, and got the urge to try it myself, just to see if I could. I still wanted to be defiant, and I certainly didn't want to \emph{learn} anything, but I couldn't help it. She was hogging all the fun. I asked if I could try one. ``Sure,'' she said, and watched as I worked on a math problem. It was a breeze, so I moved on to the next one. And the next. Eventually I just went off by myself to play with math.

I started turning in pages of math problems, taking tests, and getting more to work on. This was the most fun I'd ever had in school. I didn't like eating lunch with the other kids, so I dropped by Mrs. Adams' room and found her eating lunch by herself. I started coming into her class during lunch to work on math. When I finished one textbook, she'd pull out another one, and I started working on that. At this rate, it didn't take me long to complete my high school math requirements, which meant leaving Mrs. Adams' class. I told her I didn't want to leave, and she explained that I could also use my elective credits for math, which I did. 

I noticed the local community college was offering computer classes. I liked computers, so I wanted to take one of these classes, but it required that I take an entrance exam. I was dreading this. The college staff assured me it was just a formality and had no bearing on whether I could sign up for the class. The results of the test were mediocre, but the math score was much higher than I expected. I was sure there was some mistake. I was going to point out the bug in their test-taking software, but I decided to show the test results to Mrs. Adams first.

``There's no mistake,'' she said, nonchalantly.

I was sure she wasn't fully appreciating just how much higher that score was than what I was capable of. I tried to explain this to her, but she just reiterated that there was no mistake.

``But I'm stupid,'' I said, helplessly.

She looked me in the eyes and said, ``you're not stupid.''

This was something I hadn't even considered, but I shook it off. Of course I'm stupid. Everything and everyone in my life had confirmed that. She's just one person.

``What do you think we've been doing all this time?'' she asked. ``This is the level I've been teaching you. This is exactly the score you deserve.''

This wasn't just anyone, I realized. This was Mrs. Adams. She'd been working closely with me for almost a year, teaching me math. If anyone would know my math skills, it would be her. But I still didn't trust her. My trust in adults had vanished years ago. She was either mistaken, I figured, or doing one of those phony, ``you can do it!'' bullshit pep talks that adults were notorious for.

Still, I couldn't help but wonder. I couldn't let it go. After the computer class was over, I signed up for a math class at the community college. Mrs. Adams urged me to sign up for the class I was cleared for, but I was reluctant. She assured me that I would be bored if I took anything lower. I tried it, but after two class sessions, I panicked and switched to a lower-level class. Sure enough, I was bored. We were working on stuff I'd learned months ago. This was enough to make me seriously doubt my conviction that I was stupid.

Because of the extra work I did with Mrs. Adams, and the college classes I took on the side, I graduated from high school six months early, despite having failed every class in my first two years. I'd essentially completed four years of high school in a year and a half.

I decided to test this, to really test it, once and for all. I signed up for a full semester at the community college, and poured myself into my studies, determined to find out if I had what it takes. I still suspected I didn't, but at least this way I would definitely know. Otherwise, I knew I'd always wonder about it. I had a whole new attitude toward learning---I merely had to apply the same passion I felt with math to the other subjects. I loved every second of it, and was fascinated by what I was learning.

My parents, who were aware of none of this, were still convinced I was headed nowhere. One night I had a heated argument with my mom over something petty. The next morning, she told me to get out. My bags were already packed.

I couch surfed for a while, and this was right around the very first time I started dreaming of retiring early. It felt like a pipe dream, not something I really took seriously until the end of college, but I knew I wanted it.

One day my parents called and asked if I could meet them for dinner. I agreed, and when we sat down to eat, my father pulled out my first college report card.

Straight A's.

They said they had no idea. They apologized for kicking me out, and said they never would have if they'd known I was doing so well in school.

``No thanks to you,'' I told them.

They ignored this barb, and offered to pay for college. I really resented this. They were shamelessly admitting that their approval and support were contingent on my academic success. How dare they come back and believe in me now, after the time when I most desperately needed it had come and gone? It was like betting on the horse after the race. I didn't trust them, and I didn't want their money. By then, I knew that what parents give, parents take away, whenever they needed to flex their muscles.

But I agreed nonetheless. I really wanted to attend college, and I needed their money to do so. I figured, they were basically buying pride. They would pay for college, and in exchange, they get to feel like good parents. This was especially true for my father. College success was like a status symbol for him, a way to impress his colleagues and clients.

My mother still resented me. She called me soon after and asked if I would meet her for dinner, just the two of us. This turned out to be a long lecture about how bad a kid I was, how miserable I'd made her life, and how sorry she was that I was even born. I didn't say much in response. There was nothing to say. I knew she was right, and at least she was being honest. I had been nothing but a nuisance to her for years.

After dinner, she called out to me as I was walking away. I turned around, and she said, ``thanks for nothing.''

``You're welcome,'' I said, and meant it.

I continued to excel at the community college, getting mostly A's. I took a lot more classes than I needed, for other subjects I was passionate about, like psychology, music, and of course, math. I also became a math tutor. I wanted to help people with math the same way Mrs. Adams helped me.

However, I continued to suffer from depression. I lived in poverty conditions for the next few years. Tuition was paid for, but there were some weeks that I nearly starved. I took any job I could. These jobs were truly awful. The work was humiliating, the pay was below minimum wage, and the bosses were cruel. In one of these jobs, the owner had a habit of getting drunk and standing behind his employees as we worked, yelling at us about what losers we were. This had an eerie resemblance to fifth grade. The biggest difference was, this time, I could walk away. That job lasted all of a day, and I went without food for a week rather than be subjected to that humiliation. I felt like a runaway slave.

After three years at the community college, I transferred to a university. I had a very purist attitude about college. I figured I was there to learn, period. I wasn't there for grades, degrees, or job training. I loved learning, and that was my only reason for going to college. It was literally a matter of running out of math to play with in Mrs. Adam's class, and looking for my next fix. This attitude afforded me the freedom to ignore the politics of academia---the students grubbing for grades, and professors on power trips. I figured the grades and degree would follow naturally from my devotion to learning.

I discovered other subjects I loved, like writing, philosophy, and science. I toyed with the idea of double-majoring in computer science and physics, and I got a job as a physics research assistant. I knew I would run out of classes at some point, and then I'd probably go on to graduate school, so I could keep playing.

It didn't turn out that way, exactly because of this flawed philosophy. Although the grades did naturally follow from my passion for learning, the degree didn't. I did a lousy job at planning my college career. Half the classes I took did nothing toward a degree, and I wasn't making much progress toward my computer science major. It had already been seven years, and I was still chugging along. My hunger for knowledge was becoming satiated, but I still had a lot more to do to graduate.

Most of the classes I still needed were upper-division computer science classes, each of which were so challenging that they were designed to be taken one per semester. Even then, many students didn't pass and had to re-take them. To make up for lost time, I had to pack in \emph{four} of these classes per semester. 

The attitude among the teachers and students at this stage of my education was vastly different from what I was used to in my earlier years. It seemed to mock my purist beliefs about the joys of learning. The prevailing attitude was sink-or-swim---if you can't handle it, then get out. The teachers seemed to enjoy watching us squirm. I sought out other students who were passionate about learning like I was, but most of them were just trying to survive. The rest seemed to have a macho attitude, trying to one-up each other with their programming prowess. Sometimes I felt like I had inadvertently been conscripted into the military.

One teacher was so bad that, although I passed, I decided to re-take his class with a different teacher because I knew I'd learned nothing. He mostly talked about himself, his pet projects, and his highly-opinionated thoughts about the computer industry. He also wrote a macro language that transformed a standard programming language into something entirely his own creation, and then required all class projects be written in it. Before I could work on assignments, I had to learn his proprietary language. It usually didn't work, so I spent most of my time debugging his code before I could get my own code to work. One day in class, he pompously stated, ``I don't test my code. That's what my students are for.''

I had no choice but to persevere through this kind of thing if I intended to graduate. And what a shame it would be not to, after seven challenging years. All I had learned would mean nothing to employers without that degree. It would be such a terrible waste.

As I became increasingly burned out, my depression continued to get worse. It got so bad that just getting out of bed was a struggle, let alone making it to class and finishing my assignments. This would be devastating in a normal semester. Four upper-division computer science classes made it impossible. I knew I couldn't get the A's I was accustomed to, so I settled for C's. As the semester progressed, it became clear that even that wouldn't be possible.

I felt like I was drowning. The harder I fought for air, the deeper I sank. My passion for knowledge had dried up, and the depression that it was holding back burst forth like a tidal wave. I knew I'd never make it to graduation in this state. My mind was my own worst enemy.

Hoping to quell my mental anguish, I started digging up my past, and what I found there was not pretty. I was suppressing all the anger I'd built up over the years, and it was poisoning me. As the memories came flooding back, I started feeling like my life was one long nightmare. Once again, I realized, I felt trapped, enslaved by authorities and their expectations of me. My feeling of slavery never actually ended after high school. All I did was find new overlords. Rather than teachers and parents, now I had inhumane bosses and professors on power trips.

I started to realize that just because I have a philosophy of going to college to learn doesn't mean the college shares that philosophy. Learning was just a front. It was the sales pitch, lightly sprinkled with words like ``success'' and ``future.'' I discovered college isn't really about education, although education does happen there. It's really about money. It's a business, and the demand for their product comes almost entirely from the job market. Few students go to college to actually learn. I was a rare exception, and I knew the system wasn't set up for me. Their dirty little secret, which everyone seemed to know except me, was that learning isn't the point of college. The point is job training. Career development. Money.

Then it hit me: \emph{every ounce of suffering I experienced in my life came down to the almighty dollar.} Lack of money was the reason I had to subject myself to humiliation by bosses. It was the carrot or the stick my parents were always dangling in front of me. I had to admit that it was a big reason I wanted to graduate. Once I did finally graduate, it would be the reason I would hand over the majority of my waking life for someone else's agenda. It would be the reason I would likely be subject to further humiliation by future bosses, who feel entitled to their power over me because they know they're paying me.

I recalled the boss who got drunk and yelled at me as I worked. When I came back for my meager paycheck for one day of wages, which I desperately needed to buy food, he treated me to another dressing-down. He told me, ``with your attitude, you'll never get anywhere in life.'' In other words, \emph{your refusal to tolerate my humiliation will cause you to not make money.}

Money. That's the key to everything in this society. I had very little control over the money I earned because I was always at the whim of others. So I turned my attention to what I realized I actually did have quite a bit of control over: my \emph{relationship with money.} In a sense, I was enslaving myself with my own spending habits.

I knew I needed to transform my relationship with money. The prospect of working at a full-time job repulsed me, after barely surviving this academic onslaught to my psyche, but I didn't see much choice. The only alternative was homelessness, living in my car, maybe playing my guitar on street corners for change. This wouldn't have been that much different from how I'd been living ever since my parents kicked me out, so I figured I was already prepared for such a lifestyle. But this plan had too many flaws, aside from the obvious discomfort. What if I needed to see a doctor? What if I got into an accident? One way or another, I'd eventually be forced to work, likely with humiliating bosses, so I'd still feel like a slave.

The funny thing is, it wasn't actually working, per se, that I was dreading so much. I rather enjoyed working. What horrified me was the prospect of devoting the rest of my life to it, with no end in sight and little say over my own destiny, always at the whim of some arbitrary authority figure. I was happy to work, but not as a slave. It was freedom I was craving, not lethargy.

Out of these realizations, my depression became overshadowed by a raw anger and determination. I made a commitment to become financially independent, and finally achieve the freedom I'd craved all my life. My father, aside from being an authority figure in my life, was also a financial planner. His professional opinion was that it was impossible. When I told him I was going to do it anyway, he told me I was stupid to try. By this time, I'd had a lot of practice telling my parents where to stick their discouraging words.

I was still a student. I had no income. I knew almost nothing about investing. I had a student loan and two maxed-out credit cards, no professional job experience, a bad habit of buying junk I didn't need with money I didn't have, and a severe depression that nearly incapacitated me much of the time. I was off to a pretty rough start.

But I knew this was going to be a whole new beginning for me, an end to the life I once knew. This goal would transform me. It would become my new mission in life, my religion. It would simplify and cut away all that was pulling me down, and I would rise from my ashes like a phoenix. I still worried about how I could possibly live with all these memories haunting me, but I realized I just had to persevere long enough to start creating good memories. Once I was earning more money, I could move to a nicer area, and I could afford to get some help to make peace with my past.

Meanwhile, I had to forget my old life completely. I dropped everything, especially those memories that haunted me. I cleared away everything except my determination to graduate. I eliminated all expenses except the bare minimum. Every nerve in my body was telling me to just collapse, but my anger would pick me back up and animate me like a zombie. I mustered every ounce of energy I could and devoted it to classes, studying, and class projects. I didn't excel in those classes, but I did pass.

There was one company I wanted to work for, and in all my searching, I had no interest in working anywhere else. The plan was this: I would get hired at that company; I would forget my old life, and create a new one; I would retire as early as I possibly could. The hard deadline was 10 years---the absolute maximum I could fathom working. I was almost 25 years old, and I would graduate soon, so my deadline was my 35th birthday. This was not optional. After that, I promised myself that I would stop working, even if it meant living on the street. Guaranteeing that freedom will be in my future was the only way I could drag myself through all the hoops I had to jump through to get there.

I did it, exactly as I'd planned. I graduated, and then applied to work at my chosen company. They hired me. I moved to a beautiful town, dropped my old life, cut off ties from my family and most of the people I knew from my old town, and started creating new memories. I ended my old life in every way possible.

From the day I vowed to become financially independent, everything became aligned along this single-minded purpose. I learned how to channel my anger into the present moment, turning it toward the task at hand. What used to feel like an endless stream of rage about the past became a resource, fuel to power my goals in the present. I used up every drop of that anger until there was none left. Forgiveness was the natural byproduct of this transformed anger. It wasn't a willful act, but a natural process. Forgiveness came as naturally for me as waking up from a nightmare.

I took antidepressants and worked with a wonderful therapist, which helped lift me from my depression. I took a dozen personal growth workshops, read hundreds of self-help books, and began practicing meditation. I learned to watch my mind closely, and not let it pull me into such dark places.

Just as I used to thrive on hate, I now thrive on love. Just as I used to want to destroy the system, I'm now devoted to helping people and making the world a better place. I feel like I've gotten back to my true nature, the character traits that I manifested as a young child.

I retired on February 12, 2008, at age 32, less than eight years after I'd set my goal. Since those painful days, my life now is very different. It's filled with fascination and play. All of my time is devoted to things that matter to me. I always get plenty of sleep. My mornings are very structured. I go to the gym, meditate, play my guitar, and do various things on my to-do list. In the afternoons, I work on personal projects, such as writing, programming, music, and researching. Some afternoons, I do some volunteer work. The evenings are much less structured. I might go out with friends, play at an open mic, host a house concert, watch movies, or play computer games. It's a good life.

It has taken me a while to not be haunted by my past, and I still am occasionally, especially in my dreams. But most of my memories are bittersweet. I have the kind of distance from it that allows me to see everything that happened to me, not as painful and traumatic, but as trials that turned me into the person I am today. I don't regret any of my rebellious behavior, and I see now how necessary it was for me at the time. It was like a heavy armor I put on to protect myself in a hostile world. I'm proud of myself for the tremendous integrity I showed at such a young age. Sometimes, I feel a little resentful, and think about how my parents failed me, or how I failed them, but mostly I just feel gratitude.

Financial independence is a very personal journey. This goal is radically different from the norm, and it requires a lot of effort, so people don't earnestly enter into this journey without something very compelling from their lives that motivates them. Financial independence is, for me, the natural conclusion to the perennial struggles of my youth, the happy ending to my otherwise painful story. Freedom had been beckoning to me all my life.
