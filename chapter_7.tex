\chapter{Abolishing Slavery}
\fancyhead[CO]{\emph{Abolishing Slavery}}
\fancyhead[CE]{Freedom Beckons}

\section{An experiment}
I'm not opposed to working, but I am opposed to trading my labor for money. I have found that, insofar as I tie my livelihood to someone else's agenda, I give some of my freedom to their whims, and they feel entitled to make demands of me that they would never think to make otherwise. Usually, this treatment has been disrespectful at best, and inhumane at worst. Insofar as they control my livelihood, they control me. Insofar as they control me, I'm a slave---a \emph{wage slave.}

It may sound extreme to claim that working for a living is wage slavery. Rather than trying to persuade you, I simply want you to consider wage slavery a hypothesis that I would like you to test. Suspend disbelief for a moment and allow me to present some evidence. If you're unpersuaded, that's fine. I only want you to run the experiment and see what results you get.

The first part of the experiment is to consider how much of what I have just said rings true for you. Have you ever been treated disrespectfully by an employer, but you either couldn't leave or didn't want to leave because your livelihood or quality of life was linked to your capacity to endure this mistreatment? If you weren't being paid the amount you were earning for that work, would you have stayed?

It's possible to change jobs, but this is a very daunting proposition when your livelihood is on the line. If you can change jobs, you should count yourself lucky to be in a career or economic era in which unemployment is low enough that you can easily go elsewhere. For many people, this is not much of an option. They feel lucky to get whatever work they can get, and if they don't like how they're treated, well, that's just part of the job. Additionally, a different job often only means a new face for the same treatment. Even if what I have said does not ring true for you, it's important to remember that it very much does for many people, though to a lesser degree than in the past.

\section{History of slavery abolition}
Our civilizations have a long, sordid history of slavery. An enormous amount of labor throughout history was done by slaves, much of which we still benefit from today. These slaves were coerced with the threat of suffering and ultimately death for themselves and those they care about. They had no say over their work conditions, they could not choose their masters, and they could not quit unless they were willing to suffer or die. Some of them were willing to do just that.

Today, in America, slavery is illegal. This happened in 1865, which was practically yesterday compared to the total time that slavery was common. So, only very recently in history, the ethics of work have started to change. But what changed, exactly? What specific freedoms were won with the abolition of slavery?

At first, the only changes were the rights to earn and accumulate wealth in exchange for labor. Most workers still had little say over their work conditions, and wages were often so low that it was very difficult to accumulate wealth. Technically, they could quit, but they soon had to find another job, or face the consequences. So, actually, the only real, significant change was the ability to choose their masters. No matter which master they chose, the conditions weren't much different than they were before---particularly for the former slaves---and they couldn't really quit unless they were willing to suffer and die.

It didn't take long for the next major change in the ethics of work. This change was focused on wages and the ability for workers to have some say over their working conditions. They had to unionize to win these rights, quitting en masse and therefore risking suffering and ultimately death for themselves and those they care about. Without the ability to pit workers against one another, their masters were forced to cede some rights. Strength in numbers was the only power these workers had to demand those rights.

But there's still one freedom that we have not gained: the freedom to quit. Workers can quit, and so could the slaves in the past, but quitting has consequences, the same ones the slaves faced: suffering and ultimately death for themselves and those they care about. They can change jobs, but that only means they can choose their master, not that they're not slaves. This inability to quit is the final aspect of slavery that remains for us to abolish.

\section{Wage slavery}
In ancient Rome, Cicero said that ``whoever gives his labor for money sells himself and puts himself in the rank of slaves.'' In this view, trading ones labor for money is analogous to slavery, except that the person doing the buying and selling is the slave himself. As a wage slave, you have the right to sell yourself as property to a new owner, but that doesn't change the fact that you're still property.

Wage slavery is not a socialist concept. Karl Marx also drew an analogy between slavery and working for wages, but he was borrowing from the grassroots labor movements that had been around for centuries. Socialists still believed people should be treated as property. They just thought this property should be owned by the state rather than capitalists.

Wikipedia defines slavery as ``a form of forced labor in which people are considered to be the property of others. Slaves can be held against their will from the time of their capture, purchase or birth, and deprived of the right to leave, to refuse to work, or to receive compensation, such as wages.''\cite{wikipedia-slavery}

Let's break this down.

Wikipedia defines \emph{forced labor} as ``a generic or collective term for those work relations, especially in modern or early modern history, in which people are employed against their will by the threat of destitution, detention, violence (including death), or other extreme hardship to themselves, or to members of their families.''\cite{wikipedia-forced-labor}

Working is usually \emph{not a choice.} This is because, if people don't work, they or those they care about will suffer and ultimately die because they can't afford to feed or house themselves. Thus, they're effectively threatened by destitution, death, or other extreme hardship to themselves or to members of their families, as defined by \emph{forced labor.}

People who work are also the property of others. Employers demand ownership of a person's body during the hours they stipulate. Employees are contractually required to use their bodies in the ways the employer dictates, in exchange for wages. In most cases, the punishment for breach of this contract is to simply end the contract, but in some cases, the penalty can be more severe and can extend to all hours. For example, soldiers are the property of the government. Stealing that property is called AWOL or desertion. AWOL is punishable by prison and fine, and desertion is punishable by death. Actors and models sometimes have contractual stipulations in which their body parts are owned by their agencies. They can be sued for breach of contract if they damage that property. Many corporations have non-compete clauses, which stipulate that whatever employees produce, even off-hours, is owned by them. They might also stipulate where former employees may or may not look for work after they leave their employment.

It may be argued that these are contracts that are mutually agreed-upon and freely entered into. However, if their lives and the lives of their loved ones are being threatened if they refuse to enter the contract, then it is done under duress, and therefore not freely entered into.

Something employees get that slaves don't is compensation or wages. However, \emph{indentured servitude} is considered a form of slavery, even though it includes compensation and has a limited-term stipulated contractually. The reason is that indentured servants have very few choices or negotiating power with their employers, so for all intents and purposes, it may as well be slavery. Wages by themselves don't constitute the absence of slavery. What matters is how much choice they have.

\section{Freedom is about choices, not convenience}
Benjamin Franklin once said, ``they who can give up essential liberty to obtain a little temporary safety, deserve neither liberty nor safety.'' Here's an analogy I like to use to demonstrate this: which provides more freedom, a hotel with no amenities like a gym and room service, or a hotel with these amenities but has an armed guard out front to enforce a late night curfew? That you have no plans to leave your room late at night isn't the point. The point is, you don't have the \emph{choice} to leave, even if you wanted to. You could say the second hotel offers you the \emph{choice} to go to the gym, but that's not really what it offers. Both hotels offer that choice, but only the second hotel offers you the \emph{convenience} of having a gym on the premises.

It's also important that the choices be reasonable. If someone holds a gun to your head and says, ``your money or your life,'' they have given you a choice, and therefore, you could argue, freedom. But that's not really freedom because the choice is totally unreasonable. In fact, they've \emph{taken away} choices you had before they came along, choices that are your right as a human being. Likewise, slavery gives workers the choice to work or not, but that's not freedom because the consequences of that choice make it totally unreasonable.

So, what distinguishes a free human being from a slave is \emph{reasonable choice.} People think they're not enslaved simply because they have a choice of who to work for. But work they must, or else they will starve. That's not a reasonable choice. They must be able to choose to not work at all before they can be a truly free, autonomous individual. Once they have that choice, they are no longer a slave, even if they still choose to work. Only then is it certain that their work is entirely motivated from within, rather than from without, through coercion. That's the choice that financial independence offers.

Self-employment doesn't count either. It may afford you more choices, a larger array of slave owners in the form of clients, but the fundamental choice about whether or not to work is still not there without financial independence.

It may be argued that we must work to survive, whether in a society or in the wild, and so what I'm speaking of is just slavery to our own biology, not external coercion. In developed countries, thanks to time-saving technologies, we work fewer hours than those in third-world countries. People in Mali, for example, work nearly three times as much as those in America. The slavery I speak of isn't about the number hours we work, but for whose agenda they are spent.

It's not like we're given the tools we need to survive when we're born into the world, and then given a choice about how we want to live. We're forced by law to attend an educational institution against our will. After 12 years of this indoctrination, all we know how to do anymore is to go to an institution on time and do the work ordered by authority figures that we had no hand in choosing. Throughout history, 2.5-10.8\% of us aren't even afforded that privilege.\cite{multpl-unemployment} The many for whom this system fails are granted no other choice but homelessness.

Working to survive on our own, meeting our needs directly, although it lacks the \emph{convenience, comfort,} and \emph{security} we now enjoy, and certainly requires more time, does afford more \emph{freedom from coercion.} That doesn't mean we should go back to being hunter-gatherers, or that that lifestyle was better. I'm not even advocating that work be abolished. I like the system we have, and I don't want to give it up. I just think it can be better.

\section{Revolution}
When you're financially independent, work must still be done in order for you to survive, but you have the choice about whether it's you who does that work. Financial independence is a machine that does your work for you. I mean this literally, but indirectly. Financial independence means owning investments that give you regular payments. Some of these investments are likely in the form of corporate stocks, which are shares of these companies. You literally own these companies, which includes the machinery for manufacturing goods that generate cash flow. In the case of bonds, you own property that companies basically rent from you. You transition from being part of the labor class to being part of the ownership and management class in society. And you don't have to wait to ``strike it rich'' to make this transition. All you need is a surplus, which is created by spending less. You start buying your investments right away.

There is some irony in this. From a Marxist perspective, I am exploiting the very wage slavery that I detest in order to escape from it myself. But I'm not a Marxist or a socialist. Like Marx, I agitate for freedom for the working class, but not through violent revolution, overthrowing the managers, and taking over the means of production. These require theft and violence, neither of which I believe in, nor do I think they are all that effective, ultimately.

I'm also not an anarchist. I sympathize with some anarchist philosophies, but I don't see any historical evidence that people can rule themselves on a large scale, without any leadership whatsoever. I believe in the free market, and I believe we can use the power of the free market to gain more freedom for everyone.

I don't believe welfare to be an appropriate way to achieve freedom. Welfare can be helpful as a temporary safety net for wage slaves, but someone has to pay those welfare checks, most of whom are other wage slaves. It's not fair to them to abuse this system and mooch off their hard work just so some can feel free. This is irresponsible parasitism, not true freedom.

Libertarians argue that the true form of modern slavery is taxes, not work. Taxes are not voluntary, and ultimately enforced through coercion and violence. If you doubt this, try not paying your taxes. If the initial, gentler attempts to get you to pay fail, the government will use violence to coerce you. Libertarians argue for the \emph{non-aggression principle,} that all violence and threat of violence is a form of slavery.

Unless I find more persuasive evidence, I believe taxes are necessary for freedom. Ultimately, aggression is the only recourse any government with true power has to enforce its laws. Instead of abolishing taxes and all forms of government aggression, I think our goal should be to create a democracy that works, so that we as a people can decide together how we're taxed, and how those taxes are spent. If aggression is necessary to prevent even worse forms of tyranny, then the only possible way we can call ourselves free is if we as a society are in charge of how this is governed.

A political solution I like better is called \emph{basic income,} as a replacement for social security and welfare. Under such a system, every citizen, without exception, is guaranteed a certain level of income. Anything beyond that must be earned, and those earnings are taxed. Although this may seem to reward laziness, I think of it as rewarding frugality. But it also rewards industriousness with luxuries unavailable to those who choose only the basic income.

There's no shortage of revolutionary philosophies, all over the political spectrum, that advocate radical, top-down social upheaval. The form of revolution that I advocate is a \emph{grassroots economic revolution} like Henry David Thoreau argued for in his book \emph{Walden.} I agitate for each person, one by one, to buy their freedom, with their own dollars. Rather than steal the means of production, they would purchase it, paid for in full by the sweat of their own brow.

What would happen to the economy if this revolution were to succeed, and everyone were to become financially independent? I have no pretensions about the probability of this. It's not easy to achieve financial independence, and most people probably don't have the desire and dedication needed to accomplish it. But what if, hypothetically, they did?

Obviously, people have to work in order for the economy to function.\footnote{Although, as the book \emph{The End of Work} argues, the need for workers will drop as technology improves. Without a corresponding trend of financial independence, these displaced workers will join the ranks of the unemployed and homeless, which is dangerous for society.} I'm not arguing that people not work. I'm arguing that they earn the choice to work or not. That means that financially independent people must be given enough incentive to work, since they may no longer be counted on to do it out of desperation and coercion. The burden is on employers to entice people to work for them, not on everyone else to ensure employers have a sufficient workforce. Everyone has their price. If you sweeten the pot enough, people will be willing to work, even if they no longer have to. But they must be given a better reason than that they'll starve to death if they don't, which is the only reason they are given presently.

Thus, employees would have more bargaining power with employers, including higher wages, better benefits, and a more enjoyable workplace. Of course, this would in turn decrease the amount of profits the owners of these companies earn, including financially independent shareholders who depend on that income. Additionally, people would be a lot more frugal in this paradigm, causing the profits of these companies to drop. Both of these forces would mean less growth in dividends and stock prices. This would raise the bar for financial independence. Fewer people would be financially independent, thus shifting more negotiating power back to the employers.

So there is a tension between these two forces. Slowly and subtly, a new dynamic between employee and employer would be created, with a balance between these two competing forces maintained along the way. Over time, employees would gain better wages and better treatment. Ultimately, the companies would all be owned by the working class. And so we come full circle, back to socialism, where the workers own the means of production, but not through Marxist revolution, and not by forsaking the free market values of capitalism. Capitalism and socialism can be brought into harmony at last, by achieving the ideals of socialism through the free market of capitalism.

Maybe equating a lack of freedom with slavery is a stretch, but the word ``slavery'' is too limiting. We're not either slaves or not-slaves. Instead, there are ways that we all lack freedom of choice, and these are subtle forms of slavery to that which denies us that choice. As anarchist philosopher Bob Black says, ``if you like to control your own time, you distinguish employment from enslavement only in degree and duration.''

My ultimate vision is not even financial independence, but freedom. Financial independence is only one form of freedom, one piece of my larger quest for greater freedom.

\section{Forms of slavery and freedom}
I'm not saying that people who work for a living are no more free than Southern slaves or Egyptian pyramid workers or indentured servants. These are all \emph{forms} of slavery, some providing more freedom and comfort than others. I'm saying that working for a living is a \emph{form} of slavery, and some lines of work provide more freedom and comfort than others. Southern slaves were less free and therefore more enslaved than indentured servants, who are in turn more enslaved than migrant lettuce pickers, who are more enslaved than secretaries, who are more enslaved than corporate executives. So, there is a \emph{spectrum of slavery} that each of these jobs fall into, but they are all more or less forms of slavery. On the other hand, a financially independent lettuce picker, who picks lettuce for fun, is not a slave, no matter how uncomfortable the work is.

The central variable in the slavery equation is \emph{control.} That's really the motivation behind slavery. \emph{The more control employers have over their employees and customers, the more they profit, and the less they risk.} So there's also the client side to this equation. It's in the interests of companies to maintain as much control as possible over what they sell. If it were up to them, they'd be able to control everything we do with what we buy from them. For example, they'd love to prevent us from peeking inside their products to see how they work, to prevent us from creating them ourselves or selling them cheaper. They also wouldn't want us to tweak what we buy to meet our own needs, but instead pay them to do it for us. Intellectual property laws give them some rights to do this in order to incentivize innovation, but these rights are limited. It also hasn't been technologically feasible for them to enforce many of these desires for control, until now.

More and more of the products we buy these days are electronic. There has been a trend in the last couple of decades to build software into everything we buy, to interact with it in such a way that it will stop working if we try to tinker with it. Additionally, much of the actual software we buy isn't owned by us, but licensed, under extremely restrictive terms that most consumers don't know about. As technology and software become increasingly integrated into our daily lives, it becomes essential for democracy that we demand software freedom.

For example, voting machines were impossible to peek into or tinker with, so it was impossible for independent security experts to validate that these machines work as advertised, with no back doors or pre-programming to electronically ``stuff the ballot box.'' Indeed, many security vulnerabilities were found in these machines, and they were banned in many precincts.\cite{black-box-voting} Entire governments also found they can be locked out of the content they created---including essential government documents---if the software used to create it was no longer maintained by the companies they bought them from. This gives the software company sovereignty over that government.\cite{wikipedia-opendocument-massachusetts}

It's also in the best interests of companies to exploit cheap labor, wherever they can find it. In a global economy, they have no geographical barriers to where they can seek out cheap labor. As people become more affluent in one country, corporations would try to move jobs overseas, to exploit cheap labor there. Even slave labor, or labor in conditions similar to slavery, is not out of the question unless it is explicitly outlawed in those countries.

There are 29.8 million slaves in the world today, more than ever before in history.\cite{washington-post-slave-map} Slavery didn't start in America, and it didn't end with the Civil War. It has existed throughout history, and it continues to this day. Some of it still happens right here in America; it just went underground. There is still human trafficking happening right here among us.

We have three kinds of voices with which to make our demands. One is government. Another is shareholder advocacy, but that will be harder to accomplish without more successful, lower cost, socially responsible mutual funds. Another is our dollars. Everything we buy supports something, and it's up to us to make sure our purchases support what we believe in. Money is like the air supply of corporations, and people collectively control that---seriously threaten their money supply, and the corporations will listen to any demand.

There is a science fiction theme of aliens who come to Earth and deceive and conquer us so they can steal our resources and feed on humans. This theme touches on our greatest fears, what seems to us the greatest evil because the scale of slavery is extended to the entire human race by another species, rather than just pockets of slavery of some by others of the same race. What I find even more scary than this science fiction theme is that it's not fiction at all. I'm not talking about an alien conspiracy. I'm talking about animals. Humans enslave animals, and exploiting them for labor and experiments is the least of our transgressions. We are that alien species that deceives and conquers entire species, steals their resources, and feeds on them. Truly ending slavery must also include animal welfare. This is just compassion for all living things, not veganism. How this compassion manifests in your life is your own choice.

In a sense, we're also captives of our own civilization. Our civilization has served us well for thousands of years, in the sense that it has brought significant affluence to a small number of the more privileged classes. It's also been a disaster in some ways. Throughout history, for every class of affluence, there has been a much larger poverty class that resented that affluence, and sought to overthrow it. This historical pattern has not ended.

Other cracks have been appearing in our civilization, just in the last hundred years, in the form of environmental catastrophes. Just as our civilization is not in harmony with the well-being of many classes of humans and animals, it's also not in harmony with ecology. Environmental dangers remind us of our limitations, that we are just another species of animal, after all, subject to the same ecological laws as the rest of nature. Ultimately, our civilization must be in line with these laws, or our civilization is unsustainable. Sustainability is almost a buzzword now, so it can be easy to forget its meaning. To be unsustainable is to be \emph{doomed to collapse.} Insofar as we feel powerless to change our civilization, to create a new one that works better for humans as a species of animal in ecological harmony with nature, we are slaves to our civilization.

Another way to look at this is that we're treating nature as our slaves. We don't limit our enslavement of other species to just other animals. We treat the entire ecosystem as our property, forced to do work for our benefit only. We kill anything that tries to benefit from nature besides humans, except that which we find useful to us, or that doesn't get in our way. Nature is not our property, but the reverse. We belong to nature, we intimately depend on the health of the ecosystem of which we're a part, and we will only survive if we are not a threat to the ecosystem as a whole.

\section{Freeing the children}
Children tend to be the most likely victims of slavery because they're helpless and dependent on adults, and more likely to be ignorant of their rights. I am going to go into great detail about children's freedom because they constitute nearly a quarter of our population.\cite{census-quickfacts}

It's so easy for adults to enslave children that they often do it unconsciously. Grounding kids, for example, is a form of captivity. Corporal punishment for not doing the will of parents is a form of torture. Forcing children by law to attend school and learn what they're told to learn is also a form of slavery, since they are not compensated for this labor and don't have a choice.

We live in a country that prides itself on freedom and democracy. People are given power only by consent of the governed. Justice is served by a jury of our peers. And yet, we lock children in small encampments that are run like dictatorships. One person, called the principal, serves as judge, jury, and executioner. There is no due process of law. The accused have no rights to a fair trial. Children can be convicted on hearsay alone. There is no democratic process, in which the kids have the power to choose their laws and elect the officials who represent them. Then, after over a decade of this, they're suddenly released from captivity and expected to somehow be effective participants in democracy.

When people think about schooling, they usually frame it in terms of \emph{should we educate our young people?} I believe a better way to frame it is in terms of \emph{should we torture people because they're different from us?}

For me, schooling was a system of torture. I had to wake at a painfully early hour, forced into a small encampment against my will, and made to labor on someone else's behalf. When I resisted, I was locked in an even smaller room, called ``in school suspension.'' It had no windows and we were not allowed to see the sunlight or even use the bathroom except for certain times of day. While trapped in this tiny room all day, we were tasked with the tedious and useless job of writing out the school rules. If we completed the job, we were to start over from the beginning, much like the story of Sisyphus, who was punished by being forced to roll a boulder up a hill, only to watch it roll back down, for eternity. By the end of the day, my hand was in pain. I believe this torture of children is not only unethical, but should be unlawful.

If it sounds extreme to call this torture, consider whether this is only because we're accustomed to thinking of children as lesser beings, less worthy of respect. We derisively call behaviors common of them ``childish'' or ``immature.'' When someone is behaving poorly, we tell them to ``grow up.'' What if we treated certain classes of adults this way? Imagine, all disabled people, or all Koreans, were physically assaulted for ``misbehaving,'' and locked in small rooms for hours, for no other reason than that they belong to that particular class of people and that they behave in a way we don't like. Wouldn't \emph{that} be called torture?

Compulsory schooling is not about education. Kids learn very little in primary and secondary schools, especially considering the massive amount of their lives it consumes. It's considered important only in terms of its capacity to entice universities to accept them, where they are to get their actual educations. Young people do want a quality education, as evidenced by the long waiting lists at top universities. And yet, only months before, it's assumed these same young people must be forced by threat of legal action to get an education. This makes no sense if it were actually about providing a real education.

Also strange is the overwhelming problem of preventing kids from blowing up their schools. This isn't as much of a concern in universities\footnote{K-12 school shootings, such as Sandy Hook and Columbine, get far more media attention than colleges. In general, there are more incidents of school shootings in K-12 schools than on college campuses.\cite{wikipedia-school-shootings}}, where the students are not much older. People don't tend to become violent when they're free and happy. Rather than address the real problem---forcing people to go to a place they hate---adults tend to look at superficial things like clothing, music, and video games. There is no evidence of a causal relationship between these things and violence.

Compulsory schooling is better understood in terms of discipline and child care. While some of the mistreatment of children in schools raises the concerns of parents, for the most part it's considered good for building their character. The importance of discipline over education is evidenced everywhere in schools. Grades are considered important more as an indication of conforming to the demands of teachers than of education. Grades in primary and secondary schools are determined by their behavior in the classroom. Grades mean everything in school, but if learning was valued over discipline and conformity, then an F grade would be irrelevant, as long as they're learning.

Schools are mostly designed to benefit adults, not kids. School hours are always conveniently established such that adults can take their kids to school and still have time to get to work. The fact that young people require significantly more sleep than adults is not considered a priority. Schooling is also conveniently compulsory exactly until the age that parents are no longer legally required to support them, which indicates that it's really just subsidized child care. If not for schools, child care costs would be prohibitive. Parents might leave their kids at home by themselves, where they might get into trouble, or force them to labor on their behalf as was common in the past.

The tragedy is that learning is one of the most fun things in the world, and people learn so much more effectively when they're free and not being tortured. This is especially true for kids, who are like sponges. They learn so easily, especially when they're excited about it. Schools should tap into this, not suppress it. Schools should teach democracy---which is supposedly a virtue in this country---by allowing the kids to elect their teachers and staff. Schools should teach true justice by having a system of due process and a jury of their peers. There are schools that do exactly this, called Sudbury schools.\cite{sudbury-valley-school}

There is also a movement called ``unschooling.'' This is a radical form of home schooling in which the child is allowed to direct their own education. Both Sudbury schools and unschooling have a remarkable track record of success, although this is difficult to measure because there is no objective definition of success. Even college enrollment is a poor measure of success, but it's all we have. One study found that 83 percent of unschooled kids went to college,\cite{unschooling-psychology-today} versus 66 percent of high school kids.\cite{college-nytimes}

Although I'm very critical of compulsory schooling and the public school system, this doesn't mean that I think parents shouldn't send their kids to these schools. I don't have kids, and I recognize how easy it is to have idealistic, impractical opinions about something that doesn't affect me personally. Perhaps, after hearing all my criticisms, it might surprise you to know that, if I had kids, I would send them to a public school, at least at first. I know there are some good teachers that are more interested in the joy of learning than in keeping kids in line. I'd much rather such teachers spend time with my kids than not. The real difference is the role I would take as a parent.

For one thing, I would not punish them, and I certainly would not allow a school, which does not have a loving relationship with the child, to punish them. Instead, I would talk with them, and make sure they understand how their behaviors impact others. Most importantly, I would listen to them, and understand their needs and what motivated them to behave in certain ways. When people act out, especially kids, it's usually because they're trying to communicate something but don't know how, or they think nobody is listening. I suppose it's possible there can be some rare circumstances in which I would find it appropriate to punish them, but I most definitely would not physically assault them or confine them in a small space for long periods of time.

I would not tell my kids to get good grades and leave it at that. Grades are not a measure of a child's worth, intelligence, or abilities. It's more useful as a measure of interest and a sufficient understanding of the material being taught. I would not be worried about the effect of bad grades on their future because I know it has very little effect. Even if they flunk every class, as I did, I know there are options available to them, such as community college. Learning should be fun. It \emph{is} fun, if presented properly. That doesn't mean it's easy. Some of the most fun things in the world are also the most challenging. When a kid is having fun, they'll get good grades, painlessly.

It may sound like I would be a lenient parent, but that's not how I see it. I would believe in my kids, and when you believe in someone, you expect a lot from them. I would also be a loving parent, and when you love someone, you set them free. But I would do so in stages. Too much freedom, too early, before a child has developed the capacity to understand their choices and the consequences of those choices, can be detrimental. Freeing them at the appropriate stages would be my job as a parent. Initially, this starts with making choices for them, but ultimately, my real job is to free them.

That they might make the wrong choices is a risk every parent faces, and must face. It's impossible to learn how to make their own choices, or to understand the consequences of those choices, if they're never given the chance to practice. This means giving them the opportunity to make bad choices and to face the consequences. This hurts many parents because they think their only role is to protect their children, not to free them.

I understand that parenting is hard work, which is a big reason I chose not to do it. It's difficult for parents to always live up to their ideals, and I doubt I would do it perfectly. But I would do my best, and I would ask for my kids' forgiveness for my mistakes, for I would know that it was my choice to be their parents, not their choice to be my kids.

\section{Conclusion}
I started this book, and this chapter, with financial independence, and I ended it with talk of animal liberation, environmentalism, education, and parenting. All too often, I find that these subjects degenerate into moralizing, and that is not my goal here. I only want to help you explore freedom and all of its possible meanings. These are political ideas, and people disagree fiercely over such issues.

I am very critical of the ways that we are still not living in a free society, hoping that it will give you a new perspective. Above all, I believe the most important political freedom is the freedom to disagree openly. You may have disagreed with some or all of what I've said on this subject. If so, I thank you for considering them, and I encourage you to find your own way.

Earning our own freedom by achieving financial independence is only the beginning. There are billions of people---both adults and children---animals, even our entire civilization and its relationship with the ecosystem, that need more freedom, on multiple levels. Financial independence is only one step, like opening your own prison cell door. There are billions of other cell doors that need opening, and who better to help the captives than one who is already emancipated?

Freedom is a spectrum. There are many forms of freedom, and we've made tremendous strides in providing more freedom than ever before in history. We've done this without resorting to anarchy, which can be just as oppressive as slavery. In a state of anarchy, there is no way to enforce laws against slavery and require citizens to abstain from oppressing one another.

But there is so much more freedom we can create, in so many forms. I believe our descendants will look back and say that our era was only the beginning. We're just scratching the surface.

Freedom beckons.

\newpage
\section{Resources}
\begin{itemize}
\item \textbf{\emph{Ishmael} by Daniel Quinn.} This is a perspective on captivity and freedom like none I've ever seen before or since. This book changed my thinking more than any book I've ever read.

\item \textbf{\emph{Walden} by Henry David Thoreau.} This is a classic book by one of America's greatest minds and writers, describing his two year experiment in simple living. He built a cabin in the woods, next to Walden Pond, so he could face the essentials of life and gain perspective on society.

\item \textbf{\emph{Free at Last} by Daniel Greenberg.} Gives an overview of Sudbury Valley School, which is governed by a participatory democracy, and believes in the birth right of children to their freedom, and the innate ability of children to learn.
	
\item \textbf{\emph{How to Talk So Kids Will Listen \& Listen So Kids Will Talk} by Adele Faber \& Elaine Mazlish.} Teaches parents and teachers how to communicate with their children without ever resorting to punishment.

\item \textbf{\url{http://www.gnu.org/philosophy/shouldbefree.html}} -- Makes the case for software freedom.
\end{itemize}
