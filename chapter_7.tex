\chapter{Abolishing Slavery}

\section{An experiment}
As I've stated before, I'm not opposed to working, but I am opposed to trading my labor for money. I have found that, insofar as I tie my livelihood to someone else's agenda, I give some of my freedom to their whims, and they feel entitled to make demands of me that they would never think to make otherwise. Insofar as they control my livelihood, they control me. Insofar as they control me, I'm a slave---a \emph{wage slave.}

It may sound extreme to claim that most work is wage slavery. Rather than trying to persuade you, I simply want you to consider wage slavery a hypothesis that I would like you to test. Suspend disbelief for a moment and allow me to present some evidence. If you're unpersuaded, that's fine. I only want you to run the experiment and see what results you get.

\section{History of slavery abolition}
Our civilizations have a long, sordid history of slavery. An enormous amount of labor throughout history was done by slaves, much of which we still benefit from today. These slaves were coerced with the threat of suffering or ultimately death for themselves or those they care about. They had no say over their work conditions, they could not choose their masters, and they could not quit unless they were willing to suffer or die. Some of them were.

Today, in America, slavery is illegal. This happened in 1865, which was practically yesterday compared to the total time slavery was common. So, very recently, the ethics of work have started to change. But what changed, exactly? What specific freedoms were won with the abolition of slavery?

At first, the only changes were the rights to earn and accumulate resources in exchange for labor. Workers still had little say over their work conditions, and wages were often so low that it was very difficult to accumulate resources. Technically, they could quit, but they soon had to find another job, or face the consequences. Some of them did just that, which we call the homeless. So, actually, the only real, significant change was the ability to choose masters. No matter which master you chose, the conditions weren't much different than they were before (particularly for the former slaves), and you couldn't really quit unless you were willing to suffer and die.

It didn't take long for the next major change in the ethics of work. This change was focused on wages and the ability for workers to have some say over their work conditions. They had to unionize to win these rights, quitting en masse and therefore risking suffering or ultimately death for themselves or those they care about. Without the ability to pit workers against one another, their masters were forced to concede some rights. Strength in numbers was the only power these workers had to demand those rights.

But there's still one freedom left: the freedom to quit. Workers can quit, and so could the slaves, but quitting has consequences, the same ones the slaves faced: suffering and ultimately death for themselves or those they care about. They can change jobs, but that only means they can choose their master, not that they're not slaves. This inability to quit is the final aspect of slavery that remains for us to abolish.

\section{Wage slavery}
In ancient Rome, Cicero said that, ``whoever gives his labor for money sells himself and puts himself in the rank of slaves.''\cite{cicero} In this view, trading one's labor for money is analogous to slavery, except that the person doing the buying and selling is the slave himself. As a wage slave, you have the right to sell yourself as property to a new owner, but you're still property. Henry David Thoreau even went so far as to argue that wage slavery is worse than slavery. In \emph{Walden,} he wrote, ``it is hard to have Southern overseer; it is worse to have a Northern one; but worst of all when you are the slave-driver of yourself.''\cite{walden}

Wage slavery is not a socialist concept. Karl Marx also drew an analogy between slavery and working for wages\cite{marx}, but he was borrowing from the grassroots labor movements that had been around for centuries. Socialists still believe people should be treated as property. They just think this property should be owned by the state rather than businesses.

Wikipedia defines slavery as ``a form of forced labor in which people are considered to be the property of others. Slaves can be held against their will from the time of their capture, purchase or birth, and deprived of the right to leave, to refuse to work, or to receive compensation, such as wages.''\cite{wikipedia-slavery}

Let's break this down.

Wikipedia defines \emph{forced labor} as ``a generic or collective term for those work relations, especially in modern or early modern history, in which people are employed against their will by the threat of destitution, detention, violence (including death), or other extreme hardship to themselves, or to members of their families.''\cite{wikipedia-forced-labor} Working is usually \emph{not a choice,} although people may think it is. The reason it's not a choice is because, if people don't work, they or those they care about will suffer or ultimately die, because they can't afford feed or house themselves. Thus, they're effectively threatened by destitution, death, or other extreme hardship to themselves, or to members of their families, as defined by \emph{forced labor.}

People who work are also the property of others, although people may think they're not. Employers may contractually demand ownership of a person's body, during the hours they stipulate, which may include all hours. For example, soldiers are the property of the government. Stealing that property is called AWOL or desertion. AWOL is punishable by prison and fine, and desertion is punishable by death. Actors and models sometimes have contractual stipulations in which their body parts are owned by their agencies. They can be sued for breach of contract if they damage that property. Many corporations have non-compete clauses, which stipulate that whatever you produce, even off-hours, is owned by them. They also stipulate where you may or may not look for work after you leave their employment.

Something employees get that slaves don't is compensation or wages. However, many consider \emph{indentured servitude} a form of slavery, even though it includes compensation and has a limited term stipulated contractually. The reason is that indentured servants have so few choices or negotiating power with their employers, so for all intents and purposes, it may as well be slavery. So, wages by themselves don't constitute the absence of slavery. What matters is how much choice they have.

\section{Freedom is about choices, not convenience}
It's important to distinguish between \emph{freedom} and \emph{convenience, comfort,} and \emph{security.} Benjamin Franklin once said, ``they who can give up essential liberty to obtain a little temporary safety, deserve neither liberty nor safety.'' Here's an analogy I like to use to demonstrate this: which provides more freedom, a hotel with no amenities like a gym and room service, or a hotel with these amenities but has an armed guard out front to enforce a late night curfew? That you have no plans to leave your room late at night isn't the point. The point is, you don't have the \emph{choice} to leave, even if you wanted to. You could say the second hotel offers you the \emph{choice} to go to the gym, but that's not really what it offers. Both hotels offer that choice, but only the second hotel offers you the \emph{convenience} of having a gym on the premises.

It's also important that the choices be reasonable. If someone holds a gun to your head and says, ``your money or your life,'' they have given you a choice, and therefore, you could argue, freedom. But that's not really freedom, because the choice is totally unreasonable. In fact, they've taken away choices you had before they came along, choices which are your right as a human being. Likewise, slavery gives workers the choice to work or not, but that's not freedom because the consequences of that choice make it totally unreasonable.

So, what distinguishes a free human being from a slave is \emph{reasonable choice.} People think they're not enslaved simply because they have a choice about who to work for. But work they must, or else they will starve. That's not a reasonable choice. They must be able to choose to not work at all before they can be a truly free, autonomous individual. Once they have that choice, they are no longer a slave, even if they still choose to work. Only then is it certain that their work is entirely motivated from within, rather than from without, through coercion. That's the choice that financial independence offers.

Self-employment doesn't count either. It may afford you more choices, a larger array of slave owners in the form of clients, but the fundamental choice about whether or not to work is still not there without financial independence.

It may be argued that we must work to survive, and so what I'm speaking of is just slavery to our own biology, not external coercion. In developed countries, thanks to time-saving technologies, we work fewer hours than those in third-world countries. People in Mali, for example, work nearly three times as much as those in America. The slavery I speak of isn't about the number hours we work, but for whose agenda they are spent, and how much choice we have in the matter. The advantages we enjoy in developed countries are an outgrowth of a trade-based economy, but what we gain in convenience, comfort, and security, we lose in freedom.

It's not like we're given the tools we need to survive when we're born into the world, and then given a choice about how we want to live. Instead, we're forced by law (which ultimately derives its power from the threat of violence) to attend an educational institution, against our will, that we don't get to choose, for nearly one-quarter of our lives. This would be bad enough if this schooling gave us the tools we need to survive on our own, but it doesn't. It doesn't even give us the tools we need to survive in our culture. The career opportunities open to a high school graduate are scarcely higher than for drop-outs. This is why high school teachers tell their students they'll be ``flipping burgers'' if they don't go to college. College itself is prohibitively expensive for many people, and doesn't even accept everyone who applies. All of this is for the privilege of spending half our waking lives working at jobs designed to further someone else's agenda, often suffering humiliations that we would never stand for if it were completely voluntary. Often throughout history, 10\% or more of us aren't even afforded that privilege. The many for whom this system fails are granted no other choice but homelessness.

Working to survive, meeting our needs directly, although it lacks the \emph{convenience, comfort,} and \emph{security} we now enjoy, does afford more \emph{freedom.} That doesn't mean we should go back to being hunter-gatherers, or that that lifestyle was better. That's a false dichotomy. I'm not even advocating that work be abolished. I'm only agitating for more freedom.

% TODO: Need to fill this out more
A good model to describe what I'm advocating is Maslow's hierarchy of needs. This model is in the shape of a pyramid. At the bottom is our basic needs: food, shelter. Once these are met, we seek to meet

\section{Revolution}
When you're financially independent, work must still be done in order for you to survive, but you have the choice about whether it's you who does that work. Financial independence is a machine that does your work for you. I mean this literally, but indirectly. Financial independence means owning investments that give you regular payments. Some of these investments are probably in the form of corporate stocks, which are shares of these companies. You literally own these companies, which includes the machinery for manufacturing goods that generate cash flow. In the case of bonds, you own property that companies basically rent from you. You transition from being part of the labor class to being part of the ownership and management class in society. And you don't have to wait to ``strike it rich'' to make this transition. All you need is a surplus, which is created by spending less. You start buying your investments right away.

There is some irony in this. From a Marxist perspective, I am exploiting the very wage slavery that I detest in order to escape from it myself. But I'm not a Marxist or a socialist. Like Marx, I agitate for the freedom for the working class, but not through violent revolution, overthrowing the managers, and taking over the means of production. That involves theft and violence, neither of which I believe in, nor do I think they are all that effective, ultimately.

I'm not a socialist or an anarchist. I sympathize with some anarchist philosophy, but I don't see any historical evidence that people can rule themselves on a large scale, without any leadership whatsoever. I believe in the free market, and I believe we can use the power of the free market to gain more freedom for everyone. I don't believe welfare to be an appropriate way to achieve freedom. Welfare can be helpful as a temporary safety net for wage slaves, but someone has to pay those welfare checks, most of whom are other wage slaves. It's not fair to them to abuse this system and mooch off their hard work just so some can feel free. This is irresponsible parasitism, not true freedom.

Libertarians argue that the true form of modern slavery is taxes, not work. Taxes are not voluntary, and ultimately enforced through coercion and violence. If you doubt this, try not paying your taxes. If the initial, gentler attempts to get you to pay fail, the government will resort to violence. Libertarians argue for the \emph{non-aggression principle,} that all violence and threat of violence is a form of slavery.

Unless I find more persuasive evidence, I believe taxes are necessary for freedom. Ultimately, aggression is the only recourse any government with true power has to enforce its laws. Instead of abolishing taxes and all forms of government aggression, I think our goal should be to have a democracy that works, so that we as a people can together decide how we're taxed, and how those taxes are spent. If aggression is necessary to prevent even worse forms of tyranny, then the only possible way we can call ourselves free is if we as a society are in charge of how this is governed.

A political solution I like better is called \emph{basic income,} as a replacement for social security and welfare. Under such a system, every citizen, without exception, is guaranteed a certain level of income. Anything beyond that must be earned, and those earnings are taxed. This is one of the few things that both socialists and free market libertarians agree on. The ones opposed to it is the mainstream Republican-Democratic party, who have more stake in the status quo.

There's no shortage of revolutionary philosophies, all over the political spectrum, that advocate radical, top-down social upheaval. The form of revolution that I advocate is a \emph{grassroots economic revolution} like Henry David Thoreau argued for in his book \emph{Walden.} I agitate for each person, one by one, to buy their freedom, with their own dollars. Rather than steal the means of production, they would purchase it, paid for in full by the sweat of their own brow.

What would happen to the economy if this revolution were to succeed, and everyone were to become financially independent? I have no pretensions about the probability of this. It's not easy to achieve financial independence, and most people probably don't have the dedication needed to accomplish it. But what if, hypothetically, they did? Economics is tricky, with subtle ripple effects that are hard to understand, and often hard to predict. I'm not an economist, but I'll try to speculate.

Obviously, people have to work in order for the economy to function. I'm not arguing that people not work. I'm arguing that they earn the choice to work or not. That means that financially independent people must be given enough incentive to work, since they may no longer be counted on to do it out of desperation and coercion. It's up to employers to incentivize people to work for them; it's not up to people to ensure employers have a sufficient workforce. Everyone has their price. If you sweeten the pot enough, people will be willing to work, even if they no longer have to.

Thus, employees would have more bargaining power with employers, including higher wages, better benefits, and more enjoyment. Of course, this would in turn decrease the amount of profits the owners of these companies get, including financially independent shareholders who depend on that income. Additionally, people would be a lot more frugal in this paradigm, causing the profits of these companies to drop. Both of these forces would mean less growth in dividends and stock prices. This means the bar for financial independence would be raised, and less people would be financially independent, thereby shifting more negotiating power back to the employers.

So there is a tension between these two forces. Slowly and subtly, a new dynamic between employee and employer would be created, with a balance between these two competing forces maintained along the way. Over time, employees would gain better wages and better treatment. Ultimately, the companies would all be owned by the working class. And so we come full circle, back to socialism, where the workers own the means of production, but not through Marxist revolution, and not by forsaking the free market values of capitalism. Capitalism and socialism can be brought into harmony at last, by achieving the ideals of socialism through the free market of capitalism.

Equating a lack of freedom with slavery may seem to be a bit of a stretch. But the word ``slavery'' is too limiting. We're not either slaves or not-slaves. Instead, there are subtle ways that we all lack freedom of choice, and these are subtle forms of slavery to that which denies us that choice. As anarchist philosopher Bob Black says, ``if you like to control your own time, you distinguish employment from enslavement only in degree and duration.''

My ultimate vision is not even financial independence, but freedom. Financial independence is only one form of freedom, one piece of my larger quest to bring more freedom to the world.

\section{Forms of slavery and freedom}
I'm not saying that people who work for a living are no more free than Southern slaves or Egyptian pyramid workers or indentured servants. These are all \emph{forms} of slavery, some providing more freedom and comfort than others. I'm saying that working for a living is a \emph{form} of slavery, and some lines of work provide more freedom and comfort than others. Southern slaves were less free and therefore more enslaved that indentured servants, who are in turn more enslaved than migrant lettuce pickers, who are more enslaved than secretaries, who are more enslaved than corporate executives. So, there is a \emph{spectrum of slavery} that each of these jobs fall somewhere into, but they are all more or less forms of slavery. On the other hand, a financially independent lettuce picker, who picks lettuce for fun, is not a slave, no matter how uncomfortable the work is.

The central variable in the slavery equation is \emph{control.} That's really the motivation behind slavery. \emph{The more control you have over your employees and customers, the more employers profit, and the less they risk.} So there's also the client side to this equation. It's in the interests of companies to maintain as much control as possible over what they sell. If it were up to them, they'd be able to control everything we do with what we buy from them. For example, they'd love to prevent us from peaking inside their products to see how they work, so we can't create them ourselves or sell them cheaper. They also wouldn't want us to tweak what we buy to meet our own needs, but instead pay them to do it for us. Intellectual property laws give them some rights to do this, to incentivize innovation, but these rights are limited. It also hasn't been technologically feasible for them to enforce many of these desires for control, until now.

More and more of the products we can buy are electronic. There has been a trend in the last couple of decades to build software into everything we buy, to interact with it in such a way that it will stop working if we try to tinker with it. Additionally, much of the actual software we buy isn't owned by us, but licensed, under extremely restrictive terms that most consumers don't know about. As technology and software become increasingly integrated into our daily lives, it becomes essential that we demand software freedom.

For example, voting machines were impossible to peak into or tinker with, so it was impossible for independent security experts to validate that these machines work as advertised, with no back doors, or pre-programming to electronically ``stuff the ballot box.'' Indeed, many security vulnerabilities were found in these machines, and they were banned in many precincts. Entire governments also found themselves locked out of the content they created, because the software used to create it was no longer maintained by the companies they bought them from. This gives the software company sovereignty over that government! These are only a few examples that demonstrate why software freedom is essential to democracy. Ultimately, it's essential to our freedom as an increasingly electronic civilization.

It's also in the nature of companies to exploit cheap labor, wherever they can find it. In a global economy, they have no geographical barriers to where they can seek out cheap labor. As people become more affluent in one country, corporations would try to move jobs overseas, to exploit cheap labor there. As far as these corporations are concerned, it doesn't matter how cheap, so slavery is not out of the question, unless slavery is explicitly outlawed. Here I'm talking about the most extreme form of slavery, not just wage slavery like I discussed.

Slavery is in the best short-term interests of corporations, but it's not in the best interests of the people. It's therefore up to the people to demand that these corporations not unfairly exploit labor, foreign or domestic. This is a very tricky proposition, because the real problem lies with the governments and economies of these foreign lands. The awful conditions that these companies make people work in is better for them than not working at all, so capitalists justify this exploitation and say they're actually \emph{helping} these people by treating them awfully. This is that \emph{unreasonable choice} I mentioned earlier. Sure, tortuous working conditions is better than not working at all, but that's not a real choice. That's slavery, and people have a right to freedom, no matter where they live. So, we shouldn't demand that these companies not hire overseas, but to hire overseas and treat all of their employees with human dignity and respect, no matter where they live.

We have three kinds of voices with which to make our demands. One is government. Another is shareholder advocacy, but that will be harder to accomplish without more successful, lower cost, socially responsible mutual funds. Another is our dollars. Everything we buy supports something, and it's up to us to make sure our purchases support what we believe in. Money is like the air supply of corporations, and people collectively control that–seriously threaten their money supply, and the corporations will listen to any demand.

There are 27 million slaves in the world today, more than ever before in history. Slavery didn't start in America, and it didn't end with the Civil War. It has existed throughout history, and it continues to this day. Some of it still happens right here in America; it just went underground. There is still human trafficking happening right here among us.

Children tend to be the most likely victims of slavery, because they're helpless and dependent on adults, and more likely to be ignorant of their rights. It's so easy for adults to enslave children that they often do it unconsciously. Grounding kids, for example, is a form of captivity. Corporal punishment for not doing the will of parents or teachers is a form of slavery, not unlike the whippings that the slaves underwent. It's usually less violent, but not always. Forcing children by law to attend school and learn what they're told to learn is also a form of slavery, since they are uncompensated and don't have a choice. Schools are like totalitarian regimes. The principal serves as judge, jury, and executioner. There is no due process of law. The accused have no rights to a fair trial. Children can be convicted on hearsay alone. There is no democratic process, in which the kids have any real power to choose their laws and elect the officials who represent them.

There is a science fiction theme of aliens who come to Earth and deceive and conquer us so they can steal our resources and feed on humans. This theme touches on our greatest fears, what seems to us the greatest evil because the scale of slavery is extended to the entire human race by another species, rather than just pockets of slavery of some of us by others of the same race. What I find even more scary than this science fiction theme is that it's not fiction at all. I'm not talking about an alien conspiracy. I'm talking about animals. Humans enslave animals, and exploiting them for labor and experiments is the least of our transgressions. We are that alien species that deceives and conquers entire species, steals their resources, and feeds on them. Truly ending slavery must also include animal welfare. This is just compassion for all living things, not veganism. How this compassion manifests in your life is your own choice.

In a sense, we're also captives of our own civilization. Our civilization has served us well for thousands of years, in the sense that it has brought significant affluence to a small number of the more privileged classes. It's also been a disaster in many ways. Throughout history, for every class of affluence, there has been a much larger poverty class that resented that affluence, and sought to overthrow it. This historical pattern has not ended.

Other cracks have been appearing in our civilization just in the last hundred years in the form of environmental catastrophes. Just as our civilization is not in harmony with the well-being of many classes of humans and animals, it's also not in harmony with ecology. For all our civilization, environmental dangers remind us of our limitations, that we are just another species of animal, after all, subject to the same ecological laws as the rest of nature. Ultimately, our civilization must be in line with these laws, or our civilization is unsustainable. Unsustainable, as in, doomed to collapse. Insofar as we feel powerless to change our civilization, to create a new one that works better for humans as a species of animal in ecological harmony with nature, we are slaves to our civilization.

Another way to look at this is that we're treating nature as our slaves. We don't limit our enslavement of other species to just other animals. We treat the entire ecosystem as our property, forced to do work for our benefit only. We kill anything that tries to benefit from nature besides humans, except that which we find useful to us, or that doesn't get in our way. Nature is not our property, but the reverse. We belong to nature, we intimately depend on the health of the ecosystem of which we're a part, and we will only survive if we are not a threat to the ecosystem as a whole.

The rigid roles our culture pressures us to conform to, either subtly or overtly, can be considered forms of slavery to our culture. This is especially true for gender roles. Men are expected to be providers and protectors, and women are expected to be beautiful and take care of the house and kids. We've evolved this way because it helps the survival of our culture, to protect and propagate, in effect enslaving individuals for the good of the culture. With modern weaponry and overpopulation, our rigid roles are actually detrimental to our culture, so it's time to stop enslaving ourselves to these roles. Gender roles leave a lot fewer choices for both men and women, and freedom is about choices. Additionally, commercialism feeds on our fears and pressures, causing both men and women to pay dearly to maintain their roles. To some extent, gender roles are innate, but insofar as we have control over them, we should exercise that control, if we hope for true freedom. This is egalitarianism, not feminism–these days men are more inhibited by their gender roles than women.

I'm currently single, but I've found my obsession with freedom to be tremendously helpful in my relationships. It makes good relationships better and bad relationships end sooner. I'm always astonished when I hear people speak of their relationships as a ``ball and chain.'' People feel enslaved by their loved ones, and many even feel this is somehow healthy. Taken to the extreme, this is called codependency. True love can only exist when there is freedom. That doesn't mean no limits. It meas that each person chooses the other, and limits are negotiated from that choice. For example, I'm happily monogamous, but most of my closest friends are women. I adore them, I wouldn't change this for anyone, and I wouldn't want my spouse to limit their closeness with other men. If she were to fall in love with one of them, monogamy would be negotiable. If it had to end, I'd be sad for the loss, and I'd certainly feel jealous, but I'd let her go. Hopefully I'd be big enough to bless their relationship, and remain friends with both of them. Anything less would mean I would want her to stay with me even if she doesn't choose me, and that's slavery, not love.

Many of us in this culture weren't taught to relate to one another on an intimate level. We argue and strong arm each other to give us what we want. But even if one person wins and the other loses, resentment builds, more resistance builds, and in the end, we all lose. The key to freedom in all our interpersonal relationships is communication skills. The more honest we can be about our feelings and needs, and feel safe that we know how to communicate these feelings and needs in a way that won't hurt others, the more authentic and free we can be with one another.

These are forms of \emph{psychological slavery.} We're motivated by our own unconscious fears and pressures, not consciously by our own values and passions. We become a slave to our own lack of consciousness. Financial independence is about becoming more conscious of our earning and spending, the motivations we have for these, and the trade-offs we make with them. It's about being aware of our higher values at all times, and having all of our choices stem from those values. So, in a sense, true freedom lies in our awareness and mindfulness. I've found the best way to become more mindful is to set aside a time to practice it everyday. I use meditation, but anything can be used as a mindfulness practice, like yoga, weight lifting, eating, walking, even washing the dishes, as long as you stay present with what you're doing.

Many people are even slaves on a \emph{spiritual level.} Many churches depend on hierarchy and power, pressuring us to behave and live in certain ways that they dictate, to not be sinful by their own definition of sin, while using heaven as the carrot and hell as the stick. Often, this isn't the original message of the prophets and gurus these religions are based on. Most of them actually taught freedom. They taught freedom from greed and passions that might consume us. They taught people to live simply and help their neighbors. They taught people to live from a place of love, not fear. These messages are often warped by those who seek to control others.

Earning our own freedom by achieving financial independence is only the beginning. There are billions of people, adults and children, animals, even our entire civilization and its relationship with the ecosystem, that need more freedom, on multiple levels. Financial independence is only one step, like opening your own prison cell door. There are billions of other cell doors that need opening, and who better to help the slaves and captives than one who is already emancipated?

\newpage
\section{Resources}
\begin{itemize}
\item \textbf{\emph{Ishmael} by Daniel Quinn.} Explains how we are captives to our own civilization, and how this civilization is not in harmony with the ecossystem of which we're a part. This book changed my life more than any book I've ever read.

\item \textbf{\emph{Walden} by Henry David Thoreau.} This is a classic book by one of America's greatest minds and writers, describing his two year experiment in simple living. He built a cabin in the woods, next to Walden Pond, so he could face the essentials of life and gain perspective on society.

\item \textbf{\emph{Free at Last} by Daniel Greenberg.} Gives an overview of Sudbury Valley School, which is governed by a participatory democracy, and believes in the birth right of children to their freedom, and the innate ability of children to learn.

\item \textbf{\url{http://www.gnu.org/philosophy/shouldbefree.html}} -- Makes the case for software freedom.
\end{itemize}
