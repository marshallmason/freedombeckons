\chapter{Do you need to be rich?}
By global standards, most Americans are extremely rich, whether they feel that way or not. The median American income is \$31,410 per year.\cite{2007-median-income} That's in the top 6.33\% richest in the world.\cite{globalrichlist} This is the \emph{median} income, which means half of Americans are even richer. This book is intended for Americans and citizens of other first world countries. In other words, it only works for rich people.

Of course, many people want to be richer than they are. They dream of the things they can buy, but they also dream of security and freedom, of financial independence. I feel sad when see them at the corner store, buying lottery tickets. I cringe when I watch them crowding into hotels in Las Vegas, tugging levers at slot machines. The odds of realizing their dream of freedom in this way is so tiny as to be non-existent, but that dream makes the cost worth it for them. Just for a chance at that dream, no matter how minuscule, they're willing to pay large sums of money over time. I just want to shake them and tell them there's a better way.

It is possible for most Americans to achieve financial independence. That doesn't mean it's easy. By comparison to the lottery and Las Vegas, this approach is more effective, but not as much fun. How long it would it take for the average American to achieve financial independence depends less on their income and more on how much they spend and how they invest their earnings.

When I became determined to retire early, I was extremely frugal. I lived on soup and peanut butter sandwiches. I slept on the floor. I wore hand-me-down jeans with large holes in the knees. The soles of my shoes had come apart, so my feet got drenched whenever it rained. This didn't feel like deprivation. It was a gift to myself, a gift of freedom. I'd lived like this for a decade already, so I was used to it. All I had to do was keep it up. In fact, I felt like I was living in luxury, because I could finally afford a new car and a quality guitar. Those were my only significant expenses other than rent and a meager grocery bill. This lifestyle allowed me to keep my expenses came to only 40\% or less of my salary, after taxes.

Let's suppose the average American, earning \$31,410, was able to live on 40\% of their salary. Taxes would bring their salary down to \$25,000. Their annual expenses would be \$10,000. This is not easy---it's below the poverty line---but it's possible. Many Americans live on less. This person would need to save around \$250,000 to achieve financial independence. They would have \$15,000 to invest each year. At this rate, it would take them only eleven years to achieve financial independence, using the average historical return rate of a diversified portfolio, and accounting for inflation. If the expenses are higher or the return rate lower, the time it takes to achieve financial independence goes up very rapidly:

\begin{table}[ht]
\caption{How long it takes to achieve financial independence}
\centering
\begin{tabular}{c c c}
\hline\hline
Expenses & Return rate & Years \\
\hline
40\% & 8\% & 11 \\
40\% & 7\% & 13.5 \\
50\% & 8\% & 14 \\
60\% & 8\% & 17.5 \\
40\% & 6\% & 17.5 \\
70\% & 8\% & 21.5 \\
40\% & 5\% & 24.5 \\
\hline
\end{tabular}
\end{table}

If your income is above the median, then the time it takes you will be less. If it's below, it will be more. Likewise, if you're working during a bear market, or a time when inflation is high, your return rate will be pulled down, no matter how well you invest. If you're working during a bull market, or inflation is low, your return rate can be even higher than 8\%.

This example is intended to demonstrate how realistic this goal really is for the average American, not as a gauge for any one person's situation. In Chapter 5, I will show you how I came up with these numbers, and how to project for yourself how long it will take you to achieve financial independence.

\section{The difference between creativity and luck}
When someone does something creative to earn money, others might write that off as good luck. If someone toys around with something and finds a way to turn that into a lucrative business, people might assume they were just lucky. Winning the lottery is luck. Finding a clever way to make money is creativity. Luck is not necessary to achieve financial independence, but creativity is.

Everyone is lucky in some ways and unlucky in others. I was lucky to be born in America. I was lucky my parents helped out with college. I was lucky to be born with genetics that included good logical and math skills. I could probably give another list, just as long, of the ways I was unlucky. The point is, take what you're given and use it as efficiently as possible to make your life better, given whatever liabilities also stand in your way of that.

When people ask me how I retired early, maybe at a loud party, or in passing conversation, I can't sit them down and outline every detail of the program I followed, each of which is quite important. Even if they had the patience for this, I usually don't. In order to ease their anxiety and for the conversation to move forward, sometimes I give them a simple explanation: \emph{stock options.} I try to pepper this with caveats, but this answer seems to be all they want to hear. It's unique enough that they can easily chalk it up to luck.

I didn't decide to become financially independent after somehow stumbling onto stock options. It was the reverse. I sought out stock options because of my goal to become financially independent. When I fully committed to a decision to become financially independent, I had no stock options, and didn't even know such a thing existed. I was about a year away from graduating with a degree in computer science. I was aware that this was a lucrative field. My first step toward financial independence was to evaluate my skills and look for a creative way to use that to earn money. In my research, I discovered that technology companies were offering stock options to lure new employees, so I investigated this further. While most of my peers were going into graduate school or taking government jobs, I applied to a company that offered stock options. This was a choice I made, not luck. In fact, I had pretty lousy luck.

Stock options have what's called a \emph{strike price.} This is a fixed cost at which I had the option to purchase our company's shares. If the stock fell below that price, then I didn't have to exercise those options. They became worthless. If it went over the strike price, then I could do a \emph{same-day-sale,} which means that I could exercise the options and sell them immediately, using the proceeds to cover the cost. I got to pocket the difference, with no initial capital and no risk. I was given a generous grant of thousands of stock options at a strike price of something like \$130 per share. If the stock price doubled–something it had already done several times within only a few years–then I would be financially independent. Unfortunately, I was hired at the peak of the internet bubble. A year after I was hired, the stock was worth \$15. All my stock options were worthless, with no hope of recovering.

Although I was disappointed, I never intended for my strategy to depend on luck. Stock options would have made the whole project much easier, but I never counted on it working out. I focused instead on working hard and getting promoted quickly, significantly increasing my salary. I eventually received more stock options at a lower strike price, but much fewer than my initial grant. I didn't hold onto these very long, waiting to strike it rich like most people were trying to do at the time. I sold these modest stock options pretty quickly and used this money to fire up the investing component of my plan. This definitely gave me an early boost, but not a very significant one.

From the table above, you can see that the stock options didn't help me much. I retired only a few years shy of the projected 11 years, which you'd expect anyway for someone with an income higher than the median. I was also working during a bear market, so my return rate was pulled down. Without stock options, it might have taken me another year or two, maybe the full 10 years I'd originally set as my deadline. What really made all the difference was the salary increases, my low expenses, and my investment strategy.

\section{Increase your earnings}
The importance of keeping your expenses down and your investment returns up is good news, since most people have more control over these than their earnings. However, earnings are also important, and I believe most people have more control over their earnings than they think. The higher your earnings, the sooner you can achieve financial independence, and the higher your expenses can be.

Let's suppose our hypothetical average American doubled their salary, as I did in the 8 years I was working. Now they're earning \$62,820. After taxes, this would be \$51,000. This higher income allows them a lot more flexibility. They could double their expenses to \$20,000, keeping it at 40\%, and still achieve financial independence in 11 years. Or, if they could keep their expenses flat at \$10,000, then the portion of their income that they spend drops from 40\% to 20\%, and they could achieve financial independence in less than half the time. They could also choose something in between:

\begin{table}[ht]
\caption{How increasing earnings effects time to financial independence}
\centering
\begin{tabular}{c c c}
\hline\hline
Expenses & Return rate & Years \\
\hline
40\%, or double & 8\% & 11 \\
30\% & 8\% & 8 \\
20\%, or flat & 8\% & 5 \\
\hline
\end{tabular}
\end{table}

Some people feel selfish or greedy when they think about how to earn more money, so they simply avoid thinking about it. They hope to find a way to do what they love and the money will follow. This is something everyone wants but few actually find.

Every second of your life is precious. You can never get it back. It belongs to you alone. You should spend every single one of those precious seconds the way you choose to spend it, not doing what others tell you to do. Yet you need money to survive, and to enjoy that time more, so you must trade some of that precious life. Essentially, our goal here is to optimize our lives, to find a way to minimize the amount of your life you trade and maximize the joy you trade it for. If you're earning less than you are capable of earning, then you're trading more of your precious life than you need to. In a very real sense, you're throwing your life away.

Think of what you can do with that time. What do you believe in? Who and what do you care about? Think of the causes you believe in, things that will help others or make the world a better place. I guarantee that you can more effectively help this cause if you had more time or money to devote to it. The more you have, the more you can give. \emph{Maximizing your income is actually the least greedy thing you could do!}

This may sound like a radical laissez-fair capitalist philosophy, like something out of an Ayn Rand novel, in which being selfish and greedy is somehow warped into being the most charitable thing possible. That's not what I'm suggesting here. I'm advocating that you maximize your income so you can free up your life for the people you love, causes you believe in, and activities you value. Demand the maximum value for your life energy, not because you're greedy, but because your life is precious, and you want to spend it in the ways you choose to spend it.

\section{Reconsidering your career}
There are many advantages to working. The most obvious benefit is money, but there are others, such as enjoyment, community, challenges, or being of service. Often, there are trade-offs to consider. It's up to you to decide what are the right trade-offs for you, but don't get ripped off by cheap thrills. For example, if your company gives you free coffee and free t-shirts but a low salary, then you're getting a bad deal. However, if you enjoy what you do, you like the people you work with, and you're treated well, then you may decide that the trade-off is worth it. Just remember that it's possible to enjoy what you do, like the people you work with, and be treated well at a higher-paying job.

If you intend to be in the workforce for many years to come, then you need to be conscious of your career, not just your job. Your \emph{job} is the work you're currently doing. The main purpose of having a job is to earn money. Your \emph{career} is the long-term trajectory of your work choices, and includes bigger objectives with less tangible motivations, such as job satisfaction and benefiting your community or the world. When planning your career, it's important to have goals and milestones---the kind of work you'd like to be doing, the kind of company you want to work for, and a title and salary you want. Consider all your needs, not just money. Navigate your way toward the kind of work that brings you more joy as well as money, and is most in line with your values.

Careers are not always a linear trajectory. Sometimes you have to take detours to accomplish your ultimate goal. You might need to take lower-paying jobs so you can develop a particular skill, or meet people that may help you accomplish your ultimate career objectives. On the other hand, a particular job might be good for your salary but bad for your career. It's wise to avoid a high salary at a struggling company, for instance. They may not be able to afford to give raises or incentives, and they could go out of business, putting you out of work. Then you have a short stint at a failed company on your resume, which isn't good for your career.

Financial independence is a unique career goal, especially if you also plan on retiring early. When you want to achieve financial independence, your primary goal is to increase your salary as quickly as possible, while attaining skills and experience becomes secondary. Your goal is freedom, not a magnificent resume. Beware of employers that brag about ``career opportunities.'' It's much harder to measure career opportunities than salary, and therefore harder to compare with other offers. It could be a justification for paying their employees poorly.

You need to decide whether your current career is best meeting your financial independence goals. This question is even more important for people who have no career yet---students, and those still searching for direction---because they've made fewer investments in their careers, and the cost of switching careers is lower.

The point of financial independence is freedom, not huge wealth, so don't let your plan depend on high-risk choices working out. Going to work at a flashy start-up company, taking a significant pay cut in exchange for stock options, or starting your own business in a brand new field, all entail significant risks that may set you back in your financial independence goal.

If you decide to change careers, then you'll have more choices, but you'll also have higher costs and risks. Education is a huge cost. If financial independence is your goal, then I advocate minimizing your education costs. Education doesn't just cost you in the money you spend on it, but also in the time you spend studying and going to classes. I don't believe a graduate degree is a good investment in most fields, but this varies quite a bit. For some fields it's a good investment, and for others, it's a requirement.

If you're considering changing careers, or if you haven't yet chosen a career, here are some tips:
\begin{itemize}
\item Focus on your strengths. If financial independence is your goal, this is not the time to strengthen your weaknesses. You'll get greater benefits by leveraging your strengths. Make a list of all your strengths, even if you can't think of how to use them to earn money.
\item Dream big. In the brainstorming phase, ignore the limits of possibility. Too often, people confuse their fears with their perceptions of what is possible. More is possible than you may believe. Throw away the words ``can't,'' ``impossible,'' ``but,'' and ``should.'' Ignore income and location. Don't limit yourself to what you think you're good at. Assume you can do anything. Dream up the perfect job. Don't stop at one. Fantasize about lots of perfect jobs. Then include great jobs, good jobs, and decent jobs, even if they may not be perfect.
\item Only after you've dreamed big should you account for all your career goals. Include not only your skills, but also your interests, values, and other preferences such as geography and working environment.
\item Research your choices. Read books, talk to people in those fields, and bounce ideas off of other entrepreneurs to get a sense of exactly what limitations there really are. Then start eliminating ideas. For the purpose of financial independence, scratch off any idea that will take too long to get started or has too high of a startup cost.
\end{itemize}

\section{Entrepreneurship and raises}
Creative options for earning money tend to fall into two categories: \emph{entrepreneurship} and \emph{salary raises.} Entrepreneurship is usually more lucrative because you're limited only by your own imagination, but it can also be more risky. Entrepreneurship is about meeting unmet needs. Get to know the supply-and-demand dynamics in the industry or your local community. If you notice many people feeling frustrated about the same thing, that indicates an unmet need.

For example, suppose you live in a mountain town. You hate shopping for groceries, so you make few trips and stock up. Then you realize there's only one small store in town. You and your neighbors have to drive through mountainous terrain every time they shop. Everyone in town shops there, so there's never enough parking available. That's an opportunity to open a store closer to your home, with plenty of parking.

Some significant needs can be overlooked for years by those less adept at identifying them. This makes entrepreneurship a challenging and unique skill. Whenever you find a lot of demand for something, consider what opportunities you have to meet that demand. Opportunities are very specific to your own situation, skills, and creativity. One advantage of entrepreneurship is that it's often possible to do in addition to a full-time job, to supplement your main income. For example, when I was in college, I advertised in a local newspaper, offering guitar lessons.

The other category is salary raises. Opportunities for raises are usually limited by career, employer, and skills. The first thing to look for is any incentive program your employer may offer, such as stock options, cash bonuses, and stock purchase plans. \emph{Stock purchase plans} allow employees to buy shares of the company's stock at a discount. They can sell those shares at full price, pocketing the difference. This is similar to stock options, but it entails more risk, since employees have to put their own money on the line. Some employers also offer cash bonuses for innovations and patents.

The most common way to get raises is to be promoted. A vague and uncertain way to do this is to work hard and pray the boss notices. Get into work early, and leave late. Eat lunch at your desk. Take pride in your work. Ask your boss for more responsibilities. Another way to get promoted is to gain new skills. If you get trained in a certain skill, you may not only get a raise, but you can also put it on your resume, making it easier to get hired with a higher salary in the future.

You should be aware of office politics. This may sound cynical, but it is a reality you should not ignore. You may be the hardest worker, always exceeding what is expected of you, but consistently overlooked for promotions or raises, while others who don't work nearly as hard but are better at being noticed get raise after raise. A former co-worker once told me, “if nobody knows about it, it didn't happen.” Being a good employee isn't the way to get raises. Being \emph{noticed} as a good employee is.

One technique that helps with this is to keep notes. Maintain a \emph{work journal.} Keep track of every project you complete. Include the date they were started and completed, how many projects you juggled at once, some of the challenges you faced and overcame, who you worked with, and anything else that might be relevant.

Some employers offer an annual salary review. This is the logical time to gather these notes and present a case for yourself. However, I've found that a year is too long. By then, anything you did that disappointed your boss is impossible to correct. One thing you can do is ask your boss if you can meet with them more frequently, so you can gauge your performance. In these meetings, ask them for ideas of improvements you could make. Keep notes of these meetings. When it comes time to ask for a raise, you can present this as evidence.

Account for inflation when evaluating your salary raises. Each year, your salary loses buying power. Inflation can vary, usually between 1\% and 5\% each year. So, if inflation is 5\%, and you got a 4\% raise, then you've actually taken a pay cut. Subtract inflation from your raise for your true raise. Your raise may have been a pay cut in disguise.

You may already deserve a promotion or a raise, but you're being overlooked because you're not saying anything about it. Companies have an incentive to increase their earnings and minimize their costs. Employees are a cost, but replacing employees costs even more. If they can keep you happy with less money, they'll do that. Sometimes, they also try to foist more responsibility on workers without compensating them for it. This is especially common in tough economic times because they have more pressures to cut costs, and they know employees have fewer options in the job market. Years after I was hired, I was astonished to learn about the salaries being offered to new hires. This annoyed me at first, but then I realized the company has no incentive to pay me fairly unless I give them one. I made a case for myself and was given a mid-year raise.

\section{Trading up}
When you work for an employer, you are limited to their salary rates. Your labor is a commodity that you are selling and your employer is purchasing. When they're in the market for employees, they will try to get the best deal for their money, but they have to pay the current market rate for the skills they seek. They have no such limitation with existing employees. Therefore, if you want to increase your salary, you must keep shopping around with other employers. If you find another employer that offers a higher rate for your skills, you can present this to your current employer and ask them to give you a raise. If they refuse, then trade up.

\emph{Always be ready to trade up.} Keep your resume updated, and on a regular basis, interview at other jobs. Aside from keeping your interviewing skills sharp, this will ensure that you will always receive the highest possible salary.

If you feel loyal to your company, remember that your time is your life. It is precious, and you are trading it for money. Being respectful of your time means paying you what your time is worth. Also, loyalty should go both ways. When money gets tight, will your company be loyal to you? It can feel wonderful to be part of a team that works together and cares about each other's well-being, but this can obscure the business nature of the relationship.

When you're looking for a job, it's important that you know ahead of time what salary you deserve. Most employers have done some research and already have in mind a specific salary range for the position you are applying for—not an exact dollar figure. If they offer you a job, they will likely offer the lowest amount in this range they think you might accept. If you accept this offer with no negotiation, then you're cheating yourself. So have your own salary range in mind.

Make sure you consider your \emph{total compensation,} not just your salary. Include any benefits, such as health insurance, and subtract any extra commuting costs, moving costs, and living costs. Some employers also offer certain incentives for long-time employees, such as more vacation time and greater retirement account matching.

The initial salary is the basis for all of your future raises. Employers tend to give their raises as percentages of your current salary. If you start with a higher salary, the raises you get can really add up over time.

Be confident in your skills, and consider your time and labor valuable. Don't go into job-hunting with the mind-set of a beggar. Companies are not charities. They're hiring so they can meet their own needs, not yours. You must be confident in your ability to meet the needs or solve the problems they have. This doesn't mean that you must know how to do everything or solve every problem, only that you're willing and able to learn how.

Maximizing your income is a continual process. Once you find a job you're happy with and pays you well, focus on it for a while, but continue re-evaluating over time, both within your job and within the market as a whole.

\newpage
\section{Resources}
\begin{itemize}
\item \textbf{\emph{What Color is Your Parachute? 2011: A Practical Manual for Job-Hunters and Career-Changers} by Richard N. Bolles.} This is the most frequently cited resource for job-hunters and career-changers.

\item \textbf{\url{http://www.globalrichlist.com/}} -- The Global Rich List will tell you how rich you are compared to the rest of the world. You may be startled to learn just how rich you really are. This website is a little preachy, but effective.
\end{itemize}
