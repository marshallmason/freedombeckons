\chapter{Yeah, But\ldots}
\fancyhead[CO]{\emph{Yeah, But\ldots}}
\fancyhead[CE]{Freedom Beckons}

\section{Excuses}
``Yeah, but\ldots'' are words I often hear when I tell people about financial independence. When I hear these words, I take it as a sign that they don't really want financial independence so much as to convince me or themselves that my situation is unique, or their situation is unique, so they can't achieve financial independence. If you're focused on the reasons you're unable to do something, then you'll be unable to do it.

I'm not here to tell you how to live your life, or what choices you should or shouldn't be making. To do so would completely miss the point of what I'm trying to do. My goal is to help you increase your amount of freedom, and freedom means making your own choices. All I want to do is present new perspectives to help you re-evaluate your choices. If you're totally happy with your choices, and confident that they are right for you, then you don't need this book.

These are all of the common ``yeah, buts'' I've heard over the years, along with my rebuttals. Some of them are about financial independence, and some are about early retirement specifically. If you're not interested in financial independence, these rebuttals are not intended to change your mind. But if you are serious about it and you have some doubts, or you're having trouble with people in your life having doubts, refer to these.

Of course, it's up to you to find solutions to your own problems, and the first step in doing that is to believe with all your heart that you will. My answer to your concerns may sound insufficient or discouraging, but it doesn't mean you should give up. It only means there's no easy answer that I know of, so you'll need to use your determination and creativity to find your own answer.

\section{Yeah, but\ldots I don't earn enough money}
It's a common myth that you need to have a ton of money, and therefore need to earn a ton of money, in order to achieve financial independence.

You do need a lot of money, but probably less than you think (see Chapter 5 to help you figure out exactly how much you'll need to earn). Financial independence is also a factor of how much you spend and how you invest your savings, which is under everyone's control, no matter how much they earn. Financial independence is the inevitable result of living beneath your means, and spending less than you earn. If you can do this, then you can become financially independent.

The only real questions are, how long will it take, and is it worth it for you? If you don't earn very much money, that means you will have drop your expenses even more, and it will take you longer. It may not be worth all that effort for you to be financially independent, but that's a different issue.

Additionally, it's also possible to earn more money. You could change careers or advance in your existing career. Think about your skills and your gifts, and look for creative ways to put them to better use to earn you more money. This is the subject of Chapter 1.

\section{Yeah, but\ldots I have so much debt}
People who have a lot of debt feel very weighed down by it, so it seems impossible for them to ever get out of the hole. The compound interest of debt has the effect of pulling people down further and further, even as they pay it down as fast as they can. It's like running a race in which you start far behind the starting line, with a head wind against you. You run as hard as you can, and you wonder if you'll ever make it to the starting line, let alone the finish line. This is why debt can feel very discouraging, but it's not hopeless.

We all start somewhere. You have only one place you're capable of starting: right where you are. You can blame yourself for getting into debt, and you can wish you didn't have to start so far behind the starting line, but when it comes down to it, blaming and wishing won't \emph{change} where you are. You must start where you're at, for better or worse. The only way to change where you are is to start by accepting where you are. The only way to get out of debt is to pay it off. Wishing you didn't have to pay it off won't pay it off any sooner.

I too started out with some debt. One way I found to make it more encouraging was to treat it like investing. When I accepted where I was, I treated my interest payments as something like rent, just inevitable payments I must make to maintain my life. This way, any extra I pay is like an investment, which earns interest, by taking away from the payments that I'd otherwise have to make in order to maintain my life. The interest rate of this ``investment'' is equal to that of the loan I'm paying. For example, when I had a car loan at 8\% interest, I figured, what a great deal! Where else would I find a savings account that would earn me 8\%, risk-free? I excitedly invested in this savings account called my car loan.

Some people have a lot of debt. When you look at your amortization schedule (which I mentioned in Chapter 4), you may feel overwhelmed. You may find that it will take you many years to pay off your debt, so you may wonder why you should even bother. But those years will pass whether you pay off the debt or not. You can either stay in debt in those years, or you can get out of debt. Choosing to stay in debt doesn't make the debt go away. It only means you'll have to pay it later, after paying even more interest on it.

It is possible to achieve financial independence if you're in debt. It only means that it will take longer. Debt doesn't disqualify you from financial independence any more than a low-paying job or high-cost commitments. Even with debt, a low-paying job, and high-cost commitments, it is possible to achieve financial independence if you want it badly enough---if you work especially hard at the steps I outline in this book. Your biggest limit is your own devotion to your goal.

Even if you make your amortization schedule and you estimate that you won't be able to achieve financial independence for a long time, even if you do the steps in this book, that still doesn't mean that you can't achieve financial independence. It only means that it will take you longer than you would like. But your only other choice is to not do the steps in this book, in which case it will take you even longer. I know it's discouraging to have such a lousy choice, but no matter how long it will take you, doing the steps in this book will always allow you to achieve financial independence sooner than you would have otherwise, and the time you get back is your own precious life.

\section{Yeah, but\ldots I have to support a family}
Freedom is even more important for families than for single people. A lot of parents get so wrapped up in supporting the family financially that they miss the whole point. Do you spend as much time with your family as you or they would like? Would you like to learn to be a better parent, but can't find the time to read books or attend workshops? Many families are well-off, even wealthy, but they suffer from a severe time poverty. They make the wrong trade-offs. That's what financial independence is all about: finding the right trade-offs.

As a head of household, your expenses will be higher than they'd be if you were single, but the problem is not fundamentally different. You need to question all of your choices just as you would if you were single. What is it that truly brings you and your family joy? Is it lots of stuff, or is it plenty of quality time with people you love? Even if you don't retire early, the program in this book will give you more time, more options, and less stress. You may choose to cut back your hours, or change to a job or career that isn't so demanding, allowing you to spend more time with your family.

One advantage you have with a family you live with is that you can share a lot of costs. If you're single, you need to buy one of everything for yourself, in smaller portions. With a family, you have an economy of scale. You can share many of the things you buy, and you can buy more things in bulk. Make sure you leverage that advantage as much as possible. Besides, isn't sharing what love is all about?

For lots of tips for how to save money while strengthening your family, take a look at \emph{The Tightwad Gazette,} listed in the Resources. Amy Dacyczyn was so good at frugality that she was able to live in a large farmhouse with plenty of land and raise a very large, happy family, while spending less than most single people do. She did work she loved, which afforded her a lot of time with her kids, and then retired early. You may not choose to be as frugal as her, but it's good to know what's possible.

\section{Yeah, but\ldots isn't it better to find work you love?}
A common misnomer is that there is a compromise between finding work you enjoy and achieving financial independence. Who said you have to choose? Why not find work you love until you become financially independent?

When you're forced to work for a living, your livelihood is at the whim of supply-and-demand, which can be pretty merciless. Everyone wants to do what they love, but there are only so many lovable jobs to go around, so most people end up doing work they don't much care for. Loving their job is something everyone seems to believe is possible, but few actually pull off.

So, yes, definitely look for work you love. Don't let financial independence get in the way of your search. If you do find a job you love, then count yourself lucky. If you don't find work you love, also remember that it is not a personal failure. So many career books make it sound like everyone can find a job they love, which leads people to wonder what's wrong with them if they can't. Nothing is wrong with you. It's simply a negative side effect of supply-and-demand. Work at the best job you can find, with the right trade-offs, and enjoy it as much as you can. Once you become financially independent, you'll have a lot more control over your life.

A good tip in general when working toward financial independence is to maintain an \emph{``and'' attitude} rather than an \emph{``or'' attitude.} Look for ways you can get everything you want in life \emph{and} become financially independent. Don't create artificial trade-offs. If you think you've found a trade-off, a choice between a higher-paying job you hate and a lower-paying job you love, keep searching for a higher-paying job you love. Don't just give up.

If you keep searching and you really can't find it, then, yes, you must choose what is more important to you. This is a difficult trade-off that only you can make. That still doesn't mean you have to choose between finding work you love and financial independence. It only means that it will take you longer to achieve financial independence than it would at the worse job.

\section{Yeah, but\ldots I like my job!}
This was the strangest thing. All my life, I'd hear people grumble about their jobs, and ``thank god it's Friday.'' I'd hear songs with titles like ``Take This Job and Shove It.'' I'd see coffee mugs with a bedraggled worker on Monday morning. I'd hear about various techniques used by workers to manage the drudgery of their time at work, like ``clock watching'' and ``looking busy.'' Then when I suggested that it's possible to escape this, suddenly everyone starts loving their jobs.

Maybe some of them do like their jobs, but just need to grumble occasionally about some of its less savory aspects. It's got some downsides, but on balance, they get more benefit from working than not working. But then I wonder, if they do in fact like their jobs so much, why don't they do more of it? It seems unlikely they would want to do the exact amount of work that is contractually required of them. You'd think, if they get the most benefit from their lives when they're at work, then the best thing they can do is to work as much as possible. Few people do this, and when they do, they're called workaholics.

I think what's happening here, at least in part, is good old fashioned rationalizing. For one reason or another, financial independence, and the steps required to achieve it, doesn't appeal to them. They don't like working, but they also don't want to do what's necessary to stop. And that's totally fine, but I wish they would just say that.

However, I suspect they haven't really thought it through. In fact, before they met me, I doubt most of them have even considered the possibility. People don't like being caught off-guard like this. They don't like to appear ignorant or complacent. The easiest way out of this cognitive dissonance is a quick rationalization. They figure there are actually parts of their jobs they like, so then they act like they've fully considered the possibility of financial independence and rejected it because they like their jobs.

I'm just speculating here. I don't really know what's going on in their heads. It could be that everyone I talked to before I wanted to achieve financial independence hated their jobs, and then suddenly I just started meeting tons of people who love their jobs. This is probably at least partly true, since I shifted from blue collar work to white collar work right around this time, and white collar work tends to be less humiliating and provides more autonomy. Still, I suspect there's a fair bit of rationalizing going on as well.

Even if you like your job, financial independence can make it even better. Financial independence gives you more flexibility and safety. You can get a better education to improve your job even more. You can weather lay-offs much better. And things change---you may burn out some day. You might decide you'd rather do something else. You'll know you're at your job for the right reasons. You're there because it enriches your life, not just your wallet.

Unless you plan to die at your workplace, you will have to retire some day anyway. Why not choose for yourself when that will be, and do some planning? If you plan to retire at 65, why is that? Is that really when you want to retire, or are you just going along with the default? If you really like your job, why would you ever want to retire?

All I'm saying is, be honest with yourself. Ask yourself some hard questions about the choices you're making in life. If you ask yourself these questions, and you're happy with the answers, then that's great. This book is for those who are still asking.

\section{Yeah, but\ldots what's wrong with working for a living?}
There's nothing wrong with working. I love working, when I choose it. What I don't want to do is \emph{work for a living,} to depend on that job for my livelihood.

It depends on your motivation. Are you working because you want to, or because you \emph{have} to? If you're not financially independent, then you \emph{have} to, whether or not you \emph{want} to as well. If you cherish freedom, I would think this would bother you.

Here's a weird analogy. Let's say you want to play a game of golf with your buddies. Then I put a gun to your head and say, ``play golf with your buddies, or die.'' Does that change anything for you? If you care about your freedom, I expect it would. It doesn't matter how much you'd want to do it anyway---this coercion will inevitably change that golf game for you. It would certainly make it much harder to enjoy the game. It also calls into question whether you actually wanted to play it in the first place. It gets really hard to be honest about your motivations as a free agent when you're no longer a free agent.

I'm talking about freedom, not work. It's possible to work and be free, and it's possible to not work and not be free. If you choose to work, then work. But if you wouldn't choose to work if you didn't have to, or you wouldn't choose the specific kind of work you're doing now, then you're not free. If your livelihood is tied to your work, then your authentic motivations will always be distorted by your motivations for survival, so it's difficult to be sure about what you \emph{really} want.

\section{Yeah, but\ldots what about \emph{right now}?}
One misunderstanding I heard when I told people about my goal of financial independence is that I am making huge sacrifices right now for some glorious future. What if I die before I retire? Won't I wish I lived it up now? What about all the things I'm giving up for this goal?

Every choice you make should be one of getting more out of life, not less. Not every choice needs to be getting the maximum joy out of life this very second. If it were, then there would be no reason to go to school or college. We make investments in our future all the time. The real issue they seem to have, then, is that the investment I'm making is above and beyond what's \emph{culturally acceptable,} ignoring what's \emph{personally acceptable} for me.

The way I see it, I spent 8 years in college so I could have a better life afterward, and then I spent another 8 years in work so I could have a better life afterward. Why are the second 8 years so different from the first? It's not that I had to suffer and sacrifice during those 8 years in college, or the 8 in work. I enjoyed learning and contributing as much as I could, given my lack of freedom. I spent my free time doing exactly what I wanted to do, and I wasn't afraid to spend money if I knew it would bring me great joy.

Something the objectors didn't see was how much better my life \emph{already was} than only a few years before, and this improvement was directly attributable to the choices I'd made to manage my money better and work toward financial independence. These choices didn't just benefit me toward my goal of retiring early---they had immediate benefits.

This goal didn't introduce any new sacrifices or trade-offs for me. All it did was make me \emph{conscious} of my trade-offs, so I could make wiser trade-offs. Maybe people saw this as some great sacrifice because I was making different choices than they personally believed are necessary for happiness. I craved freedom, not jet skis and sports cars. My tastes simplified as a result of this goal, and therefore what made me happy had shifted from what made most other people happy.

\section{Yeah, but\ldots there are other forms of freedom}
Indeed, there are many forms of freedom, and although I focus on it, this book isn't only about financial freedom. I will describe some other freedoms in the next chapter.

When I hear this ``yeah but\ldots'' usually they'll offer some examples of freedoms they cherish or crave. The most common ones I've heard are: the freedom to not worry about money and the freedom to travel a lot or live an expensive lifestyle.

What's happening here is that two different concepts are being conflated. What they are talking about are actually forms of \emph{abundance.} What I'm advocating for is \emph{liberty.} While I agree that it can feel very freeing to travel a lot or to not worry about money, these aren't basic freedoms in the same way as the ability to make your own choices in life without coercion.

There's a good analogy for this, called the \emph{gilded cage.} Suppose you lived in a beautiful, golden cage, only allowed out occasionally and briefly. Would you feel free just because you may enjoy the beautiful sight of precious metal? Perhaps you would, and if you had the choice between living in such a cage and not living in a cage at all, you would happily choose the cage. If that's true, that's fine, but I believe many people feel differently.

Those who say this ``yeah, but\ldots'' tend to be middle- or upper-class and have jobs that afford them a lot of autonomy, at least enough to meet their needs, and therefore they don't feel much like they're living in a cage at all. All I can do is wish them the best of luck with a lifestyle they clearly prefer.

When I began my quest for financial independence, this was not the way I was living. When, largely thanks to this quest, I did earn enough to become middle-class, I was granted more freedom. But by this time, it didn't feel like I was freed---I was just given a bigger cage, and I was just as determined to escape. I believe there are others who also feel this way, and that's who this book is for.

As for the freedom to not worry about money, I believe this simply doesn't exist for anyone. To be human means putting some thought into how you are going to meet your needs in life which, in a capitalist economy, means finding ways to sufficiently accumulate and spend money. Whenever you learn a trade, go grocery shopping, or interview for a job, you are doing this, even if indirectly.

Most people, no matter how wealthy, would say they would love to earn more money than they presently do, which illustrates how insatiable our appetites for wealth can be. Most people feel that pinch, that wish to do more than their current means allow. This is why credit cards are so popular. Financial independence doesn't create this pinch, and earning more money doesn't make it go away. It may alleviate it somewhat, but it also might make it worse, since our wants tend to grow as our means grow.

The popular way to alleviate that pinch is to have more of what you want, but there's a second way, which most people overlook: \emph{you can learn to want what you have.} I can absolutely say I feel that pinch less than ever before in my life. This is true for both reasons: I've increased my means and have been able to afford more of what I want, but I've also learned to be more content with a simple life.

I also have an issue with the word, ``worry.'' To worry about something is to be extremely concerned about and preoccupied with it, and I just don't feel that describes my relationship with money after I started on my financial independence goal. In fact, I worried about money constantly before this goal because I was always running out of it. I was worried every time I went to the ATM, wondering whether this would be the time it finally denies me cash. My financial independence goal freed me from this worry entirely. I always knew exactly where my money was, and I always had plenty of it.

I do make \emph{plans} regarding money, sometimes intricate plans, a process I quite enjoy. I became very conscious of my money and my spending, so I would never have to worry about it. I don't think being aware of something and being careful about the trade-offs I make with it constitute ``worrying.''

I only think about money when I'm doing something with it. The same is true for working. If you're making careful plans about a particular project while you're at work, and then when you go home, you forget about it, that's just planning. But if you feel preoccupied with it all the time and extremely concerned about the outcome, then your planning has turned into worrying.

Financial independence doesn't mean worrying about money all the time, but it does require thinking about money in new ways. If you have no interest in doing this, that's fine, but rejecting financial independence because you think it will require worrying about money whereas without it you never would, this is likely untrue.

\section{Yeah, but\ldots won't you get bored?}
You know what I find strange is that no one ever thought to ask me that until I started working full-time. As a child, I had summers off. No one expected I would be bored if I didn't have a job. I did get bored sometimes, but that tended to be when I got tired of playing with one thing and was looking for the next fun thing to play with. What I really found boring was school.

Now that I'm retired, boredom looks a lot more like it did when I was a child. I become bored when I tire of one thing and I'm looking for something else. Boredom is a sign that my flame is dying and it's time to re-ignite it.

When boredom comes, I try not to rush to fill the void. I find boredom and lethargy to be productive downtime. It's a time when I'm most open for inspiration. I'm ready to take a new path when it opens up for me. I might spend weeks or months just watching movies, reading books, and poking around on the web before I get excited about something. This is a cherished luxury that my retirement affords me.

What I did have to learn was to be self-directed again, after years of being told what to do. This can feel a little empty, and the longer you've worked, the more empty it will feel. My advice is to embrace that emptiness. It's a kind of ``detoxification phase'' that you need to go through to adjust to a passion-based lifestyle after years of a paycheck-based lifestyle.

Another form of this ``yeah, but\ldots'' is, ``yeah, but\ldots I went for a while without working, and I hated it. I didn't value my time, and I just sat around goofing off all day.'' As I explained in Chapter 5, a lot of people use their jobs as a self-discipline crutch. They're not used to having to structure their own time, so when left to their own devices, they won't do it. Read Chapter 5 for help with this problem.

For some people, this is kind of like the problem of having lots of food around when they're on a diet. One solution is to simply refrain from eating the food. Another solution is to have someone come and take the food away, so you're not tempted. If that's what you need to do, then do that. But there are those of us who crave options, and that's who I'm trying to reach here.

This ``yeah, but\ldots'' confuses financial independence with retirement, which I talked about in the Introduction. Retirement is about not-working. Financial independence is merely about freedom. The freedom to work, if that's what you choose. You may wonder, if you're just going to work anyway, then why does freedom matter? The answer to that depends on you. If you value freedom, then it matters. If you don't, then it doesn't.

This ``yeah, but\ldots'' also makes many assumptions about what retirement is like---notably, that it's the same as unemployment. Maybe this is true for some people. It wasn't for me. It was very much like the difference between leaving the prison grounds for a few days and being released from jail. It meant I could make plans for the rest of my life that I would never have done if I was just unemployed. This was the kind of freedom I craved, not merely the freedom to not go to work.

Finally, even if you're only productive a fraction of the day after you retire, you'll still come out ahead. I was so ridiculously unproductive with my free time when I was working that now I can goof off most of the day and still get more done than when I was working. How much down time you take in your retirement is entirely up to you. If you want to get more done each day, then get more done.

\section{Yeah, but\ldots aren't you just trading one job for another?}
One concern I hear a lot is that all this time I spend managing my money, researching investments, and cutting costs is a job in itself, so I'm not actually leaving work, just changing careers to become a professional investor.

Money management is not something special that only exists when your goal is financial independence. It's something that everyone has to do, to various degrees. They have to manage their accounts, balance their checkbook, set goals, pay bills, file taxes, and cut costs when times are hard. If they have a surplus, they have to learn about investing and research investment options. Some or all of this may be outsourced to professionals if one decides the cost of this is the right trade-off for them. The same is true for financial independence.

How much time or money you invest in money management is up to you. I have found that the time I spent on this paid itself back many times over. It also helps that I enjoy it immensely. For these reasons, I'm willing to invest heavily in managing my money. That's a choice I've made.

That said, I don't really spend a lot of time on it. Most of the time I put into it was up-front, learning about financial independence, investing, taxes, and frugality. I set up a target asset allocation and I buy-and-hold my investments. I usually ignore the economy as it ebbs and flows, and I don't do much about it. Every day, I spend a couple minutes updating my budget with my expenses for the day. Each month, I add everything up, balance my checkbook, and pay my bills. At the end of the year, I do my taxes. Every year or two, I re-group, do a bit more research, and re-balance my portfolio. Other than that, my money pretty much runs on auto-pilot.

On the other hand, I no longer have a ``second shift'' that comes with working. I don't have to shop for work clothes, prepare for the next day's meeting, commute back and forth, update my resume, hunt for jobs, read trade journals, or unwind from the work day. When all is said and done, I've definitely come out ahead.

\section{Yeah, but\ldots when are you going back to work?}
Some people simply can't fathom retiring at any other age than 65. They assume I'm just taking some time off work, so they wonder when I'm going back.

I try to explain that I'm done; I'm never going to work another day in my life. This just boggles their mind, as if they've just met someone who has decided to go live in a cave for the rest of their life. Then they think it's their duty to persuade me to go back to work, lest I live my life like a miserable hermit.

I've always found this concept of working until old age extremely bizarre. I remember a conversation I had with my brother when I was around 6 or 7 years old. I was excited about the upcoming summer because then school will be over. My brother said, ``well, then you'll go to school for another year.''

``Oh,'' I said. I was disappointed, but it didn't seem like a big deal. What's one more year? ``So, then I'll be done?'' I asked.

``No.'' He said that school lasts for twelve years. There's five years of elementary school, then middle school, and then four years of high school.

I was flabberghasted. Twelve years? \emph{Twelve?!} That seemed like an eternity. What on Earth would I need to learn that would require such an enormous investment of time? It must be something pretty important, I figured. ``So, then I'll be done?'' I asked.

``No,'' he explained. ``Then you go to college.''

``Well, how long is that?''

``Four to eight years, depending on what kind of degree you get.''

Now I was really exasperated. I'd be in my \emph{twenties} by the time all this work was done. \emph{My entire childhood would be gone.} But at least then it would be over, and I could do what I want with my life.

Right?

That's when I discovered that this huge chunk of time is just the lead up for a lifetime of working, until I'm an old man. Yes, then I would be free, but most of my life would be behind me. I didn't believe him at first, but he kept a straight face. I had to ask some grown-ups, and they all concurred. To this day, I've never made peace with this.

I've always looked forward to summers because that was \emph{my time.} My time to play. My time to sleep. My time to do whatever the hell I wanted with my life. But then summer would be over, and I had to go back to giving my time to someone else. What I want, what I've \emph{always} wanted for as long as I can remember, was a never-ending summer. That's what I have now, and I'm never going back. This is the perfect life for me.

That's not to say that I'm against working again. In a sense, I \emph{am} working, just not at a full time, paycheck-bearing job. As I explained in Chapter 5, I have very demanding boss, named Whim.

Does volunteer work count? Does it only constitute working if I earn money from it? If so, exactly how much payment is required before it's called working? If I play music every night and make \$200 in tips, is that working? Does the amount matter? What if it was only \$20? What if it was only the occasional nickel? At what specific point do you call it working? Do I need to earn a living at it? In that case, what if I had a fat inheritance but worked full time anyway? Does it really make sense to not call something work just because it's not your primary source of income? If you think about it, this whole concept of \emph{working} is not very precisely defined. It kind of means whatever each person wants it to mean.

To me, payment is irrelevant. Any payments I receive I'm just going to donate anyway. Therefore, the connection between labor and income remains severed and I remain retired. What matters to me is that I'm spending my life doing what I want. Should I refuse to do something I want to do just because someone wants to pay me for it? Of course not.

I do occasionally miss working with others to build something interesting. I sometimes miss the workplace environment. But then I remember that I can usually get these things by volunteering, without having to fuss with paychecks and demanding bosses. Maybe some day I'll find something I want to do that isn't available on a volunteer basis, and then I'll ``go back to work.'' But so far, it still seems unlikely, and even if I do, I won't keep any of the money. You can call that working, or you can call it whatever you want.

\section{Yeah, but\ldots what if you run out of money?}
A common variant of this is, ``yeah, but\ldots what if some catastrophe happens (and you run out of money)?'' This risk is not unique to early retirement. Everyone faces this risk. Having a job doesn't mean you can never run out of money. It happens to people all the time. When it does, they do what they have to do. They take a second job. They take out loans. They find some way to survive.

Retirement is unique in that being out of the workforce for an extended time makes it difficult to jump right back into your old career. This is a risk all retirees face, not just early retirees. It's why good planning is so important. It's usually avoidable if you've planned well enough, added a sufficient buffer for emergencies, and were careful enough in your estimates. Even if it does happen, you find some way to make it work. If all else fails, you find some kind of job to pay the bills, even if it's not in your old career.

Suppose the catastrophe not only causes you to run out of money, but also physically incapacitates you, so that you're incapable of going back to work. Again, this is not a risk unique to retirement. This can also happen if you're working. In fact, it's more likely to happen if you're working, because you're driving more often in order to commute to work and therefore more likely to get into an auto accident, and because many physical disabilities are caused by accidents at work. Even ``safe'' office jobs expose you to the risk of repetitive strain injuries. Some jobs offer disability insurance to protect against such accidents, but not all.

What's more interesting are all the questions I \emph{don't} hear as an early retiree. I never hear, ``yeah, but\ldots what if you lose your job?'' I never hear, ``yeah, but\ldots what if you end up stuck in a job you hate, and can't leave because the economy is in shambles?'' I never hear, ``yeah, but\ldots what if you realize that your career choice wasn't a good fit for you?'' I never hear, ``yeah, but\ldots what if your skills become obsolete?'' These are all risks people face by being in the workforce, which go away when you're financially independent. Going back to work is a contingency plan for retirees, but not for workers, because they are already working. I believe financial independence eliminates more risks than it introduces.

\section{Yeah, but\ldots the stock market is too risky}
This is usually emotional reasoning: it feels scary, so therefore it is risky. People who feel this way are usually ignorant of the historical reality of the stock market. They remember the big crashes, or they see it go up one day and down the next. They overlook the usual behavior and long-term trends. Most people who feel the stock market is too risky are surprised when they learn the stock market has gone up in every 10-year period in the history of the stock market---including the crash of 1929 and the Great Depression.

It may help to remember what is \emph{constant} about the stock market, invariants that you can always count on. Think of these as the two Laws of the Stock Market. The First Law is that it goes up \emph{and} down, down \emph{and} up. It does this every day, but also over longer time periods. This may seem trite, but it's so easy to forget. The stock market goes up for a while, and bubbles form because people think there's no end in sight. Then it goes down for a while, and we go into a protracted recession due to excessive fear. Even the most sophisticated market professionals, who've been doing this for decades, can forget this. But there's an important corrolary to the First Law: the stock market always goes up in the long run.

The Second Law of the Stock Market is that it \emph{will} crash, at least once per generation. That means you can plan on being treated to at least one of these crashes in your lifetime, possibly several, and they can't be timed. The Second Law, by itself, might sound like the stock market is risky, but only if you've forgotten the First Law: it always goes down \emph{and} up. These crashes afford some tremendous buying opportunities. Short term, the stock market is extremely risky. Certainly, too risky to invest in. However, long term, it's practically a sure thing. Long term, other risks start to overshadow it: inflation risk and opportunity risk.

The biggest risk you face isn't the stock market, but your own emotions and ignorance. Your best defense is to really know the risks going in, and how much risk tolerance you have. Investment risks can be managed without too much work. If you're scared of investing because of the risks, then just know that you have a low risk tolerance. There's nothing wrong with that. When you talk to a financial planner, or when you create your own investment strategy, make sure you account for your low risk tolerance.

If you have a conservative portfolio to accomodate your lower risk tolerance, then when the big crash happens, you'll still lose some money, but it won't feel so devastating. All around you, you'll hear news of the sky falling, while your portfolio has a hiccup and just keeps chugging. If you hang onto your investments during this period, or even better, \emph{you invest even more,} you will do quite well.

\section{Yeah, but\ldots it's too hard}
Financial independence does require a lot of work, research, and introspection. Some of what you've read in this book may intimidate you. Maybe some of it went over your head, so it sounded like too much work. Only you can decide if it's worth it for you.

Remember that I didn't learn all of this overnight. The first couple of years, I had no idea how I was going to do it. I just knew I had to, and that turned out to be enough. I spent a lot of time excitedly reading, thinking, researching, and calculating. I started out knowing very little, so if you're just starting out and you know very little, you're just where you need to be. Don't feel like you need to understand everything in this book right away. Just get started, and check out some of the resources I recommend.

Also, notice that I said the reading, thinking, researching, and calculating was done ``excitedly.'' That's because I knew this was allowing me to achieve a freedom I craved very deeply. I didn't do this work out of duty or self-discipline. I did it because I wanted to know everything I possibly could to reach my goal, and to discover every contingency I might need to plan for, so I could keep the freedom I'd worked so hard for.

\section{Yeah, but\ldots it takes too much discipline}
One reaction I often get when I tell people I retired early is that they admire my self-discipline. While I appreciate the compliment, it feels misplaced. These are people who get up every morning, rain or shine, no matter how they're feeling, battle traffic, spend half their day toiling at a job that most of them don't find very gratifying or enjoyable, battle traffic again to get back home, and finally collapse on the couch from exhaustion. They're so disciplined that even the amount they're allowed to be sick is often regulated.

By contrast, I sleep as late as I want and literally do whatever I want every day. I'm usually productive, but only when I'm working on exciting projects. Some days, I'm so lazy that all I do is lay around and read books or watch movies. So it sounds very strange indeed to hear these masters of self-discipline call me the disciplined one---even that they \emph{admire} me for this.

I didn't achieve financial independence because I'm disciplined. I did it because I \emph{lack} discipline. I always have. It was the perennial problem that my parents tried and failed to remedy. But what I lack in discipline, I more than make up for in passion, determination, commitment, awareness, and maybe even obsession.

I did have a long-term perspective, and some may say that is what discipline is, but there's a key difference: discipline is about self-denial, whereas my approach was about abundance. Discipline is the ability to make yourself do things that you don't feel like doing. I didn't need that ability, because I \emph{always} felt like working toward freedom. It was all I could think about. The way I saw it, the only two choices I had was to work and not retire early, or work and retire early. What required all of my willpower---work---was necessary in both cases. I simply chose the path of abundance because, well, why wouldn't I?

When I became determined to find freedom in my life, and I fully appreciated the trade-offs I was making by spending wastefully and managing my money poorly, I began to have an adverse reaction to buying things I knew would have very little impact on my overall quality of life. I became agitated at the thought of having money sitting around uninvested that could easily be put to work on behalf of my freedom. I became so hyper-aware of the trade-offs that the truly disciplined thing would have been for me to waste my money.

If you find yourself struggling to muster the self-discipline to achieve financial independence, then maybe you're going about it wrong. It requires determination and awareness of trade-offs far more than discipline. I can't imagine this goal being accomplished by sheer willpower alone. While it certainly requires effort, it should not be the kind of effort it takes to do chores. If you're not thrilled by the project of financial independence, if you're not excited at the thought of spending your life exactly the way you choose to spend it, then it's probably not all that important to you, and you should focus your precious, limited life resources on things that truly bring you joy.

\section{Yeah, but\ldots honest hard work is good for you}
I call this the ``puritan defense.'' The line of reasoning is that work builds character, even though it can be challenging and sometimes even humiliating. I translate this as, ``suck it up, and take your lashings like a man. Stop being a pansy.''

Before I tackle this point, I want to address a very similar, less judgmental reaction to early retirement. People would hear me complain about work, and they start to wonder if I've worked anywhere else besides this one company since college. If I've only worked at one company, then maybe my perspective is skewed by a lack of experience? Maybe I just need to find a better job?

The assumption here is that most jobs are of the middle-class, white-collar variety, and therefore it is only my job experience after I graduated from college that should count. I've had dozens of jobs before that one, high-paying job as a computer programmer---janitor, data entry clerk, dish washer, public speaker, ski technician, pizza delivery, computer technician, newspaper delivery, tax preparer, research assistant, guitar teacher, and math tutor. Some of them were wonderful, but most were humiliating and painful, and I believe they far better represent the majority of jobs. These are the kinds of jobs I was reacting to when I complained about work.

Were these jobs good for me? Did they make me a better, stronger human being? Yes, some of them absolutely did. Some jobs, like my 17-hour days as a ski technician, were hard work but very rewarding. Others, like tutoring, were mentally challenging. Public speaking forced me to face some of my fears. So there is definitely some truth to this.

However, most of them, the worst ones, the ones I complained about the most, really only scarred me. I still have bitter memories that I wish I could erase. These experiences have made it more difficult to trust people, especially authority figures. Just because an experience is challenging doesn't mean you'll grow from it. Sometimes, it will just piss you off and make you determined never to be in that situation ever again.

But maybe that's the point? Perhaps I \emph{did} grow from these experiences in the sense that they motivated me to retire early? There are a couple of flaws with that line of reasoning as well. For one thing, I was dreaming of retiring early \emph{before} any of these experiences, not after. I was very motivated, but I gave up because I was in such a bad situation that I figured it was just impossible. Years later, I definitely used the anger from those experiences to motivate me to retire early, but I always wanted it, and I think I might have found a way once my situation improved.

The other flaw in this reasoning is best illustrated with an analogy. Imagine you lock someone in a basement for decades. You provide food, but he must perform a tedious combination of humiliating tasks in order to obtain it. You also make it possible for him to unlock the chains and escape, but how to do this is not obvious or even intuitive, and it will take years for him to accomplish. But he has a lot of ingenuity, and he becomes so angry that he's been locked up in this way that he devotes himself to the single-minded task of escaping, so he eventually does.

Was this experience good for his character? Should you use his escape as proof that what you did to him was all for his own benefit? Indeed, he probably developed a lot of skills in his quest to free himself from your chains, but you've also scarred him for life. Besides, what right did you have to torture and incarcerate him in the first place?

If someone wants to develop their character through hard work, I fully support that, but no one should be forced to. It's their life, and they should get to decide for themselves what kind of character they want to have. This is why I emphasize \emph{freedom} and \emph{financial independence} in this book, not retirement. Once you're financially independent, you can do whatever work, or lack thereof, that builds whatever kind of character you want. This also includes a whole array of unpaid, volunteer work, some of which can be back-breaking and stressful but arguably more rewarding.

\section{Yeah, but\ldots freedom is dangerous}
Some people fear too much freedom. ``Idle hands are the devil's workshop.'' They're afraid that too much freedom will cause people to do destructive things. They'll get bored and then start living a life of crime, or become addicted to drugs, or party every night. They look at the lifestyles of rock stars and they shudder at the dangers of having too much freedom.

This is actually very similar to an argument made by anti-abolitionists in the 1800's. Blacks were believed to be dangerous and stupid, not to be trusted with freedom. Best to keep them safely chained up and working for the rich white men, who were somehow considered perfectly capable of handling their freedom. Now we understand how flawed and hateful this reasoning was, but we also see how irrelevant the argument was in the first place. People don't deserve freedom because it's safe. They deserve freedom because it's a fundamental human right. Just because some people can choose things that will hurt them doesn't mean they don't deserve freedom.

And it's true. Some people will make poor choices with their freedom, but that's the point of freedom. It's the ability to make choices, both good and bad. Making choices sometimes means making mistakes, even deadly ones.

If it's idleness that worries you, then that's easy. Don't be idle. Financial independence isn't idleness. Financial independence is freedom. It will always give you more choices, not less. You can choose to not be idle.

\section{Yeah, but\ldots other countries are less free}
Once or twice, when I told people how much I crave freedom, they feel the need to give me some perspective. Relatively speaking, I've got far more freedom than most people in the world, as well as throughout history. This is true, and it's certainly nice to count my blessings and be grateful for the freedoms I do enjoy, rather than just whining about how I want more. On the other hand, this has the subtle implication that my feelings and cravings are invalid, just because it could be worse. As if everyone in the world should stop agitating for more freedom except for those who have the absolute least.

It depends on your frame of reference. Yes, compared to many other countries and other times in history, modern-day America is extremely free. I'm glad that I'm able to leave my house without fearing for my life, and I'm grateful to live in an economy that makes financial independence possible. But I'm kind of obsessed with freedom, so I tend to dream about what's \emph{possible,} not about what's \emph{worse.} As far as freedom is concerned, you could do a whole lot worse than present-day America, but compared to what's possible, I believe America still has a \emph{long} way to go. I'll elaborate on this in the next chapter.

Still, there are so many people, especially in third world countries, who can barely feed themselves or stay safe from the constant barrage of explosions and gunfire in their villages. How entitled must I seem to complain about the lack of freedom in America, of all places. How can I possibly speak of myself as if I'm a slave when I live in such ridiculous luxury? This one ate at me for a while. I'm guessing it has also crossed other people's minds as they listened to my rants. Perhaps it has crossed your mind as you've been reading this book.

People in the West live in an embarrassment of riches. This is partly due to the luck of geography\footnote{See the book \emph{Guns, Germs, and Steel} to understand just how large a role geography has played.}, and partly due to a more effective government and economy. The latter, some would rightly argue, was only possible because of the luck of geography.

Those who live in developed nations vary in how they feel about their position in the world. Some feel righteous and jingoistic, as if they were favored by God. They feel proud and entitled, and they assume those who are less fortunate are somehow lesser---perhaps more sinful, less disciplined, or more lazy---and therefore deserve their fate.

Others feel tremendous shame. There are a lot of people in the West with this shame gnawing away at them. They wish they could help those who are less fortunate, but then they just find themselves going right back to their high-consumption lifestyles. As a pennance, they donate a few bucks here and there to non-profit organizations who help those in need.

Social justice activists berate themselves and others for their various privileges. There is now a whole catalog of them. Aside from first-world privilege, there's also able-bodied privilege, class privilege, education privilege, male privilege, heterosexual privilege, white privilege, and more. According to this line of thinking, privilege runs in one direction and it's impossible to empathize with those without the privilege. All the privileged can do is plead mea culpa and give the underprivileged anything they ask. The privileged are not entitled to an opinion on the matter.

I too have an overly-developed sense of shame, but I strive to simply acknowledge these feelings and then let them go. I do this because I don't believe shame is a useful emotion. It can be useful in small doses, as it can shake people out of complacency and motivate them to help with important causes, but when it goes unchecked, shame can be a poisonous and even dangerous emotion.

Because it is so painful and creates cognitive dissonance, people resort to strange behaviors to alleviate it. Plus, that which is practiced within tends to be projected onto others. If people stew in shame and berate themselves for their privileges, they'll do the same to others. So they vie with each other for who is the greater victim, create strict rules about who is entitled to a voice (they call this ``checking your privilege''), or even become violent against those they feel should repent for their sins of privilege. Shame tends to lead to self-righteousness and hostility.\footnote{These problems are exacerbated when large sums of money are offered to victim classes for advocacy work. This allows people to make lucrative careers out of being professional victims. Then they're not just driven by shame, but also profit motive and job security. People are willing to go to extreme measures to protect their livelihoods, especially when they're otherwise unemployable. This is partly why I only give money to organizations that do tangible work with measurable results, not advocacy.}

However, there's a third emotional reaction available: \emph{compassion.} Compassion is one of the simplest and healthiest emotions we have, and it can be proactively nurtured. Rather than dividing ourselves into classes and then ranking each class by its privilege, compassion does the opposite. It focuses on our shared humanity. It's about the ways we are \emph{all} vulnerable to countless afflictions both internal and external. Compassion is the act of seeing in others the capacity for suffering, and wishing that they be healthy, happy, and safe from harm, just as we wish for ourselves. It is based on the assumption that it \emph{is} possible to empathize with others.

This ``yeah, but\ldots'' is based on the false assumption that compassion is a scarce resource. It's very common, but fallacious, to believe that we cannot and must not help any but the most absolutely abject suffering in the world---then, and only then, can we move onto others. This assumes that compassion spent is compassion lost. The reality is that compassion actually grows as we practice it. Having compassion for ourselves makes it easier, not harder, to have compassion for those who are less fortunate.

If we neglect compassion for ourselves out of shame, perhaps in the mistaken belief that this is an unnecessary and selfish privilege, then we actually hinder our capacity for compassion for others. We have to take care of ourselves \emph{before} we take care of others. Shame works against this process. Shame-based thoughts tell us that we don't deserve compassion because so many others are so much worse off than we are. It's a well-meaning emotion that actually makes it harder to help others.

So, I suggest that the goal should be to focus on freeing yourself first, even if, relatively speaking, you're already far more free than most people in the world or throughout history. Of course, you don't have to wait to have compassion for others as well. You can donate to those important causes right now, or you can wait until you're financially independent. Either way, financial independence will ultimately make you \emph{more} available, not less, to help those who are less fortunate, both with your time and your money. I talk about generosity more in Chapter 2.

\section{Yeah, but\ldots it's not practical for everyone}
A variant of this is, ``yeah, but\ldots you were lucky.'' I talked about the difference between creativity and luck in Chapter 1. This is a catch-all for many other ``yeah, buts.'' The argument is that it's not practical for everyone because they don't earn enough money, or they have families to support, or they aren't good enough at investing, or they have a poor education, or they lack job skills. I've lost count of all the reasons people give for why this can't work for everyone.

How bad does someone's situation need to be before it becomes impossible for them to achieve financial independence? At what specific threshold is it no longer possible for them? \$15,000 a year? \$10,000? Very little job skills? No job skills? Very little understanding of investing? No understanding of investing? How about someone who was a continual failure in high school, is in debt and struggling just to feed himself, with no job, no income, few skills, very little work experience, and no understanding of investing? That was me when I first started dreaming of retiring early.

I did have some advantages, and there are people in even harder situations than I was in. My point is, in a country like America\footnote{I am \emph{only} talking about America and other developed countries. As I explained in Chapter 1, it is not possible for most people in developing countries to become financially independent, due to limitations in their governments and economies.}, you can't tell what someone is capable of purely from their current situation. I was a pretty unlikely candidate for someone who would retire in his early 30's.

But, okay. Let's assume this was actually true, that there just are some people who are completely hopeless. No matter how much they dream, no matter how determined they are, they're just stuck. They will never achieve financial independence, ever. That's still not a good enough argument for not trying. The effort alone would still buy them more freedom. The smarter they are about money, the better off they will be. Every penny they earn, spend carefully, and invest properly, is several more pennies they will have when they lose their job, take a pay cut, or get hit with some catastrophe. Even if total financial independence is not practical for everyone, it is certainly practical in America for everyone to better their financial situation, and become \emph{more} financially independent than they are now.

\section{Yeah, but\ldots it's impossible}
This was probably the most frustrating of all the ``yeah, buts.'' ``It's impossible.'' No reason. No speculation about how one might actually become financially independent, or how to surmount some of the road blocks. Just immediately writing off the whole concept as ``impossible.'' This was probably also the most discouraging of the ``yeah, buts'' because it was so final, so certain. Any lingering doubt I still had become triggered.

Believe me, it's possible. Many people have done it. Not all have planned well enough to weather crashes in the economy. Some had catastrophes arise in their lives that required them to go back to work. Not all contingencies can be planned for, but it's possible to plan for a lot of them, and if you do, you can expect a long and glorious life of freedom.

Never trust any financial planner who tells you it's impossible to achieve financial independence. Ask for evidence. If they pull out Monte Carlo simulations that require a withdrawal rate less than 3\%, talk to someone else. There are a lot of incompetent financial planners out there, and even some of the competent ones are unreasonably conservative in their forecasts, so as to cover themselves in case things don't work out for you and you decide to come back and sue them. It's safer for them to tell you that it's impossible than it is for them to help you achieve your dreams.

And certainly don't trust anyone you meet who doesn't know much about investing, money, economics, or frugality. They have no expertise from which to draw to form such a conclusion. Most likely, they don't want to believe it's possible because if it were, they might feel bad about their own choices in life. I recommend you find friends who are more encouraging and supportive of your dreams.
