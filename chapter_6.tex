\chapter{Yeah, But\ldots}

\section{Excuses}
``Yeah, but\ldots'' are the most common two words I hear whenever I try to talk to people about financial independence. When I hear these words, I take it as a sign that they don't really want financial independence so much as to convince me or themselves that my situation is unique, or their situation is unique, so they can't achieve financial independence. If you're focused on the reasons you're unable to do something, then you'll be unable to do it. If you're determined, you'll find a way. If you're not determined, you'll find excuses.

If on some level you don't want financial independence bad enough, then you won't become financially independent. But if you really do want financial independence and you're trying to figure out how to address a concern you have about your own situation, then this chapter is for you. These are all the common ``yeah, buts'' I've heard.

Of course, it's up to you find solutions to your own problems, and the first step to doing that is to believe with all your heart that you will. My answer to your problem may sound insufficient or discouraging, but it doesn't mean you should give up. It only means there's no easy answer, so you'll need to use your determination and creativity to find your own answer.

\section{Yeah, but\ldots I have so much debt}
People who have a lot of debt feel very weighed down by it, so they feel like it's impossible for them to ever get out of the hole. The compound interest of debt has the effect of pulling people down further and further, even as they pay it down as fast as they can. It's like running a race in which you start far behind the starting line, with a head wind against you. You run as hard as you can, and you wonder if you'll ever make it to the starting line, let alone the finish line. Debt can feel very discouraging.

There is a concept in logic and cognitive psychology, called \emph{emotional reasoning.} It's common for people to assume that the way they feel is the way things really are. You may feel hopeless, so therefore you assume it is. It is not hopeless.

We all start somewhere. You have only one place you're capable of starting: right where you are. You can blame yourself for getting into debt, and you can wish you didn't have to start so far behind the finish line, but when it comes down to it, blaming and wishing won't \emph{change} where you are. You must start where you're at, for better or worse. The only way to change where you are is to start by accepting where you are. The only way to get out of debt is to pay it off. Wishing you didn't have to pay it off won't pay it off.

I too started out with some debt. One way I found to make it more encouraging was to treat it like investing. When I accepted where I was, I treated my interest payments as something like rent, just inevitable payments I must make to maintain my life. This way, any extra I pay is like an investment, which earns interest, by taking away from the payments that I'd otherwise have to make in order to maintain my life, at an interest rate equal to that of the loan I'm paying. For example, when I had a car loan at 8\% interest, I figured, what a great deal! Where else would I find a savings account that would earn me 8\%, risk-free? I excitedly invested in this savings account called my car loan.

Some people have a lot of debt. When you look at your amortization schedule (which I mentioned in Chapter 4), you may feel overwhelmed. You may find that it will take you many years to pay off your debt, so you may wonder why you should even bother. But those years will pass whether you pay off the debt or not. You can either stay in debt in those years, or you can get out of debt. Choosing to stay in debt doesn't make the debt go away. It only means you'll have to pay it later, after paying even more interest on it, or else it means making your loved ones pay it when you die and they inherit your estate. Somebody pays.

It is possible to achieve financial independence if you're in debt. It only means that it will take you longer. Debt doesn't disqualify you from financial independence any more than a low-paying job or high cost commitments. Even with debt, a low paying job, and high cost commitments, it is possible to achieve financial independence if you want it bad enough, if you work especially hard at the steps I outline in this book. Your biggest limit is your own devotion to your goal.

Anyone can retire. Indeed, \emph{everyone will.} The only question is when. Everyone stops working someday. Some stop working when they turn 65. Some stop earlier, others later. Some stop working on the day they die. The point isn't actually to retire. You're going to do that anyway. The point is to gain some control over your time line, and accelerate it as much as you can.

Even if you make your amortization schedule and you estimate that you won't be able to achieve financial independence for a long time, even if you do the steps in this book, that still doesn't mean that you can't achieve financial independence. It only means that it will take you longer than you would like. But your only other choice is to not do the steps in this book, in which case it will take you even longer. I know it's discouraging to have such a lousy choice, but no matter how long it will take you, doing the steps in this book will always allow you to achieve financial independence sooner than you would have otherwise, and the time you get back is your precious life energy.

The lessons in this book are really a no-lose proposition, because although there are trade-offs, this book doesn't create trade-offs. It only makes you conscious of them, so you can make the best trade-offs that will bring you the most joy in your life. Everything in this book should feel like a net gain.

\section{Yeah, but\ldots I don't earn enough money}
It's a common myth that you need to have a ton of money, and therefore need to earn a ton of money, in order to achieve financial independence. You do need a lot of money, but probably less than you think (see Chapter 5 to help you figure out exactly how much you'll need to earn). Financial independence is also a factor of how much you spend, and how you invest your savings, and that is under everyone's control, no matter how much they earn. Financial independence is the inevitable result of living beneath your means, and spending less than you earn. If you can do this, then you can be financially independent.

The only question is, is it worth it for you? Earning less money will have to mean spending less money, and it will take you longer. It will be harder work for you. It may not be worth all that work for you to be financially independent, but that's a different issue.

Additionally, it's also possible to earn more money. You could change careers, or advance in your existing career. Think about your skills and your gifts, and look for creative ways to put them to better use to earn you more money. This is the subject of Chapter 1.

\section{Yeah, but\ldots I have to support a family}
Freedom is even more important for families than for single people. A lot of families put the cart before the horse. They get so wrapped up in supporting the family financially that they miss the whole point. Do you spend as much time with your family as you or they would like? Would you like to learn to be a better parent, but can't find the time to read books or attend workshops? Many families are well-off, even wealthy, but they suffer from a severe time poverty. In other words, they make the wrong trade-offs. That's what financial independence is all about: finding the right trade-offs.

You may be surprised just how little money many families earn and spend in this country. Do some research on the median household income in America and in your own city, and see how you compare. Now consider that half of the population earns \emph{less} than that. There are families that live on very little, but still find a way to have a strong family.

Your expenses as a head of household will necessarily be higher than they would if you were single, but the problem is not fundamentally different. You need to question all of your choices just as you would if you were single. What is it that truly brings you and your family joy? Is it lots of stuff, or is it plenty of quality time with people you love? Even if you don't retire early, if you follow the program in this book, you will have less stress, more time, and more options. You may choose to cut back your hours, or change to a job or career that isn't so demanding, allowing you to spend more time with your family.

One advantage you have with a family you live with is that you can share a lot of costs. If you're single, you need to buy one of everything for yourself, and you need to buy in smaller portions. With a family, you have an economy of scale. You can share many of the things you buy, and you can buy more things in bulk. Make sure you leverage that advantage as much as possible. Besides, isn't sharing what love is all about?

For lots of tips for how to save money while strengthening your family, take a look at \emph{The Tightwad Gazette,} listed in the Resources. Amy Dacyczyn was so good at frugality that she was able to live in a large farmhouse with plenty of land, raise a very large, happy family, while spending far less than most single people do. She did work she loved and retired early. You may not choose to be as extreme as her, but it's possible, and her book will show you how.

\section{Yeah, but\ldots isn't it better to find work you love rather than waiting to retire?}
This is a false dichotomy. Who said you have to choose? Why not find work you love until you become financially independent? Then you'll have even more choices. There might be something you'd enjoy even more, but it doesn't pay well enough. When you're forced to work for a living, you're at the mercy of supply-and-demand, and the law of supply-and-demand can be pretty merciless. Everyone wants to do what they love, but there are only so many lovable jobs to go around, so most people end up doing work they don't much care for. Loving your job is something everyone seems to believe is possible, but few actually pull off.

So, yes, definitely look for work you love. Don't let financial independence get in the way of your search. If you can find work you love, then count yourself lucky. But if you don't find work you love, also remember that it is not a personal failure. So many career books make it sound like everyone can find a job they love, which leads people to wonder what's wrong with them if they can't. Nothing is wrong with you. It's simply a negative side effect of supply-and-demand. Work at the best job you can find, with the right trade-offs, and enjoy it as much as you can. Once you become financially independent, you'll have a lot more control over your life.

\section{Yeah, but\ldots I like my job!}
Then why not do more of it? If you get the most benefit from your life energy when you're at work, then the best thing you can do for your life is to work as much as possible. If you really like your job, but you only work the amount that is expected of you, while you have the option to work more hours, then are you really doing it because you like it, or are you just rationalizing? I have no personal stake in your answer. I only ask that you be honest with yourself. This really is your own life here. Be honest about how you want to spend that life, because you only get one of them.

Even if you do like your job, financial independence can make it even better. Financial independence gives you more flexibility and safety. You can get a better education to improve your job even more. You can weather lay-offs much better. And things change–you may burn out some day. You may decide you'd rather do something else. You'll know you're at your job for the right reasons. You're there because it enriches your life, not just your wallet.

Unless you plan to die at your workplace, you will retire some day. Why not choose for yourself when that will be, and do some planning? If you plan to retire at 65, why is that? Is that really when you want to retire, or are you just going along with the default? If you really like your job, why would you ever want to retire?

If you still insist that you like your job and that financial independence is not for you, then all I can suggest is that you put this book down and get back to work!

\section{Yeah, but\ldots what's wrong with working for a living?}
It depends on your motivation. Are you working because you want to, or because you \emph{have} to? If you're not financially independent, then you \emph{have} to, whether or not you \emph{want} to as well. If you cherish freedom, I think this would bother you.

Here's a weird analogy. Let's say you want to play a game of golf with your buddies. Then I put a gun to your head and say, ``play golf with your buddies, or die.'' Does that change anything for you? If you care about your freedom, I expect it would. It doesn't matter how much you'd want to do it anyway---this coercion will inevitably change that golf game for you. It would certainly make it much harder to enjoy the game. It also calls into question whether you actually wanted to in the first place. It gets really hard to be honest about your motivations as a free agent when you're no longer a free agent. If you looked at my gun to your head and say, ``uh, um, no big deal! I \emph{want} to play that game of golf!'' I wouldn't believe you, and you shouldn't believe you either.

I'm talking about freedom, not work. It's possible to work and be free, and it's possible to not work and not be free. If you choose to work, then work. But if you wouldn't choose to work if you didn't have to, or you wouldn't choose the specific kind of work you're doing now, then you're not free. If your livelihood is tied to your work, then your authentic motivations will always be distorted by your motivations for survival, so you can't honestly know what you \emph{really} want.

\section{Yeah, but\ldots what about \emph{right now}?}
One misunderstanding I heard when I told people about my goal of financial independence is that I am making huge sacrifices right now for some glorious future. What if I die before I retire? Won't I wish I lived it up now? What about all the things I'm giving up for this goal?

Every choice you make should be one of getting more out of life, not less. Not every choice needs to be getting the maximum joy out of life this very second. If it were, then there would be no reason to go to school or college. We make investments in our future all the time. The real issue they seem to have, then, is that the investment I'm making is above and beyond what's \emph{culturally acceptable,} ignoring what's \emph{personally acceptable} for me. The way I see it, I spent 8 years in college so I could have a better life afterward, and then I spent another 8 years in work so I could have a better life afterward. Why are the second 8 years so different from the first? It's not that I had to suffer and sacrifice during those 8 years in college, or the 8 in work. I enjoyed learning and contributing as much as I could, given my lack of freedom. I spent my free time doing exactly what I wanted to do, and I wasn't afraid to spend money if I knew it would bring me great joy.

Something the objectors didn't see was how much better my life already was than only a few years before, and this improvement was directly attributable to the choices I'd made to manage my money better and work toward financial independence. These choices didn't just benefit me toward my goal of retiring early---they had immediate benefits.

This goal didn't actually introduce any new sacrifices or trade-offs. People make trade-offs all the time, usually unconsciously. All this goal did was make me \emph{conscious} of my trade-offs, so I could make wiser trade-offs. Maybe people saw this as some great sacrifice because I was making different choices than they personally believed are necessary for happiness. I craved freedom, not jet skis, expensive cars, and fancy technology. My tastes simplified as a result of this goal, and therefore what made me happy had shifted from what made most other people happy. That's a healthy byproduct of working toward financial independence.

\section{Yeah, but\ldots won't you get bored?}
You know what I find strange is that no one ever thought to ask me that until I started working full-time. As a child, I had summers off. No one expected I would be bored if I didn't have a job. I did get bored sometimes, but that tended to be when I got tired of playing with one thing and was looking for the next fun thing to play with. What I really found boring was school and work.

Now that I'm retired, boredom looks a lot more like it did when I was a child. I become bored when I tire of one thing and I'm looking for something else. Boredom is a sign that my flame is dying and it's time to re-ignite it. When boredom comes, I try not to rush to fill the void. I find boredom and lethargy to be productive downtime. It's a time when I'm most open for inspiration. I'm ready to take a new path when it opens up for me. I might spend weeks or months just watching movies, reading books, and poking around on the web before I get excited about something. This is a cherished luxury that my retirement affords me.

What I did have to learn was to be self-directed again, after years of being told what to do. This can feel a little empty, and the longer you've worked the more empty it will feel. My advice is to embrace that emptiness. It's a kind of ``detoxification phase'' that you need to go through to adjust to a passion-based lifestyle after years of a paycheck-based lifestyle.

\section{Yeah, but\ldots aren't you just trading one job for another?}
One concern I hear a lot is that all this time I spend managing my money, researching investments, and cutting costs is a job in itself, so I'm not actually leaving work, just changing careers into a professional investor.

Money management is not something special that only exists when your goal is financial independence. It's something that everyone has to do, to varying degrees. They have to manage their accounts, balance their checkbook, set goals, pay bills, file taxes, and cut costs when times are hard. If they have a surplus, they have to learn about investing and research investment options. Some or all of this may be outsourced to professionals if one decides the cost of this is the right trade-off for them. The same is true for financial independence.

How much time or money you invest in money management is up to you. I have found that the time I spent on this paid itself back many times over. It also helps that I enjoy it immensely. For these reasons, I'm willing to invest heavily in managing my money. That's a choice I've made.

That said, I don't really spend a lot of time on it. Most of my time investment was up-front, learning about financial independence, investing, taxes, and frugality. I set up a target asset allocation and I buy and hold my investments. I keep an eye on the economy as it ebbs and flows, but I don't really do much about it. Each night, I spend a couple minutes updating my budget with my expenses for the day. Each month, I add everything up, balance my checkbook, and pay my bills. At the end of the year, I do my taxes. Every year or two, I re-group, do a bit more research, and re-balance my portfolio. Other than that, my money pretty much runs on auto-pilot.

On the other hand, I no longer have a ``second shift'' that comes with working. I don't have to shop for work clothes, prepare for the next day's meeting, commute back and forth, update my resume, hunt for jobs, read trade journals, or unwind from the work day. When all is said and done, I'm pretty sure I've come out ahead.

\section{Yeah, but\ldots what if you run out of money?}
A variant of this is, ``yeah, but... what if some catastrophe happens (and you run out of money)?'' This risk is not unique to early retirement. Everyone faces this risk. Having a job doesn't mean you can never run out of money. It happens to people all the time. When it does, they do what they have to do. They take a second job. They take out loans. They find some way to survive.

Retirement is unique in that being out of the workforce for an extended time makes it difficult to jump right back into your old career. This is a risk all retirees face, not just early retirees. It's why good planning is so important. It's usually avoidable if you've planned well enough, added sufficient buffer for emergencies, and were careful enough in your estimates. Even if it does happen, you find some way to make it work. If all else fails, you find some kind of job to pay the bills, even if it's not in your old career.

Suppose the catastrophe not only causes you to run out of money, but also physically incapacitates you, so that you're incapable of going back to work. Again, this is not a risk unique to retirement. This can also happen if you're working. In fact, it's more likely to happen if you're working, because you're driving more often in order to commute to work and therefore more likely to get into an auto accident, and because many physical disabilities are caused by accidents at work. Some jobs offer disability insurance to protect against this, but not all.

What's more interesting are all the questions I \emph{don't} hear as an early retiree. I never hear, ``yeah, but... what if you lose your job?'' I never hear, ``yeah, but... what if you end up stuck in a job you hate, and can't leave because the economy is in shambles?'' I never hear, ``yeah, but... what if you realize that your career choice wasn't a good fit for you?'' I never hear, ``yeah, but... what if your skills become obsolete?'' These are all risks people face by being in the workforce, which go away when you retire. Going back to work is a contingency plan for retirees, but not for workers, because they are already working. I believe financial independence eliminates more risks than it introduces.

\section{Yeah, but\ldots what about health insurance?}
Health insurance sucks. It's probably one of the biggest road blocks to financial independence, but as with any road block, it is surmountable if you want it bad enough. Individual plans are expensive. This means you will need to save more in order to pay for it. Talk to an insurance broker to see how much you can expect to pay, and plan for that in your withdrawal rate calculations.

People who are financially independent can have health insurance, just like everyone else. They just have to pay for it themselves. Actually, working folk have to pay for it too, but their payments are hidden from them by their employers. People who are financially independent also have a paycheck, just like everyone else, measured by their withdrawal rate. They even get raises! Read Chapter 5 for an explanation of that perspective. I discuss health insurance at length in Chapter 5, and what special adjustments you need to make to accommodate some of the quirks with health insurance.

\section{Yeah, but\ldots what about health costs when you age?}
Health care costs go up as we age, it's true. How do you handle that? The same way you handle everything else: by planning for it. I discussed how to do this in Chapter 5. There is Medicare insurance and Medigap insurance. I talked about how you can plan for \emph{age inflation} and \emph{health insurance insurance.} This planning isn't much different for younger people than for older people. If you retire at 65, how will you know what your health costs will be when you're 85? You really can't. You just have to use what you do know and make predictions, with some buffer in your planning for unforeseeable circumstances. Nevertheless, most retirement planners actually recommend that people assume they will be spending less in the future than they are now, not more, because overall, that is the trend that usually happens.

Youth offers two advantages for health care planning that older retirees lack. One advantage is more time to plan, prepare, and save. It's far easier to prepare for aging when you're young than when you're already old. Another advantage, another form of preparation, really, is that you can take good care of your health while you still have it. Now is the time to quit smoking, cut back on alcohol, exercise more, and eat healthy. These small changes in habits pay huge dividends in the future, and it's easier to make these changes when you don't have spend so much stress and time on working.

\section{Yeah, but\ldots I'm scared of investing}
Investing isn't as scary as most people think. It's just a matter of taking the time to learn about it and executing on an investment strategy. It requires only basic math, and there are many calculators available to help with that. Some people believe only masters can invest well. These masters are human just like everyone else, but their pride leads them to make big mistakes. All the big crashes in history happened because of these big players making big mistakes. Many smaller investors can do quite well by buying a modest portfolio of stocks and bonds, and just sticking with it. Chapter 3 describes several easy ways to do this.

\section{Yeah, but\ldots the stock market is too risky}
This is usually emotional reasoning: it feels risky, so therefore it is risky. People who feel this way are usually ignorant of the historical reality of the stock market. They remember the big crashes, or they see it go up one day and down the next. They overlook the usual behavior and long-term trends. Most people who feel the stock market is too risky are surprised when I tell them that the stock market has gone up over every 10-year period in the history of the stock market–including the crash of 1929 and the Great Depression.

It may help to remember what is \emph{constant} about the stock market, invariants that you can always count on. Think of these as the two Laws of the Stock Market. The First Law is that it goes up \emph{and} down. It does this every day, but also over longer time periods. This may seem trite, but it's so easy to forget. The stock market goes up for a while, and bubbles form because people think there's no end in sight. Then it goes down for a while, and we go into a protracted recession due to excessive fear. Even the most sophisticated market professionals, who've been doing this for decades, forget this.

The Second Law of the Stock Market is that it \emph{will} crash, at least once per generation. That means you can plan on being treated to at least one of these crashes in your lifetime, possibly several, and they can't be timed. That was the mistake I made. I figured I survived one of the biggest crashes in history, so I was safe. Then another crash happened eight years later, just as bad. These crashes are scary as hell. Imagine working hard for years, or even decades, to save up \$200,000. Then, within a couple weeks, your hard-earned savings have been demolished to \$100,000, and there's no sign of it stopping. At that rate, you figure you'll be totally broke within a month unless you sell it all, \emph{now.} This is the worst mistake you can make, and those who make it usually do so because they've forgotten the Second Law, so they panic. You must be prepared for this inevitability. When you look at your portfolio balance, mentally discount it to account for the Second Law. Plan contingencies for how you will behave \emph{when, not if} it crashes.

When it does crash, remember the First Law. It always goes down \emph{and} up. These crashes afford some tremendous buying opportunities. After a severe crash, I bought a couple of stocks of solid companies that had been hit so hard that they were going for less than their book value. In other words, if these companies went out of business the next day and sold all their assets, they'd have more in cash than what the stock was going for. These two stocks were scary to buy, because it felt all the stock market ever does is crash, but I knew that they were good values. I sold them nine months later. One of them doubled in value and the other tripled.

The biggest risks you face aren't the stock market, but your own emotions and ignorance. You're your own worst enemy. Your biggest defense is to really know the risks going in, and how much risk tolerance you have. Investment risks can be managed without too much work. If you're scared of investing because of the risks, then just know that you have a low risk tolerance. There's nothing wrong with that. When you talk to a financial planner, or when you create your own investment strategy, make sure you account for your low risk tolerance.

\section{Yeah, but\ldots it's too hard}
Financial independence does require a lot of work, research, and introspection. Some of what you've read here may intimidate you. Maybe some of it went over your head, so it sounded like too much work. Only you can decide if it's worth it for you. However, remember that I didn't learn all of this overnight. I spent a lot of time excitedly reading, thinking, researching, and calculating. I started out knowing very little, so if you're just starting out and you know very little, you're just where you need to be. Don't feel like you need to understand everything in this book right away. Just get started, and check out some of the resources I recommend.

Also, notice that I said the reading, thinking, researching, and calculating I did was done ``excitedly.'' That's because I knew that this was allowing me to achieve a freedom I craved very deeply. I didn't do this work out of duty or self-discipline. I did it because I wanted to know everything I possibly could to reach my goal, and to discover every contingency I might need to plan for, so I could ensure keeping the freedom I'd worked so hard for.

\section{Yeah, but\ldots it takes too much discipline}
One reaction I often get when I tell people I retired early is that they admire my self-discipline. While I appreciate the compliment, it feels misplaced. These are people who get up at the crack of dawn every morning, rain or shine, no matter how they're feeling, battle traffic, spend half their day toiling at a job that most of them don't find very gratifying or enjoyable, battle traffic again to get back home, and finally collapse on the couch from exhaustion. They're so disciplined that even the amount they're allowed to be sick is often regulated.

By contrast, I sleep as late as I want, and do literally whatever I want every day. I'm usually incredibly productive, but only when I'm working on an exciting project, which is most of the time. That's the case because I always get to choose exactly which project I'm going to work on next. But some days, I'm so lazy that all I do is lay around and read books or watch movies. So it sounds very strange indeed to hear these masters of self-discipline call me the disciplined one---even that they \emph{admire} me for this.

I didn't achieve financial independence because I'm disciplined. I did it because I \emph{lack} discipline. I always have. It was the perennial problem that my parents tried and failed to remedy. But what I lack in discipline, I more than make up for in passion, determination, commitment, awareness. Maybe even obsession. I did have a long-term perspective, and some may say that is what discipline is, but there's a key difference. Discipline is about self-denial, whereas my approach was about abundance. Discipline is the ability to make yourself do things that you don't feel like doing. I didn't need that ability, because I \emph{always} felt like working toward freedom. It was all I could think about. The way I saw it, the only two choices I had was to work and not retire early, or work and retire early. What required all of my willpower–work–was necessary in both cases. I simply chose the path of abundance because, well, why wouldn't I?

When I became determined to find freedom in my life, and I fully appreciated the trade-offs I was making by spending wastefully and managing my money poorly, I began to have an adverse reaction to buying things I knew would have very little impact on my overall quality of life. I became agitated at the thought of having money sitting around that could easily be put to work behalf of my freedom. I became so hyper-aware of the trade-offs that the truly disciplined thing would have been for me to waste my money.

If you find yourself struggling to muster the self-discipline to achieve financial independence, then maybe you're going about it wrong. It requires awareness of trade-offs far more than discipline. I can't imagine this goal being accomplished by sheer willpower alone. While it certainly requires effort, it should not be the kind of effort it takes to do chores. If you're not thrilled by the project of financial independence, if you're not excited at the thought of spending your life exactly the way you choose to spend it, then it's probably not all that important to you, and you should focus your precious, limited life resources on things that truly bring you joy.

\section{Yeah, but\ldots freedom is dangerous}
Some people fear too much freedom. ``Idle hands are the devil's workshop.'' They're afraid that too much freedom will cause people to do destructive things. They'll get bored and then start living a life of crime, or become addicted to drugs, or party every night. They look at the lifestyles of rock stars and they shudder at the dangers of having too much freedom.

This is actually very similar to an argument made by anti-abolitionists in the 1800's. Blacks were believed to be dangerous and stupid, not to be trusted with freedom. Best to keep them safely chained up and working for the rich white man, who was, somehow, considered perfectly capable of handling their freedom. Now we understand how flawed and hateful this reasoning was, but we also see how irrelevant the argument was in the first place. People don't deserve freedom because it's safe. They deserve freedom because it's a fundamental human right. Just because some people can choose things that will hurt them doesn't mean they don't deserve freedom.

And it's true. Some people will make poor choices with their freedom, but that's the point of freedom. It's the ability to make choices, both good and bad. Making choices sometimes means making mistakes, even deadly ones.

If it's idleness that worries you, then that's easy. Don't be idle. Financial independence isn't idleness. Financial independence is freedom. It will always give you more choices, not less. You can choose to not be idle.

Financial independence itself is a choice. If you know you can't handle freedom, then choose not to be free. For that matter, if you really don't think you could handle that freedom, then you should also be concerned whether you can handle dangerous ideas agitating for freedom, such as this book, so it's probably best that you put this book down now and get back to work.

\section{Yeah, but\ldots it's not practical for everyone}
A variant of this is, ``yeah, but\ldots you were lucky.'' I talked about the difference between creativity and luck in Chapter 1. This is a catch-all for the other ``yeah, buts.'' The argument is that it's not practical for everyone because they don't earn enough money, or they have families to support, or they aren't good enough at investing, or because they have a poor education, or because they lack job skills. I've lost count of all the reasons people give for why this can't work for everyone.

How bad does someone's situation need to be before it becomes impossible for them to achieve financial independence? At what specific threshold is it no longer possible for them? \$15,000 a year? \$10,000? Very little job skills? No job skills? Very little understanding of investing? No understanding of investing? How about someone who was a nearly continual failure in high school, that is struggling just to feed himself, with no job, no income, few skills, very little work experience, and no understanding of investing? That was me when I first started dreaming of retiring early.

I did have some advantages, and there are people in even harder situations than I was in. My point is, in a country like America, you can't tell what someone is capable of purely from their current situation. I was a pretty unlikely candidate for someone who would retire only 14 years later.

But, okay. Let's assume this was actually true, that there just are some people that are completely hopeless. No matter how much they dream, no matter how determined they are, they're just stuck. They will never achieve financial independence, ever. That's still not a good enough argument for not trying. The effort alone would still buy them more freedom. The smarter they are about money, the better off they will be. Every penny they earn, spend carefully, and invest properly, is several more pennies they will have when they lose their job, take a pay cut, or get hit with some catastrophe. Even if total financial independence is not practical for everyone, it is certainly practical in America for everyone to better their financial situation, and become \emph{more} financially independent than they are now.

\section{Yeah, but\ldots it's impossible}
This was probably the most frustrating of all the ``yeah, buts.'' ``It's impossible.'' No reason. No speculation about how one might actually become financially independent. Just immediately writing off the whole concept as ``impossible.'' This was probably also the most discouraging of the ``yeah, buts'' because it was so final, so certain. Any lingering doubt I still had became triggered.

Believe me, it's possible. Many people have done it. Not all have planned well enough to weather crashes in the economy. Some had catastrophes arise in their lives that required them to go back to work. Not all contingencies can be planned for, but it's possible to plan for a lot of them, and if you do, you can expect a long and glorious life of freedom.

Never trust any financial planner who tells you it's impossible to achieve financial independence. Ask for evidence. If they pull out Monte Carlo simulations that require a withdrawal rate less than 3\% or 4\%, talk to someone else. There are a lot of incompetent financial planners out there, and even some of the competent ones are unreasonably conservative in their forecasts, so as to cover themselves in case things don't work out for you and you decide to come back and sue them. It's safer for them to tell you that it's impossible than it is for them to help you reach your dreams.

And certainly don't trust anyone you meet who doesn't know much about investing, money, economics, or frugality. They have no expertise from which to draw to form such a conclusion. Most likely, they don't want to believe it's possible, because if it were, they might feel bad about their own choices in life. I recommend you find friends who are more encouraging and supportive of your dreams.
