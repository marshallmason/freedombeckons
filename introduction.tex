\chapter{Introduction}
Americans hate working more than ever before in history. According to a survey conducted by The Conference Board, a consumer confidence research center, less than half of American workers are satisfied with their jobs. This is down by 22\% since the 1970's.\cite{livescience}
Since we dislike working so much, you'd think we would do less of it, but America is one of the hardest working nations in the world. Americans work an average of 1,815 hours per year, the fourth highest among developed nations. This is up by 12\% since the 1970's.\cite{anderson-cnn} This means most Americans spend 85,305 hours of their lives doing something most of them don't much like. That's nearly a quarter of their waking, adult lives. And it's getting worse.

The good news is that Americans have a lot more money to spend on things that make them happy and improve their quality of life. Americans spent \$10.21 trillion in 2009,3 up from \$879.3 billion in 1973, adjusted for inflation.\cite{american-consumer-spending} That's over 11 times more spending!

The bad news is that all of this additional money isn't actually leading to more happiness. If it were, you'd expect people to be 11 times happier now than they were in 1973. But according to a Gallup poll, the percentage of Americans who reported being ``very happy'' with their lives was 38\% in 2001, down slightly from 45\% in 1973.\cite{1973-consumer-spending} We could philosophize endlessly about what leads to happiness and unhappiness, but these numbers do not show a positive correlation between spending and happiness.

We've all heard the phrase, ``money can't buy happiness,'' but the truth is, we do feel happier when we buy something we want. That's why we buy them. Our first few purchases in life are particularly gratifying. We experience firsthand this connection between money and happiness. Since we naturally think in linear terms, we assume that we just need to keep spending more and more and we'll be happier and happier, and marketing feeds this myth.

But it doesn't work that way. It tapers off at some point. The happiness we bought doesn't seem to last as long as it used to. Like drug addicts craving our next fix, we're back to buying something else that will make us happy. We also have to maintain all that stuff we bought. We have to clean it, fix it, insure it, find a place to keep it, and when we're all done with it, get rid of it. As we accumulate more, the time we spend and the stress we feel maintaining our stuff goes up. There is a problem of diminishing returns with spending. I believe the trend reverses when spending gets too high, and we become less satisfied with our lives as we spend more.

Although the connection between spending and happiness breaks down, the connection between earning and spending gets even stronger. Every penny we spend, we must earn. In fact, we must earn even more, due to taxes. If you enjoy working, then this connection will be a positive one, but this isn't true for most Americans. If you don't enjoy working---or even if you do enjoy working, but would be happier if you did less of it---then spending will make you less happy.

The more we spend, the more dependent we become on a source of income that we don't enjoy, and the less freedom we have to choose our lives. Our jobs dictate how we dress, where we live, and how we spend our time. Some jobs demand even more. If you live for your job, you won't mind losing this freedom, but most people work out of necessity, and many of them feel like slaves. This is a problem in a country whose most celebrated value is freedom.

\section{What is freedom?}
The word \emph{freedom} is thrown around so indiscriminately that it's become a political buzzword. Some people use this word when the exact opposite is implied. They talk about freedom and then try to push their own agendas. Freedom means a lot of things, and each person has a different idea of what it means for them. How freedom manifests in your life is something you get to decide. That's the whole point of it. Freedom is \emph{the ability to make choices.} It can be taken away when someone limits your choices. The most common understanding is that this someone is another person. This book focuses on the times when that someone is you.

One way this can happen is if you're ignorant of your choices. You're free only insofar as you know all of your options. Any options you're unaware of won't be factored into your choices, so you won't be free to choose them. This is very common, but it can be remedied by learning the facts. Research can open up many options you didn't know existed. In this book, I provide as many options as possible. Some of them may not be relevant for you, so feel free to skip them. However, this book isn't meant to be an encyclopedia, only a guide for your research. I've provided a Resources section for each chapter, of books and websites that I think are important for further study.

Another way you can limit your choices is to block some of your options from awareness. This can happen when some emotion arises whenever you consider particular options, so you overlook them in order to avoid feeling that emotion. There are two emotions in particular that are very relevant for the purposes of financial independence: guilt and fear. I will remind you throughout this book to consider the influence of these emotions on your choices. Another way it can happen is when a specific option has hijacked your awareness. You become compulsive, get a kind of tunnel vision, and all you can see is one option, pushing out all other options from your awareness.

Being ignorant or unaware of other options isn't a problem if all of your options were independent from one another. It only becomes a problem when choosing one options limits your other options. If this happens, then you're limiting your options by choosing it. These are called \emph{trade-offs.} These arise frequently, due to limited resources. They're not a problem, as long as you're aware of them, but when you're unaware of them, trade-offs limit your freedom in the worst way possible. Often, we overlook the reality that, when we choose one thing, we may be simultaneously choosing \emph{against} other things. This will be a common theme throughout this book: always know the trade-offs, and then make the right choice for you.

\section{Financial independence}
Financial independence is the \emph{freedom to choose how you spend your life,} rather than having your choices limited by financial necessity. This is often confused with \emph{independently wealthy,} which includes additional freedoms and luxuries. This book is not designed to help you become independently wealthy, and it definitely won't show you how to get rich quick. Financial independence takes a lot of time, patience, and perseverance. In this book, I push for a time line on the order of 10--15 years.

Financial independence may or may not include retirement. All retired people are financially independent, but not all of those who are financially independent are retired. When you're financially independent, work become a choice. Some people enjoy working, particularly when they know it's optional.

I define retirement as \emph{permanently severing the connection between labor and income.} This definition allows room for retirement at any age. Some people call themselves retired just because they're old and no longer active in their main career, even though they still work. This connotation of retirement is strictly about old age, since young people who do this aren't called retired. I've also known people who take a few years off and called this retirement. This isn't retirement, but \emph{unemployment} or \emph{sabbatical.}

Being financially independent means you can earn a living from sources other than your labor, and not being dependent on anyone else for your income. It means being able to choose how you spend your time, rather than having your financial requirements choose it for you. It means having your money work for you, rather than the other way around.

This book focuses on financial independence because this book is about freedom, and I believe the ability to choose how you spend your life is the most fundamental freedom of all. I'm not here to tell you what you should believe, how you should spend your money, or how you should invest it. I cannot tell you what trade-offs are right for you. I only try to highlight oft-overlooked trade-offs, and urge you to stay aware of them. I will give you some ideas to start with, and I will spell out the choices I made for myself. I don't expect you to value the same things I do, or go about it the same way I did.

There is one exception: for this to work, you must value freedom. If you don't crave more freedom in your life, then you will not make the trade-offs necessary to achieve financial independence. I've met many people who say financial independence would be nice, but won't entertain even the simplest trade-offs necessary to achieve it. Of course it would be nice. Lots of things are nice when they come for free. Or they may say they want to be financially independent when what they really want is to become independently wealthy. They crave money and the things it can buy, not freedom.

If you don't crave freedom, this book may still be worth reading, as you might gain some interesting tips on frugality and money management. This is a fortuitous side-effect of this book. If you decide you're not really interested in financial independence, or that the trade-offs necessary to achieve it aren't worth it for you, that's fine. But know that the real objective of this book is teach you how to achieve financial independence.

\section{How this book is organized}
Financial independence requires three financial components:
\begin{itemize}
\item Income (Chapter 1)
\item Frugality (Chapter 2)
\item Investing (Chapter 3)
\end{itemize}

Think of this as the financial independence formula.

When I decided to become financial independent, I had no steady income. Frugality only slowed the rate that I went into debt, and it didn't matter how good I was at investing because I had nothing to invest. I had to focus on getting my career going first. Many people have a decent income and a good investment strategy, but they spend a lot of money, and retire comfortably at an old age. Others have a good income and are very frugal, but they don't invest that money properly. They either invest in something too risky and lose it all, or they invest in something too safe and watch inflation chew away its value.

But put all three together, and you have an amazing, even magical formula for financial independence. I say magical because it astonishes many people when they see the numbers for the first time. We naturally think in linear terms. When you put these three components together, each of them compliments the others, and the savings curve transforms from linear to exponential. When managed properly, money has the capacity to imperceptibly add up, from small amounts to enormous amounts.

I've organized this book so you can get the basic formula right way. If you really want to know how I retired early, the first three chapters are what you need to read. The rest of the book fills in a lot of the blanks left by the first three chapters.

Chapter 4 talks about the records you should keep, and how to use them to track your progress.

Chapter 5 discusses some of the things the workplace does for us that we need to do for ourselves in order to be financially independent.

Two words I hear a lot when I tell people about financial independence is, ``yeah, but\ldots'' These are reasons I've heard people give for why they can't achieve financial independence. Chapter 6 refutes all the ``yeah, but\ldots'' challenges I've heard over the years. Think of this as a ``frequently asked questions'' section.

My goal isn't just to teach people how to achieve financial independence, but how to find more freedom, for themselves and the rest of the world. I want this book to stand as a testament to freedom. In Chapter 7, I discuss freedom at length, what it means for different people, and what we can do to find more freedom in our lives, in ways other than financial independence.

\newpage
\section{Resources}
\begin{itemize}
\item \textbf{\emph{Your Money or Your Life: 9 Steps to Transforming Your Relationship With Money and Achieving Financial Independence} by Vicki Robin, Joe Dominguez, and Monique Tilford.} This is the granddaddy of financial independence books. This was the one that got me started. Although some of the details of the program in this book are problematic, I still consider this required reading for anyone who wants to achieve financial independence.

\item \textbf{\emph{Work Less, Live More: The New Way to Retire Early} by Bob Clyatt.} This is the best book I've found for financial independence, next to Your Money or Your Life. It fills in a lot of holes in Your Money or Your Life, such as the withdrawal rate, investing, health care, and taxes. It's also updated more frequently. This too I consider required reading.

\item \textbf{\emph{Awaken the Giant Within: How to Take Immediate Control of Your Mental, Emotional, Physical and Financial Destiny!} by Anthony Robbins.} Financial independence requires a certain attitude that is not very common. It requires a lot of creativity, determination, and awareness. These came naturally for me from my rebellious childhood and my interest in philosophy, but I later found most of what I'd discovered on my own in this book by Tony Robbins. If you can get past his excessive use of exclamation points and his tendency to commoditize common philosophical and psychological concepts, I would definitely recommend this book.

\item \textbf{\url{http://www.yourmoneyoryourlife.org/}} -- This is the website for \emph{Your Money or Your Life.}
\end{itemize}
