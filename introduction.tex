\chapter*{Introduction}
\addcontentsline{toc}{chapter}{Introduction}
\fancyhead[CO]{\emph{Introduction}}
\fancyhead[CE]{Freedom Beckons}
Americans hate working more than ever before in history. According to a survey conducted by The Conference Board, a consumer confidence research center, less than half of American workers are satisfied with their jobs. This is down by 22\% since the 1970's.\cite{livescience}

Since we dislike working so much, you'd think we would do less of it, but America is one of the hardest working nations in the world. Americans work an average of 1,815 hours per year, the fourth highest among developed nations. This is up by 12\% since the 1970's.\cite{anderson-cnn} This means most Americans spend 85,305 hours of their lives doing something most of them don't much like. That's nearly a quarter of their waking, adult lives. And it's getting worse.

The good news is that Americans have more money to spend on things that make them happy and improve their quality of life. The average American household expenditures in 2012--2014 was \$53,495\cite{bls-spending-2014}, compared to \$44,511 in 1973\cite{bls-spending-1973}, adjusted for inflation. That's 20\% more spending.

We've all heard the phrase, ``money can't buy happiness,'' but the truth is, we do feel happier when we buy something we want. That's why we buy it. Our first few purchases in life are particularly gratifying. We experience firsthand this connection between money and happiness. Since we naturally think in linear terms, we assume that we just need to keep spending more and more and we'll be happier and happier, while marketing feeds this myth.

But it doesn't necessarily work that way. It tapers off at some point. The happiness we bought doesn't seem to last as long as it used to. Like drug addicts craving our next fix, we're back to buying something else that will make us happy. We also have to maintain all that stuff we bought. We have to clean it, fix it, insure it, find a place to keep it, and when we're all done with it, get rid of it. As we accumulate more, the time we spend and the stress we feel maintaining our stuff goes up. There is a problem of diminishing returns with spending.\footnote{As with most things, it's not quite this simple. Different types of spending diminish at different rates. Happiness gained by buying material possessions diminishes quickly, but research has found that buying experiences, giving to charity, and buying time as I advocate here, can bring further happiness.\cite{wsj-can-money-buy-happiness}}

Although the connection between spending and happiness breaks down, the connection between earning and spending gets even stronger. Every penny we spend, we must earn. In fact, we must earn even more, due to taxes. If you enjoy working, then this connection will be a positive one, but this isn't true for most Americans. If you don't enjoy working---or even if you do enjoy working, but would be happier if you did less of it---then spending will make you less happy.

The more we spend, the more dependent we become on a source of income that we don't enjoy, and the less freedom we have to choose our lives. Our jobs dictate how we dress, where we live, and how we spend much of our time. Some jobs demand even more. If you live for your job, you won't mind losing this freedom, but most people work out of necessity, and many of them feel like slaves. This is a problem in a country whose most celebrated value is freedom.

\section{What is freedom?}
The word \emph{freedom} is thrown around so indiscriminately that it's become a political buzzword. Some people use this word when the exact opposite is implied. They talk about freedom and then try to force their own agendas. Freedom means a lot of things, and each person has a different idea of what it means for them. How freedom manifests in your life is something only you can decide. That's the whole point of it. Freedom is \emph{the ability to make choices.} It can be taken away when someone limits your choices. The most common understanding is that this someone is another person. This book focuses on the times when this someone is you.

One way this can happen is if you're ignorant of your choices. You're free only insofar as you know all of your options. Any options you're unaware of won't be factored into your choices, so you won't be free to choose them. This is very common, but it can be remedied by learning the facts. Research can open up many options you didn't know existed. In this book, I provide as many options as possible. Some of them may not be relevant for you, so feel free to skip them.

Another way you can limit your choices is to block some of your options from awareness. This can happen if some emotion arises whenever you consider particular options, so you overlook them in order to avoid feeling that emotion. There are some emotions in particular that are very relevant for the purposes of financial independence: \emph{guilt, shame} and \emph{fear.} I will remind you in this book to consider the influence of these emotions on your choices. Another way it can happen is when a specific option has hijacked your awareness. You become compulsive, get a kind of tunnel vision, and all you can see is one option, pushing out all other options from your awareness.

Being ignorant or unaware of other options wouldn't be as much of a problem if all of your options were independent from one another. It becomes a real problem when choosing one option limits your other options. If this happens, then you're limiting many of your other options by choosing it. These are called \emph{trade-offs.} They arise frequently, due to limited resources. They're just a part of life, so they aren't a problem as long as you're aware of them, but when you're unaware of them, trade-offs limit your freedom in the worst way possible. Often, we overlook the reality that, when we choose one thing, we may be simultaneously choosing \emph{against} other things. This will be a common theme throughout this book: always know the trade-offs, and then make the right choice for you.

I've found freedom to be one of the more contentious concepts, partly because it can mean so many things to different people, and partly because it's so political. For this reason, at the end of the book I address some of the specific issues I've heard people raise about freedom and I go in-depth about my philosophies about freedom. For now, consider the idea very broadly as \emph{the ability to make choices.}

\section{Financial independence}
Financial independence is the \emph{freedom to choose how you spend your life,} rather than having your choices limited by financial necessity. This is often confused with \emph{independently wealthy,} which includes additional freedoms and luxuries. This book is not designed to help you become independently wealthy, and it definitely won't show you how to get rich quick. Financial independence takes a lot of time, patience, and perseverance. In this book, I push for a time line on the order of 10--15 years.

Financial independence may or may not include retirement. All retired people are financially independent, but not all of those who are financially independent are retired. When you're financially independent, work becomes a choice. Some people enjoy working, particularly when they know it's optional.

I define retirement as \emph{permanently severing the connection between labor and income.} This definition may seem overly rigid, as many have a looser definition of retirement. I prefer my stricter definition because it eliminates ambiguity and clears up common connotations. Some people call themselves retired just because they're old and no longer active in their main career, even though they still work. This connotation is strictly about old age, since young people who do this aren't called retired. I've also known people who take a few years off and called this retirement. This isn't retirement, but \emph{unemployment} or \emph{sabbatical.}

Being financially independent means you can earn a living from sources other than your labor, not being dependent on anyone else for your income. It means being able to choose how you spend your time, rather than having your financial requirements choose it for you. It means having your money work for you, rather than the other way around.

This book focuses on financial independence because it is about freedom, and I believe the ability to choose how you spend your life is the most fundamental freedom of all. Although it does address some issues specific to it, \emph{this book is not about retirement.}

The only difference between financial independence and retirement is \emph{what you choose to do with your newfound freedom.} If you decide to leave work entirely and completely, then you've retired as I define it here. This book isn't about what you should do with your freedom, only about how to achieve it, which is the same in both cases. Financial independence and retirement are equivalent in the sense that most of the same financial considerations need to be made for both goals.

Whenever I speak of financial independence, you can think of that as code for early retirement, if that is your goal. If that is not your goal, I assure you that financial independence does not necessarily mean early retirement, even though it did for me. Since so much of this book is personal, I often speak of my retirement because that's what financial independence looked like for me.

I'm not here to tell you what you should believe, how you should spend your money, or how you should invest it. I cannot tell you what trade-offs are right for you. I only try to highlight oft-overlooked trade-offs, and urge you to stay aware of them. I will give you some ideas to start with, and I will spell out the choices I've made for myself. I don't expect you to value the same things I do, or go about it the same way I did.

There is one exception: for this to work, you must value freedom. If you don't crave more freedom in your life, then you will not make the trade-offs necessary to achieve financial independence. I've met many people who say financial independence would be nice, but won't entertain even the simplest trade-offs necessary to achieve it. Of course it would be nice. Lots of things are nice when they come for free. Or they might say they want to be financially independent when what they really want is to be independently wealthy. They crave money and the things it can buy, not freedom in the sense that I mean.

If you don't crave freedom, this book may still be worth reading, as you might gain some interesting tips on frugality and money management. This is a fortuitous side-effect of this book. If you decide you're not really interested in financial independence, or that the trade-offs necessary to achieve it aren't worth it for you, that's fine. Just know that you are not my target audience. The real objective of this book is to teach you how to achieve financial independence.

\section{How this book is designed}
This book is small and extremely limited in scope. It is thorough only in breadth, not depth. It mentions every concept I needed to achieve financial independence, but doesn't go into much detail on the concepts that can easily be learned from other sources. I don't want to invest my time writing about things that other authors have already covered much better than I can hope to. For this, I simply refer to them in the Resources sections. However, there were many things I discovered and had to learn to do myself in my quest for financial independence. That's really what this book is for.

I try to keep this book personal as much as possible. I believe personal stories are the easiest way to relate my dream of financial independence to others. People like stories. They can just listen without feeling like they should do or agree with anything. They might see themselves in these stories, but they can easily overlook the parts they don't relate to because they know it's not their story.

I've found many books that covered specific concepts well, such as job hunting, frugality, and investing. I've found lots of articles about inflation and retirement withdrawals. What was missing was a book that tied it all together into a complete story of someone who actually did what I was trying to do, with their struggles and triumphs along the way.

This is a book that lightly touches on everything: income, frugality, investing, withdrawals, taxes, record-keeping, time management, and philosophy, uniting them into a single, coherent picture. It's really only a launching point for more learning, so readers can pick and choose what to learn more about, and when. It is not a complete encyclopedia.

\section{How this book is organized}
Financial independence requires three financial components:
\begin{itemize}
\item Income (Chapter 1)
\item Frugality (Chapter 2)
\item Investing (Chapter 3)
\end{itemize}

Think of this as the financial independence formula.

When I decided to become financially independent, I had no steady income. Frugality only slowed the rate that I went into debt, and it didn't matter how good I was at investing because I had nothing to invest. I had to focus on getting my career going first. Many people have a decent income and a good investment strategy, but they spend a lot of money, and retire comfortably at an old age. Others have a good income and are very frugal, but they don't invest that money properly. They either invest in something too risky and lose it all, or they invest in something too safe and watch inflation chew away its value.

But put all three together, and you have an amazing, even magical formula for financial independence. It astonishes many people when they see the numbers for the first time. When you put these three components together, each of them compliments the others, and the savings curve becomes exponential. When managed properly, money has the capacity to imperceptibly add up at a magnificent rate.

I've organized this book in such a way that you'll have the basic formula right away. If you really want to know how I achieved financial independence, the first three chapters are what you need to read. These first chapters are a bit dense because my goal is only to present a broad overview of all the concepts you'll need to learn about. The rest of the book fills in some of the blanks left by the first three chapters. Beyond that, I recommend you refer to the Resources section of each chapter.

Chapter 4 talks about the records you should keep, and how to use them to track your progress.

Chapter 5 discusses some of the things the workplace does for us that we need to do for ourselves in order to be financially independent.

Two words I hear a lot when I tell people about financial independence is, ``yeah, but\ldots'' These are reasons I've heard people give for why they can't achieve financial independence, or shouldn't try. Chapter 6 refutes all the ``yeah, but\ldots'' challenges I've heard over the years. Think of this as a ``frequently asked questions'' section.

My goal isn't just to teach people how to achieve financial independence, but also to find more freedom, for themselves and the rest of the world. I want this book to stand as a testament to freedom. In Chapter 7, I discuss freedom at length, what it means for different people, and what we can do to find more freedom in our lives, in ways other than financial independence.

\newpage
\section{Resources}
\begin{itemize}
\item \textbf{\emph{Your Money or Your Life: 9 Steps to Transforming Your Relationship With Money and Achieving Financial Independence} by Vicki Robin, Joe Dominguez, and Monique Tilford.} This is the granddaddy of financial independence books. This was the one that got me started. Although some of the details of the program in this book are problematic, I still consider this required reading for anyone who wants to achieve financial independence.

\item \textbf{\emph{Work Less, Live More: The New Way to Retire Early} by Bob Clyatt.} This is the best book I've found for financial independence, next to Your Money or Your Life. It fills in a lot of holes in Your Money or Your Life, such as the withdrawal rate, investing, health care, and taxes. It's also updated more frequently. This too I consider required reading.

\item \textbf{\emph{Awaken the Giant Within: How to Take Immediate Control of Your Mental, Emotional, Physical and Financial Destiny!} by Anthony Robbins.} Financial independence requires a certain attitude that is not very common. It requires a lot of creativity, determination, and awareness. These came naturally for me from my rebellious childhood and my interest in philosophy, but I later found some of what I'd discovered on my own in this book by Tony Robbins. If you can get past his excessive use of exclamation points and his tendency to commoditize common philosophical and psychological concepts, I would definitely recommend this book.

\item \textbf{\url{http://www.mrmoneymustache.com/}} -- Mr. Money Mustache is a fabulous financial independence blog, with many helpful articles.

\item \textbf{\url{http://www.madfientist.com/podcast/}} -- The Financial Indepedence Podcast, with interviews and discussions related to financial independence.
\end{itemize}
