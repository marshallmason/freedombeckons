\chapter{Open For Business}
\fancyhead[CO]{\emph{Open For Business}}
\fancyhead[CE]{Freedom Beckons}

\section{Balance sheet and cash flow}
One of the clich\'es in personal finance is that you should treat your life like a business. I used to figure this was a kind of business worship, as if business isn't just something you do to put food on the table, but a lifestyle, even a spiritual practice.

But in some ways, our lives are very similar to businesses. We have different motives, of course. Rather than a profit motive, we have a love motive, a joy motive, a safety motive, and a comfort motive. But like businesses, we trade our services for money. We have money coming into and out of our lives like a business does, and we have some amount of investments and cash on hand like a business does. But since businesses are so laser-focused on the profit motive, they have developed some very effective tools for managing those things, which we can use in our own lives.

One tool is called the \emph{balance sheet.} This is simply a list of everything you own, called \emph{assets,} and everything you owe, called \emph{liabilities.} Subtract the liabilities from the assets to come up with the balance. This represents your \emph{net worth.} You should also calculate your \emph{liquid net worth,} which you will need for your withdrawal rate calculation. This is simply your net worth minus any assets you can't or don't intend to sell in order to raise cash.

Another useful tool is the \emph{cash flow.} This is an outline of all the money coming into and going out of your life.\footnote{The cash flow as I describe it here is not quite how a business does it, but it's the same basic idea.} The first part of your cash flow is your job income. The second part is your investment income. The third part is your expenses. You will need to know your expenses for your withdrawal rate calculation.

In order to keep track of your cash flow, you will need to keep records of everything you earn and spend. Every penny. Some people find this intimidating, but I found it easy once I got used to it. I just keep receipts, and later I record the expenses. Sometimes, I'll depend on my memory, and write it down when I get the chance, but it's so easy to forget to write them down later. It's also possible to use credit card statements or bank statements, but these aren't as detailed, so it's harder to pay close attention to your trade-offs.

You can look into personal finance software that will make the balance sheet and cash flow easier for you. Another option is spreadsheet software. That requires you to organize it all and construct the formulas yourself, but it does allow for the most flexibility. If you hate technology and have plenty of time on your hands, you can also do it by hand with a pen, paper, and a calculator. I use a spreadsheet to maintain my records.

In its simplest form, a balance sheet looks something like Table~\ref{tab:balance-sheet}

\begin{table}[ht]
\caption{Balance sheet}
\label{tab:balance-sheet}
\centering
\begin{tabular}{l r}
\\\hline
\\\hline
\textbf{Assets} & \textbf{\$162,000.00}\\
\hline
Cash & \$500.00\\
Checking account & \$1,500.00\\
Savings account & \$5,000.00\\
Balanced fund & \$50,000.00\\
Roth IRA -- Balance fund & \$100,000.00\\
Car & \$5,000.00\\
\\
\textbf{Liabilities} & {\color{red} \textbf{-\$500.00}}\\
\hline
Credit card & {\color{red} \$\-500.00}\\
\\
\textbf{Total net worth} & \textbf{\$161,500.00}\\
\textbf{Liquid net worth} & \textbf{\$156,500.00}\\
\end{tabular}
\end{table}

This can become more complex if you have lots of investments or loans. You only need two columns: the name and the amount. You have an assets section and a liabilities section. List everything you own and everything you owe, of any significant value. At the bottom, there are two calculations. The first is your total net worth. Simply subtract your total liabilities from your total assets. Then calculate your liquid net worth. In this example, I deducted the value of the car from the net worth to get the liquid net worth.

\begin{table}[ht]
\caption{Cash flow}
\label{tab:cash-flow}
\centering
\begin{tabular}{l r l}
\\\hline
\\\hline
\textbf{Income} & \textbf{\$3,000.00}\\
\hline
09/01/2015 & \$3,000.00 & Net salary\\
\\
\textbf{Interest} & \textbf{\$910.00}\\
\hline
09/05/2015 & \$10.00 & Savings account\\
09/30/2015 & \$300.00 & Balanced fund appreciation\\
09/30/2015 & \$600.00 & IRA appreciation\\
\\
\textbf{Expenses} & \textbf{{\color{red} \-\$1,084.49}}\\
\hline
09/01/2015 & {\color{red} \-\$600.00} & September rent\\
{\color{blue} 09/02/2015} & {\color{red} \-\$1.99} & {\color{blue} Soda}\\
{\color{blue} 09/03/2015} & {\color{red} \-\$25.00} & {\color{blue} Dinner with friends}\\
09/04/2015 & {\color{red} \-\$150.00} & DVD player\\
09/05/2015 & {\color{red} \-\$300.00} & Clothes\\
09/30/2015 & {\color{red} \-\$7.50} & Brokerage fee\\
\hline
\end{tabular}
\end{table}

Table~\ref{tab:cash-flow} shows an example of cash flow for one month. There are three sections. In the income section, list all of your paychecks and tips. This section is only relevant while you have a job. The second section is your investment income. List the interest you earn in your accounts, the growth or losses of your investments, and the dividends you receive. The third section is your expenses, the most important of the three. List here all the money you spend. This example only shows a few days of expenses, but you'll have an entire month of expenses in this section. 

The first column is the date of the transaction. The second column is the total amount of the transaction. The third column is the name of the transaction. Due to limited space, I could not show all possible columns, but there are some other columns you may wish to add.

You might find it useful to create a column for the category of the transaction. If you have detailed records for your expenses, you may want to get a summary grouped by categories such as rent, transportation, food, etc. It's up to you how you categorize your expenses. For example, some people like to have two categories for transportation: one for their automobile, and one for everything else.

I also find it indispensible to have a few columns to represent \emph{amortization.} This is a fancy word, but all it means is that you distribute the amount of a transaction over its entire lifetime. Most things you buy aren't monthly expenses. Without amortization, they will disappear from future month expense statements, even though you're still getting value from it. For example, one of my utility bills comes every other month. I could put the entire bill on one month's spreadsheet, and then nothing the next, but the more accurate way to do it is to put half this month and half next month. You could choose to amortize only your big costs, but I do it for every transaction with a lifetime of two months or more.

When I first started maintaining my cash flow, I didn't use amortization, and it wreaked havoc on my planning. One month, my costs would be extremely low, and I'd pat myself on the back. The next month, my costs would shoot through the roof, due to a large expense. It was impossible for me to calculate my withdrawal rate with any hope of accuracy.

To do this, I've added three extra columns, which I did not show in the table due to limited space. One is the \emph{amortized value.} This is the total cost divided by its lifetime. For example, if the utility bill that comes every other month was for \$100, the amortized value is \$50. This column is calculated automatically from the other two columns I've added, representing the lifetime. One column is the date of the expense and the other is the end of the lifetime. For the bi-monthly utility bill, I would put today's date in one column and next month in the other column. For some expenses, this is only an estimate, but I find that even an inaccurate amortization is better than no amortization at all.

There are some hidden expenses you should remember to include in your Expenses section. One is brokerage and investment fees, because the withdrawal rate calculations don't account for these. Brokerage fees are more obvious because you have to actually pay them when you do your trading. Mutual fund fees are a lot more hidden. Mutual fund companies will deduct management fees from your fund each year unless you agree to send them a check or give them some other source of payment. Management fees in ETFs are even more hidden because they're actually deducted from the share price. It's not very clear when the deductions take place, or for how much. What I do is divide the expense ratio by 12 months, and multiply that by the ETF balance in my Expenses each month.

Another hidden expense you need to be aware of is car depreciation, if you own one. Cars lose a significant amount of value over time, and this really is an expense. You spend money just by owning and driving the car.

I use colors in the spreadsheet as a mnemonic. You can get as creative with this as you like. Whatever makes your spreadsheet easier for you to read and understand. My spreadsheet shows the negative values as red. I also use blue for every row that isn't a recurring transaction. Every transaction that is amortized will not be highlighted, but I also don't highlight recurring, monthly transactions, such as utility payments. When I make the next month's spreadsheet, I simply copy all of the black rows from the previous month, and update the date, until it expires.

These spreadsheets are for your own records only. Spreadsheets are very flexible tools. Put anything there that you think will help you reach your goal. As you get more practice with them, you may think of even more uses for them, and you'll become more efficient when you update them.

\section{Regular updates}
Many businesses update their public records quarterly (every three months). You could do it every week, every year, or anything in between. I do mine every month, so that's what I'll use in my examples. If you do update your balance sheet and cash flow infrequently, it will require less work and you'll see less volatility in your investments, but you'll also be working with outdated information much of the time. More frequency wins you better information upon which to base your trade-offs.

I keep track of every transaction I make. I keep a record of all the money that comes into or goes out of my life. I've found that you can be as specific as you want. For example, when you go grocery shopping, you could just record the whole trip as ``Food'' or you may record every transaction separately. Being more specific takes more work, but provides you with very detailed records, which I personally find extremely useful. The first spreadsheet I did, in 2003, was very sparse and simple compared to my modern spreadsheets. So if you want to start with the easier, less detailed record keeping, it's fine if you decide to change it later.

Most spreadsheets allow you to create multiple sheets in one document. I use this feature to keep all spreadsheets related to the balance sheet in one document, and all spreadsheets related to the cash flow for the same year in one annual cash flow document. Each month, you can make a copy of last month's sheet, and then work with that, so you have monthly records of your trends, or you can simply overwrite last month's sheet each month. I find it very helpful to have records of prior months' cash flows, and not so helpful to keep prior month's balance sheets.

The first thing I do each month is balance my checkbook. I look at my bank records, check off everything in my checkbook that coincides, and correct any inconsistencies. This ensures I always have an accurate checkbook.

Next, I update my balance sheet. I count all of the cash I have on hand, and look at my bank records, brokerage statements, and credit card statements (most banks make these available online as well), and update my balance sheet to reflect them. This gives me a new net worth and liquid net worth. I keep records of the monthly values, and I construct a graph that shows these trends over time.

In my balance sheet document, I also maintain a spreadsheet for my asset allocation. This shows the percentages of each asset compared to their target allocations, which will get skewed over time. Whenever I re-balance my portfolio, I use this as a guide for how much I need to buy and sell of each asset.

Next, I update my cash flow. When I was working, I would look at my paycheck each month to update my job income section. Then I look at my bank statements, brokerage statements, and credit card statements, and list any interest I pay or earn in my investment income section. I also list the growth of my investments, by comparing this month's value with last month's. Since I do not re-invest my dividends, I also include here any dividends my investments earned.

Finally, I finish updating my expenses for the month. When I'm done, I have a grand total of job income, investment income, and expenses. I construct a graph for the trends here as well. The big pay-off came after years of working with these spreadsheets and I could see the trends of my job income, investment income, and net worth trending upward as my expenses drop.

\section{Paying yourself first}
``Paying yourself first'' is a common phrase among financial planners. What it means is that your first priority should be your future: you should deduct from your income all the payments for saving for retirement, and then use whatever is left over to pay your bills and other incidentals.

Many people use their income as a gauge for how much they have to spend each month. There is a psychological tendency to spend what we earn, whatever that is. As earnings increase, so does our appetite for spending. Paying yourself first short-circuits this tendency, so that the only earnings you see, and therefore believe is available for spending, has already accounted for what you need to reach your goals.

However, I've always thought there was a big problem with this philosophy: most of what people spend \emph{is} for themselves, unless they are philanthropists. Whether they buy food, pay their utilities, or buy something that brings them joy, that money is for themselves, so spending this money \emph{is} paying themselves first. Just because they're giving their money to someone else doesn't mean they're paying someone else before themselves.

The above technique that financial planners recommend assumes that you can \emph{afford} to pay off your debt and invest for retirement. When I started on my goal, I was nearly starving. At one point, I had enough money to either buy oil for my motorcycle or food for my belly, and I chose my belly. I still had to get to work, so I drove my motorcycle without oil and destroyed the engine. This wasn't stupidity. It was desperation. Choosing to eat rather than paying off debts was what ``paying myself first'' looked like for me.

The problem with the strategy of investing for your future first is that it treats this as the top priority, and it obviously isn't. You shouldn't deny this in order to play some sort of psychological trick on yourself. As I mentioned in Chapter 2, you should make your choices consciously, so you can make the right trade-offs at all times. If you do this, then you don't need to trick yourself into investing for your future. You will always know which choice is the most important to you, whether it be for your current needs or future goals.

What, then, should be your priorities? Here is a summary of the entire process of achieving financial independence, in order of priority.

Your first priority, obviously, should be your survival and the survival of your loved ones. Once your basic needs are met, then you must decide what's most important for you. You may have some desires that are more important to you than freedom. Start with that first. That's what paying yourself looks like for you.

Once you have your basic survival needs met, and a few desires that bring you immense joy that is worth more to you than your freedom, only then is it time to start thinking about financial independence. That would be a good time to create the balance sheet and cash flow spreadsheets so you can keep track of your progress toward this goal.

The first part of this you should focus on is increasing your income and advancing in your career, the subject of Chapter 1. You need money to achieve financial independence, and your job is your source of money, so you need to think about how you can make more money. Make a plan for your income growth.

Next, you need to drop your expenses, the subject of Chapter 2. I say this is the second step, but you can start on it right away. Also, unlike your income, there is no glass ceiling, and unlike investing, you're not done once you've decided on a strategy. The process of dropping your expenses is on-going. Keeping expenses low is the backbone of financial independence.

After you've made a plan to increase your income, and you started working on dropping your expenses, you should start noticing some nice trends in your balance sheet and cash flow. Your net worth should be going up, your income should be going up, and your expenses should be dropping. Once you start spending less than you earn, you will need a place to put all that extra cash, the subject of Chapter 3.

If you have debt, this will probably be the most logical place to start. I recommend consolidating your debt as much as possible, and then just chug through it. It can help to create an amortization schedule for your loan payments. There are online tools to help you do this. I've included one in the Resources. It will show you how much interest and principal you are paying with each payment, and how many such payments you will need to make to pay off your loan. This can help put your debts into perspective.

Pay your debts off before you start investing, except for loans with really low rates. Start paying off debts with the highest rates first. Think of your loans as high-interest, risk-free savings accounts. You'd want to invest in the highest interest account first. Once you're left with only low-rate loans (less than, say, 5\%), you might be better off just paying the minimum payment and investing the rest.

Before you start creating a portfolio of stocks and bonds, prepare for emergencies and unforeseen circumstances like losing your job. You should have a certain amount in a savings account for such circumstances. Some people feel fine with only a month or two worth of living expenses. I recommend at least two months. I chose six months. It felt really good that I could be out of work for a half a year and still make ends meet. Once you do your cashflow spreadsheet, you should know your monthly expenses. Just multiply that number by however many months you choose.

Now you're ready to start constructing your portfolio. This is where you really need Chapter 3. You should start by taking advantage of every tax shelter available to you, up to the limits imposed by the government. Probably the best place to start is with a 401(k) or 403(b), if your employer offers one. 401(k) is not only the most important tax shelter---it's also the easiest. It's deducted directly from your paycheck pre-tax, so you don't need to make any changes on your tax return. Your tax payments are automatically adjusted accordingly, so the impact is huge while the effect on your bottom line is not as noticeable.

Once you've contributed to your 401(k) to the maximum allowed by law, then invest in an IRA or Roth IRA, as I mentioned in Chapter 3. The limits for these are much lower than the 401(k). Contribute to an IRA or Roth IRA to the maximum allowed by law.

Once you've maxed out your 401(k) and IRA or Roth IRA, you can start investing your money outside of a tax shelter. If you do this, look into tax-managed funds, as I mentioned in Chapter 3.

\newpage
\section{Resources}
\begin{itemize}
\item \textbf{\url{http://www.edmunds.com/used-cars/}} -- Gives you the true market value of a used car, which you can use to estimate depreciation.
\item \textbf{\url{http://www.amortization-calc.com/}} -- An online calculator for creating an amortization schedule for your loans.
\end{itemize}
