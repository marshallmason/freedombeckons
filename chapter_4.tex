\chapter{Open For Business}

\section{Balance sheet and cash flow}
One of the clich\'e~s in personal finance is that you should treat your life like a business. Don't we give enough of ourselves over to business as it is? Now our whole life should be a business? But in some ways, our lives are very similar to businesses. We have different motives, of course. Rather than a profit motive, we have a love motive, a joy motive, a safety motive, and a comfort motive. But like businesses, we trade our services for money. We have money coming into and out of our lives like a business does, and we have some amount of investments and cash on-hand like a business does. But since businesses are so laser-focused on the profit motive, they have developed some very effective tools for managing those things, which we can use in our own lives.

One tool is called a \emph{balance sheet.} This is simply a list of everything you own, called \emph{assets,} and everything you owe, called \emph{liabilities.} Subtract the latter from the former to come up with the balance. This represents your \emph{net worth.} You should also calculate your \emph{liquid net worth,} which you will need for your withdrawal rate calculation. This is simply your net worth minus any assets you can't or don't intend to sell in order to raise cash and any related liabilities.

Another useful tool is the \emph{cash flow.} This is an outline of all the money coming into and out of your life. The first part of your cash flow is your job income. The second part is your investment income. The third part is your expenses. You will need your expenses for your withdrawal rate calculation.

In order to keep track of your cash flow, you will need to keep records of everything you earn and spend. Every penny. Some people find this intimidating, but I found it easy and second-nature once I got used to it. Sometimes I keep a little note pad with me and whenever I spend money, I make a note of it. When I get receipts, I can wait until later that day to record the expenses. Sometimes, I'll depend on my memory, and write it down when I get the chance, but I don't recommend that, since it's so easy to forget to write them down later. Missing any expenses will mean that your records will be off, and your withdrawal rate will be inaccurate.


You can look into personal finance software that will make the balance sheet and cash flow easier for you. Another option is spreadsheet software. That requires you to organize it all and construct the formulas yourself, but it does allow for the most flexibility. If you hate technology and have plenty of time on your hands, you can also do it by hand with a pen, paper, and a calculator. I personally use a spreadsheet to maintain my records, so that is what I will present here.

In its simplest form, a balance sheet looks something like this:

\begin{table}[ht]
\caption{Balance sheet}
\centering
\begin{tabular}{l r}
\\\hline
\\\hline
\textbf{Assets} & \textbf{\$162,000.00}\\
\hline
Cash & \$500.00\\
Checking account & \$1,500.00\\
Savings account & \$5,000.00\\
Balanced fund & \$50,000.00\\
Roth IRA -- Balance fund & \$100,000.00\\
Car & \$5,000.00\\
\\
\textbf{Liabilities} & {\color{red} \textbf{-\$500.00}}\\
\hline
Credit card & {\color{red} \$\-500.00}\\
\\
\textbf{Total net worth} & \textbf{\$161,500.00}\\
\textbf{Liquid net worth} & \textbf{\$156,500.00}\\
\end{tabular}
\end{table}

This can become more complex if you have lots of investments or loans. You only need two columns: the name and the amount. You have an assets section and a liability section. List everything you own and everything you owe, of any significant value. At the bottom, there are two calculations. The first is your total net worth. Simply subtract your total liabilities from your total assets. Then calculate your liquid net worth, which is your total net worth minus any assets you wouldn't want to sell to raise cash. In this example, I deducted the value of the car from the net worth to get the liquid net worth. That's really all you need.

Here is an example of cash flow for one month:

\begin{table}[ht]
\caption{Cash flow}
\centering
\begin{tabular}{l r l l r l}
\\\hline
\\\hline
\textbf{Income} & \textbf{\$3,000.00}\\
\hline
09/01/2011 & \$3,000.00 & & & \$3,000.00 & Net salary\\
\\
\textbf{Interest} & \textbf{\$902.50}\\
\hline
09/05/2011 & \$10.00 & & & \$10.00 & Savings account\\
09/30/2011 & \$300.00 & & & \$300.00 & Balanced fund appreciation\\
09/30/2011 & \$600.00 & & & \$600.00 & IRA appreciation\\
09/30/2011 & {\color{red} \-\$7.50} & & & {\color{red} \-\$7.50}\\
\\
\textbf{Expenses} & \textbf{{\color{red} \-\$2,500.00}}\\
\hline
09/01/2011 & {\color{red} \-\$600.00} & & & {\color{red} \-\$600.00} September rent\\
{\color{blue} 09/02/2011} & {\color{red} \-\$1.99} & & & {\color{red} \-\$1.99} & {\color{blue} Soda}\\
{\color{blue} 09/03/2011} & {\color{red} \-\$25.00} & & & {\color{red} \-\$25.00} & {\color{blue} Dinner with friends}\\
09/04/2011 & {\color{red} \-\$4.17} & 08/2010 & 07/2013 & {\color{red} \-\$150.00} & DVD player\\
09/05/2011 & {\color{red} \-\$8.33} & 10/2010 & 09/2013 & {\color{red} \-\$300.00} & Clothes\\
\hline
\end{tabular}
\end{table}

There are three sections. In the income section, list all of your paychecks and tips. This section is only relevant while you have a job. The second section is your investment income. List the interest you earn in your accounts, the growth or losses of your investments, and the dividends you receive. The third section is your expenses, the most important of the three. List here all the money you spend. This example only shows a few days of expenses, but you'll have an entire month of expenses in this section. 

You may choose to have a fourth section to summarize expense categories. If you have detailed records for your expenses, you may want to combine those expenses into categories like rent, transportation, food, etc. It's up to you how you categorize your expenses. For example, some people like to have separate categories for transportation, one their automobile, and one for everything else. If you know how to make charts in your spreadsheet program, you might also want to put a pie chart to give a pictoral representation of your spending patterns.

The first column is the date of the transaction. The second column is the \emph{amortized value.} The third and fourth columns represent the \emph{lifetime} of the transaction. The fifth column is the total amount of the transaction. The sixth column is the name of the transaction. You could also add a seventh column for the category for this transaction, if that helps you. I didn't include this here, due to the limited space on the page.

Every transaction has a lifetime. For example, the rent you pay has a life of one month, since you have to pay it every month, whereas clothes can have a lifetime of many years. Amortization is the process of distributing the amount of a transaction over its entire lifetime. To do this, put the date of the transaction in the third column, and the date of the end of its lifetime in the fourth column. This is only an estimate, for your own needs. I find that even an inaccurate amortization is better than no amortization at all.

I find amortization extremely important. When I first started maintaining my cash flow, I didn't use it, and it wreaked havoc on my planning. One month, my costs would be extremely low, and I'd pat myself on the back. The next month, my costs would shoot through the roof, because of something big that I bought. It was impossible for me to calculate my withdrawal rate with any hope of accuracy whatsoever. You could choose to amortize just your big costs, but I do it for every transaction with a lifetime of two months or more.

The second column uses the third through fifth columns to calculate the amortized amount. The mathematics for this are complex enough to make it beyond the scope of this book. The formula needs to calculate the amortization for that row if it sees that there is a lifetime in the third and fourth columns. Otherwise, it will just reflect the amount in the fifth column, unaltered. This way, you can choose which transactions you'd like to amortize.

One question I had to answer when I started maintaining my cash flow was where I listed job-related expenses and investment expenses. For example, income taxes for my job are job-related expenses and broker fees are investment expenses. Should I list them under Expenses, or as offsets under job income and investment income? I decided to put all job-related expenses as deductions in the job income section. These are all costs that will go away once I was retired, so I didn't want to account for it in my withdrawal rate. However, I put investment expenses in the expenses section, not in the investment income section. The reason is that my target withdrawal rate is not adjusted for investment costs, not even taxes. As I mentioned in Chapter 3, I don't pay taxes, but I list my broker fees, as well as the expense ratio for my mutual funds, in my expense section.

There are some hidden expenses you should remember to include in your Expenses section. One is broker and investment fees. The withdrawal rate calculators don't account for these. Brokerage fees are more obvious, because you have to actually pay them when you do your trading. Mutual fund fees are a lot more hidden. Mutual fund companies will deduct management fees from your fund each year unless you agree to send them a check or give them some other source of payment. Management fees in ETFs are even more hidden, because they're actually deducted from the share price. It's not very clear when the deductions take place or for how much. What I do is divide the expense ratio by 12 months, and multiply that by the ETF balance in my Expenses each month. Also remember to include any other bank or credit card fees you pay.

Another hidden expense you need to be aware of is car depreciation, if you own one. Cars lose a significant amount of value over time, and this really is an expense. Think of it like a savings account that works in reverse--it depreciates over time rather than appreciating. When you close the account (sell the car), it will have less money than when you opened it, without spending any of it. Or, more accurately, you spend the money just by owning and driving the car. I use Edmunds.com to estimate to value of my car. Estimating my depreciation was a source of difficulty for me, because it can vary wildly. Some months, it would lose thousands of dollars, and then some months it would stay even. It just depends on the supply and demand that exists for that car at that time and place. What I do is maintain a \emph{moving average} over two years. This means that the average moves forward by one month each month. I keep track of its value each month, and on my Expenses I list the average change over the last 24 months.

I use colors in the spreadsheet as a mnemonic. You can get as creative with this as you'd like. Whatever makes your spreadsheet easier for you to read and understand. My spreadsheet colors all negative values red. I also color blue every row that isn't a recurring transaction. Obviously, every transaction that is amortized will not be highlighted, but I also don't highlight recurring, monthly transactions like utility payments. I put this to very practical use when I update my cash flow each month, which I will explain next.

These spreadsheets are for your own records only. Spreadsheets are very flexible tools. Put anything there that you think will help you reach your goal. As you get more practice with them, you may think of even more to use the spreadsheets for, and you'll become more efficient when you update them.

\section{Regular updates}
Something you must decide is how frequently you want to update your balance sheet and cash flow. The less you do it, the less work you'll have, and the less volatility you will see in your investments, but you'll also be working with outdated information much of the time. Many businesses update theirs quarterly (every three months). You could do it every week, every year, or anything in between. I do mine every month. I will assume here that you do it every month, so if you choose a different frequency, adjust what I say here accordingly.

Each month, keep track of every transaction you make. Keep a record of all the money that comes into or goes out of your life. Your records can be as specific as you'd like. For example, when you go grocery shopping, you could just record the whole trip as ``Food'' or you may record every transaction separately. Each choice has a trade-off. Being more specific is more toilsome, but provides you with very detailed records, which I personally find extremely useful and very worth it. It's also fine to change from month to month, so you could start with the easier, less detailed record keeping, and then in later months you could be more specific. The example spreadsheet is not very detailed, to make it more compact and easier to understand, but I'm pretty detailed in my own spreadsheets.

You could use your credit card statements, bank statements, and receipts, if that's easier for you. I personally find a stack of receipts and statements overwhelming, so I add my expenses to my spreadsheet at the end of each day, from receipts, memory, or a note pad I wrote them down on. At the end of the month, I have it all there, and I just need to add it all up and evaluate it. You could update every week, or month, or whenever you have the time. This is all up to you, whatever makes the most sense for you personally. All that matters is that you don't miss any expenses..

Most spreadsheets have a feature that allows you to create multiple sheets in one document. I use this feature to keep all spreadsheets related to the balance sheet in one balance sheet document, and all spreadsheets related to the cash flow in one cash flow document. Each month, you can make a copy of last month's sheet, and then work with that, so you have monthly records of your trends, or you can simply overwrite last month's sheet each month. I find it very helpful to have records of prior months' cash flows.

The first thing I recommend you do each month is balance your checkbook. Look at your bank records, check off everything in my checkbook that coincides, and correct any inconsistencies. This ensures you always have an accurate checkbook.

Next, update your balance sheet. Count all of the cash you have on hand, and look at your bank records, brokerage statements, and credit card statements (most banks make these available online as well), and update your balance sheet to reflect them. This gives you a new net worth and liquid net worth. You may want to keep records of the monthly values, and maybe even construct a graph that shows these trends over time. As you work with this book, the overall trend should be upward, with some occasional drops as your investments cycle.

In your balance sheet document, you may also want to maintain a spreadsheet for your asset allocation. This shows the percentages of each asset compared to their target allocations, which will get skewed over time. Whenever you re-balance your portfolio, you can use this as a guide for how much you need to buy and sell of each asset.

Next, update your cash flow. Look at your paycheck each month to update your job income section. Then look at your bank statements, brokerage statements, and credit card statements, and list any interest you pay or earn in your investment income. Also list the growth of your investments, by comparing this month's value with last month's. Be sure to include any dividends your investments earn, unless you're having them re-invested.

Then update all of your expenses for the month, if you haven't already. For each expense, estimate the lifetime of the expense, and for any that are longer than one month, put its lifetime in the third and fourth columns. You'll need to carry over any amortized expenses from last month's cash flow, which is why I highlight with blue any transactions that aren't recurring. This way, I know that all of the black lines need to be carried over and updated from month to month.

When you're done, you should have a grand total of job income, investment income, and expenses. You may want to construct a graph for the trends here as well. As you work through the advice in this book, you should see your job income and investment income trend upward and your expenses drop over time.

\section{Paying yourself first}
There is a phrase common among financial planners: ``pay yourself first.'' What this means is that your first priority should be planning for your future, that you should deduct from your income all the payments for paying off your debt and saving for retirement, and then use whatever is left to pay your bills. The reason for this is that most people use their income as a gauge for how much they have to spend each month. There is a psychological tendency for people to spend what they earn, whatever that is. As their earnings increase, so does their appetite for spending. Paying yourself first short-circuits this tendency, so that the only earnings you see and therefore believe is available for spending has already accounted for what you need to reach your goals.

However, I see one problem with this otherwise sound philosophy: most of what people spend is for themselves, unless they are philanthropists. If people buy food, or pay their utilities, or buy something that brings them joy, that money is for themselves, so spending this money is paying themselves first. Just because they're giving their money to someone else doesn't mean they're paying someone else before themselves. The technique that financial planners recommend assumes that you can afford to pay off your debt and invest for retirement. When I started on my goal, I was nearly starving. I went almost a week once without food because I simply couldn't afford it. Finally, I had enough money to either buy oil for my motorcycle or food for my belly, so I chose my belly. I still had to get to work, so I drove my motorcycle without oil and destroyed the engine. This wasn't stupidity. It was desperation. Choosing to eat rather than paying off debts was what ``paying myself first'' looked like for me.

The problem with the strategy of investing for your future first is that it treats this as the top priority, and it obviously isn't. Denying this in order to play some sort of psychological trick on yourself is not what I recommend. As I mentioned in Chapter 2, I recommend that you make all of your choices consciously, knowing all of your trade-offs, so you can make the right trade-offs for you at all times. If you do this, then you don't need to trick yourself into investing for your future. You will always know which choice is the most important to you, whether it be for your current needs or future goals. You will be aware of your priorities and values, and you will make that choice consciously.

Your first priority, obviously, should be your survival and the survival of your loved ones. You can use Chapter 1 to increase your income and Chapter 2 to get more value for each expense you have, so you can stretch each dollar as much as possible to meet your survival needs. Just make sure you distinguish between needs and wants. People need food, shelter, clothing, warmth, security, and transportation, but there are cheap ways to get all of these. You need shelter, but you don't need an extra bedroom to store your extra stuff. That's a want. You need transportation to get to work, but you don't need a brand new SUV. That's a want. You need a phone for job hunting, but you don't need the latest smart phone. That's a want.

Once you have your basic needs met, then you must decide what's most important for you. You may have some desires that are more important to you than freedom. I recommend that you make sure you have enough money for these before you start investing for your future. That's what paying yourself looks like for you. What's the point of freedom if you have to be miserable along the way? Just make sure you are very honest with yourself about how much joy these things bring you, rather than just buying on impulse anything you think might bring you joy. This whole book is about being aware of your values and being honest with yourself.

Once you have your basic survival needs met, and a few desires that bring you immense joy that is worth more to you than your freedom, only then is it time to start thinking about financial independence. That would be a good time to create the balance sheet and cash flow spreadsheets so you can keep track of your progress toward this goal.

The first thing you should think long and hard about is increasing your income, and advancing in your career, the subject of Chapter 1. You need money to achieve financial independence, and your job is your source of money, so you need to think about how you can make more money. Make a plan for your income growth.

Next, you need to drop your expenses, the subject of Chapter 2. This is the second step, but overall it is the most important of all the steps because unlike your income, there is no glass ceiling, and unlike investing, you're not done once you've decided on a strategy. The process of dropping your expenses is on-going. Keeping expenses low is the backbone of financial independence.

After you've made a plan to increase your income, and you started working on dropping your expenses, you should start noticing some nice trends in your balance sheet and cash flow. Your net worth should be going up, your income should be going up, and your expenses should be dropping. Once you start spending less than you earn, you will need a place to put all this extra cash, the subject of Chapter 3.

If you have debt, this will probably be the most logical place to start. I recommend consolidating your debt as much as possible. This can make it feel more manageable. If you have credit card balances, I'd suggest finding a credit card with as low a rate as possible, and moving your other balances to that one card. Then cancel your other cards. Once you're done paying off the low rate card, then cancel that as well, unless you're sure you can trust yourself not to accumulate a balance on it anymore. Credit card companies make a huge profit on the likelihood that you will accumulate a balance again, so don't play their game unless you know you can win it.

I recommend that you create an amortization schedule for your loan payments. There are online tools to help you do this. I've included one in the Resources. It will show you how much interest and principal you are paying with each payment, and how many such payments you will need to make to pay off your loan.

I also recommend that you pay your debts off before you start investing, except for loans with really low rates. You should take advantage of the investments that earn the highest rates first. Think of your loans as high-interest, risk-free savings accounts. You'd want invest in the highest interest account first. Once you're left with only low-rate loans, you might be better off just paying the minimum payment and investing the rest.

Before you start creating a portfolio of stocks and bonds, I recommend you first prepare for emergencies and unforeseen circumstances like losing your job. You should have a certain amount in a savings account for such circumstances. How much is up to you. It depends on how much you need to feel secure. Some people feel fine with only a month or two worth of living expenses. I recommend at least two months. I chose six months. It felt really good that I could be out of work for a half a year and still make ends meet. Once you do your cashflow, you should know your monthly expenses. Just multiply that number by however many months you choose.

Now you're ready to start constructing your portfolio. This is where you really need Chapter 3. As I mentioned in that chapter, you should start by taking advantage of every tax shelter available to you, up to the limits imposed by the government. Probably the best place to start is with a 401(k) or 403(b), if your employer offers one. Many employers offer 401(k) matching, which means that they will contribute extra to your 401(k) if you contribute to it too. In other words, free money! Another reason a 401(k) is good is because it's easy. It's deducted directly from your paycheck pre-tax, so you don't need to make any changes on your tax return. These features make the 401(k) really cool because the impact is huge while the effect on your bottom line is not as noticeable. You may have a certain amount deducted from your paycheck, but your paycheck is decreased by a smaller amount because you have to deduct less taxes.

Once you've contributed to your 401(k) to the maximum allowed by law, then invest in an IRA or Roth IRA, as I mentioned in Chapter 3. The limits for these are much lower than the 401(k). Contribute to an IRA or Roth IRA to the maximum allowed by law.

Once you've maxed out your 401(k) and IRA or Roth IRA, you can start investing your money outside of a tax shelter. If you do this, I suggest you look into tax-managed funds, as I mentioned in Chapter 3.

You may still want to employ a trick financial planners recommend for “paying yourself first” which is to make deductions directly from your paychecks, if your employer supports this. It may be possible to directly deposit portions of your paycheck into your credit card accounts or investment accounts. In the case of the 401(k), this is done automatically. If you don't trust yourself to prioritize properly, or if you want to ensure that some portion of your income always goes toward investing, this might be a good strategy for you. Just don't do anything that feels like deprivation, because that is \emph{not} paying yourself first.

\newpage
\section{Resources}
\begin{itemize}
\item \textbf{\url{http://www.moneychimp.com/calculator/compound_interest_calculator.htm}} -- An online compound interest calculator.
\item \textbf{\url{http://www.edmunds.com/used-cars/}} -- Gives you the true market value of a used car, which you can use to estimate depreciation.
\item \textbf{\url{http://www.amortization-calc.com/}} -- An online calculator for creating an amortization schedule for your loans.
\end{itemize}
